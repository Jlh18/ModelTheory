\section{Noetherian Modules}
All of the following material is from Atiyah and McDonald's book \cite{atiyah}.
\begin{dfn}[Ascending chain]
    If $(\Si,\leq)$ is a partially ordered set then an ascending chain 
    is a functor $F : (\N,\leq) \to (\Si,\leq)$.

    A chain is stationary if there exists a $m \in \N$ such that
    for any $n \in \N_{\ge m}$, $F(m) = F(n)$.
\end{dfn}

\begin{prop}[Noetherian chains]
    \link{noeth_chain}
    Let $\Si$ be a partially ordered set. 
    Then every non-empty subset has a maximal element
    if and only if every ascending chain is stationary.
\end{prop}
\begin{proof}
    \begin{forward}
        If $F$ is a chain then $I = \set{F(i) \st i \in \N}$
        has a maximal element $F(m)$. 
        Hence for any $n$ greater than or equal to $m$, 
        $F(m) = F(n)$.
    \end{forward}
    \begin{backward}
        Let $\nothing \ne I \subs \Si$ and
        suppose for a contradiction that for each $m \in I$
        there exists $a \in I$ such that $m < a$.
        Then create a chain by recursively taking such an $a$,
        starting with the non-empty element:
        $a_0 < a_1 < \dots$.
        This chain is not stationary.
    \end{backward}
\end{proof}

\begin{dfn}[Noetherian Module]
    \link{noetherian_definition}
    Order the submodules of an $A$-module $M$ by $\subs$.
    $M$ is Noetherian if it satisfies on one of the following
    equivalent definitions:
    \begin{itemize}
        \item Every non-empty set of submodules has a maximal element.
        \item Every ascending chain of submodules is stationary.
        \item Any submodule of $M$ is finitely generated.
    \end{itemize}
\end{dfn}
\begin{proof}
    $1. \iff 2.$ follows from \linkto{noeth_chain}{Noetherian chains}.
    $1. \implies 3.$: Let $N \leq M$ and take the set 
    \[I = \set{S \leq N \st S \text{ finitely generated}}\]
    Then $0 \in I$ hence it is non-empty hence has a maximal element,
    which we call $N_0$.
    Suppose for a contradiction that $N_0 < N$. 
    Then there exists $x \in N \setminus N_0$ such that $N_0 < N_0 + Ax \leq N$.
    This is finitely generated hence belongs to $I$, 
    contradicting with the maximality of $N_0$.
    $3. \implies 2.$: Let $N_0 \leq N_1 \leq \dots$ 
    be a chain of submodules.
    Their union $N$ is a submodule of $M$ hence is finitely generated.
    There exists $N_m$ such that all the generators are in $N_m$.
    At $m$ the chain becomes stationary.
\end{proof}

\begin{prop}[Exactness and Noetherian modules]
    \link{exact_noeth}
    If 
    \[0 \to N \to M \to P \to 0\]
    is exact then 
    $M$ is Noetherian if and only if both $N$ and $P$ are Noetherian.
\end{prop}
\begin{proof}
    \begin{forward}
        If $M$ is Noetherian then it is finitely generated,
        hence the image of $N$ is finitely generated and so $N$ is 
        finitely generated.
        Furthermore any element in $P$ has a preimage in $M$ which can 
        be written as a linear combination of the generators.
        Thus the image of the generators finitely generate $P$.
    \end{forward}
    \begin{backward}
        Let $(M_i)_{i \in \N}$ be an ascending chain of submodules in $M$.
        Its preimage $(N_i)$ in $N$ and its image 
        $(P_i)$ in $P$ become stationary at 
        $m_n, m_p$.
        We claim that $(N_i)$ becomes stationary at the maximum of the two.
        We first show that 
        \[N_i \to M_i \to P_i\]
        is exact.
        Indeed 
        \begin{align*}
            &\ker (M_i \to P_i) \\
            &= M_i \cap \ker (M \to P)\\
            &= M_i \cap \Im (N \to M)\\
            &= \Im (N_i \to M_i)
        \end{align*}
        We apply the snake lemma to
        \begin{cd}
            N_i \ar[r]\ar[d] &M_i \ar[r] \ar[d]&P_i\ar[d]\\
            N_{i+1} \ar[r]  &M_{i+1} \ar[r]  &P_{i+1}
        \end{cd}
        Obtaining the exact sequences
        \[0 = \ker (N_i \to N_{i+1}) \to \ker (M_i \to M_{i+1}) 
        \to \ker (P_i \to P_{i+1}) = 0\]
        \[0 = \coker (N_i \to N_{i+1}) \to \coker (M_i \to M_{i+1}) 
        \to \coker (P_i \to P_{i+1}) = 0\]
        Noting that by stability the end points are $0$.
        Hence by exactness $M_i \to M_{i+1}$ is an isomorphism (via $\subs$)
        and $M_i = M_{i+1}$.
    \end{backward}
\end{proof}

\begin{prop}[Direct sum of Noetherian is Noetherian]
    \link{direct_sum_of_noeth}
    If $M_i$ are Noetherian $A$-modules then the direct sum 
    $\bigoplus_{i = 1}^n M_i$ is Noetherian.
\end{prop}
\begin{proof}
    Induction. Consider the exact sequence
    \[0 \to M_n \to \bigoplus_{i = 1}^n M_i \to 
    \bigoplus_{i = 1}^{n-1} M_i \to 0\]
    Apply \linkto{exact_noeth}{the proposition on exact sequences}.
\end{proof}

\begin{dfn}[Noetherian ring]
    A ring is Noetherian if it is a Noetherian module over itself.
\end{dfn}

\begin{prop}{Finitely generated modules of Noetherian rings}
    If $A$ is a Noetherian ring then any finitely generated $A$-module is 
    Noetherian.
\end{prop}
\begin{proof}
    $M$ is the image of a projection of $A^n$ for some $n$.
    \linkto{exact_noeth}{It suffices that $A^n$ is Noetherian} by considering
    the exact sequence
    \[0 \to \ker \to A^n \to M \to 0\]
    \linkto{direct_sum_of_noeth}{$A^n$ is a direct sum of Noetherian
        modules hence is Noetherian.}
\end{proof}

%\begin{prop}[Projections of Noetherian rings are Noetherian]
%    If $A$ is Noetherian and $\pi$ is a ring surjection $A \to B$
%    then $B$ is Noetherian (as a ring).
%\end{prop}
%\begin{proof}
%    By \linkto{exact_noeth}{considering the exact sequence of $A$-modules}
%    \[0 \to \ker \to A \to B \to 0\]
%    we conclude that $B$ is Noetherian as an $A$-module.
%    Hence it is finitely generated by images of elements of $A$.
%    Hence it is finitely generated by elements in $B$.
%    Thus it is Noetherian.
%\end{proof}

\begin{prop}[Hilbert basis theorem]
    \link{hilbert_basis}
    If $A$ is Noetherian then the polynomial ring $A[x_1, \dots, x_n]$ 
    is Noetherian.
    In particular for any field $K$, $K[x_1, \dots, x_n]$ 
    is Noetherian.
\end{prop}
\begin{proof}
    By induction it suffices to show that $A[x]$ is Noetherian.
    Let $\f{a}$ be an ideal of $A[x]$. 
    We show that $\f{a}$ is finitely generated.
    Let $I$ be the ideal of $A$ 
    formed by the leading coefficients of polynomials in $\f{a}$.
    $I$ finitely generated as $A$ is Noetherian.
    Hence $I = \sum{i\in S} A a_i$ for some leading coefficients $a_i$ of 
    polynomials $f_i$.
    We consider the ideal $\sum_{i \in S}A[x]f_i$.

    We show that $\f{a} \leq \sum_{i \in S}A[x]f_i + \sum_{i=0}^{d-1}A[x]x^i$ 
    which is finitely generated, and 
    \linkto{noetherian_definition}{hence $\f{a}$ is finitely generated.}

    Let $p \in \f{a}$ then if $d := \max_{i \in S} \deg f_i \leq \deg p$
    we can write the leading coefficient of $p$ as a sum 
    $\sum_{i \in S} \la_i a_i$ (for $\la_i \in A$).
    So $p - x^{\deg p - d} \sum_{i \in S} \la_i f_i$ had degree strictly
    less than before.
    By induction (stopping when the degree of $p$ is less than $d$) we obtain 
    $p \in \sum_{i \in S}A[x]f_i + \sum_{i=0}^{d-1}A[x]x^i$.
    Hence $A[x_1, \dots, x_n]$ is Noetherian.

    For the `in particular' we note that fields only have two ideals,
    hence trivially satisfy the ascending chain condition.
\end{proof}