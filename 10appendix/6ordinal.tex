\section{Ordinals}
\cite{jech}

\begin{dfn}[Linear ordering, well-ordering, transitive]
    Let $X$ be a set (or class or whatever) 
    \linkto{partial_ordering}{partially ordered}
    by $\leq$. 
    It is a linearly ordered if for each $x, y \in X$, 
    $x \leq y$ or $y \leq x$.
    It is a well-ordered if each non-empty subset $S \subs X$ 
    has a least element,
    i.e. there exists $s \in S$ such that for any $x \in S$, $s \leq x$.

    A set $X$ is transitive if for any member of $X$ is a subset of $X$.
    An example of transitive sets are the naturals:
    \[0 := \nothing, 1 := \set{0}, 2:= \set{0,1}, \dots\]

\end{dfn}

\begin{ex}[Transitive]
    The reason for the use of the word transitive is due 
    to its equivalent definition: 
    $x$ is transitive if for any $y \in x$ and any $z \in y$, $z \in x$.
\end{ex}


\begin{ex}[Well-orderings are linear]
    \link{well_ord_to_lin}
    Take two elements and consider
    the set containing these two elements. 
    This has a minimum.
\end{ex}

\begin{dfn}[Ordinal, succesion, limit ordinal]
    A set in an ordinal if it is transitive and well-ordered by taking 
    $<$ to be $\in$.
    The class of ordinals is denoted $\ord$.

    If $\al \in \ord$ then $\al + 1$ is defined to be $\al \cup \set{\al}$,
    $\al + 1$ is then called a successor ordinal.
    Check that the successor of an ordinal is an ordinal.
    If $\al \in \ord$ is not a successor ordinal then 
    it is called a limit ordinal.

    We endow $\ord$ with an ordering given by 
    $\al \leq \be$ if and only if $\al \subs \be$.
\end{dfn}

\begin{lem}[Transitivity of $<$]
    \link{transitivity_of_less_than}
    Suppose $\leq$ is a partial order.
    The relation $<$ (defined by $\al < \be$ if and only if $\al \leq \be$ and 
    $a \ne b$) is transitive.

    Hence $\al \leq \be < \ga$ implies $\al < \ga$.
\end{lem}
\begin{proof}
    Suppose $a < b$ and $b < c$.
    Then $a \leq b$ and $b \leq c$ and by transitivity of $\leq$ we have $a \leq c$.
    It remains to show that $a \ne c$.
    Suppose $a = c$, 
    then $a \leq b$ and $b \leq a$ and by antisymmetry $a = b$.
    This is false since $a < b$ and so $a \ne b$ by definition.
\end{proof}

\begin{prop}[Basic facts about ordinals]
    \link{basic_facts_ordinals}
    Most importantly, we show that $\ord$ is well-ordered.
    \begin{enumerate}
        \item $0$ is a limit ordinal.
        \item Subsets of ordinals are well-ordered by $\in$.
        \item The intersection of a non-empty subclass of is an ordinal.
        \item Ordinals are closed under membership: if $\al \in \ord$
            and $\be \in \al$ then $\be \in \ord$.
        \item Let $\al \ne \be$ be ordinals. 
            If $\al \subs \be$ then $\al \in \be$.
        \item $\ord$ is well-ordered by $\subs$
        (which is by the fifth part the same thing as $=$ or $\in$).
    \end{enumerate}
\end{prop}
\begin{proof}~
    \begin{enumerate}
        \item $0$ is trivially well-ordered and transitive.
            If $0 = \al \cup \set{\al}$ then $\al \in 0$, 
            which is a contradiction.
        \item Let $\al \in \ord$ and $S \subs \al$.
            By moving elements into $\al$ we see that $S$ 
            inherits the linear ordering given by $\in$.
            Any subset of $S$ is a subset of $\al$,
            hence any subset has a minimal element.
            Thus $S$ is well-ordered
        \item Let $S$ be a non-empty class of ordinals, containing $\al$, say.
            $\bigcap S$ is a subset of $\al$ thus is
            well-ordered by the second part. 
            It is transitive: any element in the intersection satisfies 
            \[\forall \be \in S, x < \be\] 
            hence for every $\be \in S$, as $\be$ is transitive, $x \subs \be$.
            Thus $x \subs \bigcap S$.
        \item Let $\al \in \ord$ and $\be \in \al$.
            As $\al$ is transitive $\be \subs \al$.
            By the second part we have that $\be$ is well-ordered by $<$.

            It remains to show that $\be$ is transitive.
            Let $\ga \in \be$ and suppose for a contradiction that 
            $\ga \nsubseteq \be$.
            Then there exists $\de \in \ga$ such that $\de \notin \be$.
            By \linkto{transitivity_of_less_than}{transitivity of $<$}
            (in this case $\in$), 
            $\de \in \ga \in \be \in \al$ implies $\de \in \al$.
            
            As $\al$ is linearly-ordered we have that $\be \leq \de$ or 
            $\de \leq \be$, equivalently three cases
            $\be \in \de$ or $\be = \de$ or $\de \in \be$.
            The last case is false by assumption. 
            Then first case gives $\de \in \ga \in \be \in \de$,
            hence $\de < \de$ by \linkto{transitivity_of_less_than}{
                transitivity of $<$}.
            In the second case $\de \in \ga \in \be  = \de$,
            hence by transitivity again $\de < \de$.
            In either case $\de \ne \de$, a contradiction.
        \item Let $\ga$ be the minimum element of $\be \setminus \al$,
            using well-ordering of $\be$.
            Then we claim that $\al = \ga$, which implies $\al \in \be$.
            Indeed if $x \in \al$ then $x \in \be$ by assumption.
            By linearity of $\be$, 
            $\ga \leq x$ or $x < \ga$.
            In the first case we have by 
            \linkto{transitivity_of_less_than}{transitivity of $<$} that 
            $\ga < \al$, which is a contradiction as 
            $\ga \in \be \setminus \al$.
            Thus $x \in \ga$ and so $\al \subs \ga$.

            On the other hand suppose $x \in \ga$ and $x \notin \al$.
            Then $x \in \be \setminus \al$ and so by minimality $\ga \leq x$.
            This is a contradiction as $\ga \leq x < \ga$ and by transitivity
            we have $\ga < \ga$.

            We can see $\al$ as the initial segment $\be$ given by $\ga$, 
            i.e. $\set{x \in \be \st x < \ga}$.
        \item Reflexivity, antisymmetry and transitivity are clear.
            It \linkto{well_ord_to_lin}{suffices to show} 
            that it is a well-ordering.
            Let $S$ be a non-empty set of ordinals.
            $\bigcap_{\al \in S} \al$ is an ordinal by the third part.
            We want to show that there exists $\al \in S$ such that 
            $\bigcap S = \al$.
            Suppose not, then by the fifth part we have for any $\al \in S$,
            $\bigcap S \in \al$.
            Hence $\bigcap S \in \bigcap S$, which is a contradiction.
    \end{enumerate}
\end{proof}

\begin{prop}[Transfinite induction]
    \link{transfinite_induction}
    Let $C \subs \ord$ be defined inductively:
    \begin{itemize}
        \item If $\al$ is a limit ordinal and $\forall \be < \al, \be \in C$
        then $\al \in C$. (In particular $0 \in C$.)
        \item If $\al \in C$ then $\al + 1 \in C$.
    \end{itemize}
    Then $C = \ord$. 
    We have the first constructor as the limit case because 
    the smallest ordinal $0$ is a limit ordinal.

    (`Strong') Let $D \subs \ord$ be defined with a single constructor:
    \begin{itemize}
        \item If $\al \in \ord$ and $\forall \be < \al, \be \in C$
        then $\al \in C$.
    \end{itemize}
    Then $D = \ord$.
\end{prop}
\begin{proof}
    Suppose $C \ne \ord$.
    Then as \linkto{basic_facts_ordinals}{$\ord$ is well-ordered},
    there exists $\be \in \ord$ that is the least ordinal such that 
    $\be \notin C$.
    If $\be$ is a successor $\al + 1$ then by minimality of $\be$, $\al \in C$ 
    and applying the first property gives $\be \in C$,
    which is a contradiction. 
    Otherwise, $\be$ is a limit ordinal. 
    Thus by minimality
    $\forall \al < \be, \al \in C$.
    Applying the second property of $C$ gives $\be \in C$.

    Check that the $C \subs D$ which implies $D = \ord$.
\end{proof}

Here is an example of transfinite induction in use:
\begin{lem}[Less than and the successor]
    \link{less_than_and_succ_of_ord}
    If $\al < \be$ are ordinals then $\al + 1 \in \be$ or $\al + 1 = \be$.
\end{lem}
\begin{proof}
    We induct on $\be$.
    Suppose $\be$ is a limit ordinal.
    Then $\al + 1 \notin \be$ implies $\be \leq \al + 1$ by $\ord$ being 
    \linkto{basic_facts_ordinals}{well-ordered}.
    This implies $\al < \al$ which is a contradiction.
    Hence in this case $\al + 1 \in \be$.

    Suppose $\be = \ga + 1$.
    then $\al < \ga$ or $\al = \ga$.
    In the second case $\al + 1 = \ga + 1 = \be$ and we are done.
    In the first case, by the induction hypothesis $\al + 1 \in \ga$ or 
    $\al + 1 = \ga$. 
    In either case $\al + 1 \in \be$.
\end{proof}


To define a function on $\ord$, it then suffices to define a function on $C$, 
using the recursor (in Type theory) of $C$ or of $D$.
For our purposes it means that only need to say what the function does 
to successor and limit ordinals, 
or only what it does to ordinals given the images of all smaller ordinals.

\begin{prop}[Transfinite recursion]
    We denote the class of all sets by $V$.
    Let $G : V \to V$.
    Then there exists a unique map $F : \ord \to V$ such that 
    for each ordinal $\al$, 
    \[F(\al) = G(\res{F}{\al})\]
    $F$ can be thought of as a sequence indexed by $\ord$.\footnote{As 
        we are in set theory it makes 
        sense to consider $\res{F}{\al}$ as a set.}
\end{prop}
\begin{proof}
    We show by \linkto{transfinite_induction}{transfinite induction}
    that for each ordinal $\al$ there exists a unique $F_\al : \al \to V$
    such that for each $\xi < \al$ 
    \[F_\al(\xi) = G(\res{F_\al}{\xi})\]
    \begin{itemize}
        \item If $\al$ is a successor ordinal and there exists a unique 
            $F_\be : \be \to V$ such that $\forall \xi < \be$,
            \[F_\be(\xi) = G(\res{F_\be}{\xi})\]
            Then take $F_\al : \al \to V$ such that it restricts to $F_\be$
            on $\be$ and maps $\be \mapsto G(F_\be)$.
            $F_\al$ is the unique map that satisfies 
            $\forall \xi < \al,F(\xi) = G(\res{F}{\xi})$ since its restriction
            to $\be$ is the unique. 
        \item If $\al$ is a limit ordinal and for any $\be < \al$ there exists a
            unique $F_\be : \be \to V$ satisfying $\forall \xi < \be$
            \[F_\be(\xi) = G(\res{F_\be}{\xi})\]
            Define $F_\al : \al \to V$ as the union of all the $F_\be$.
            $F_\al$ is well-defined since all of the $F_\be$ agree upon restriction
            by uniqueness.
            $F_\al$ satisfies $\forall \xi < \al, F_\al(\xi) = G(\res{F}{\xi})$ by 
            construction.
            $F_\al$ is the unique map satisfying this: if $H$ satisfies the same
            then for any $\be \in \al$, 
            $\res{F_\al}{\be}$ (similarly $\res{H}{\be}$) 
            is the unique map $f$ satisfying 
            \[\forall \xi < \be, f(\xi) = G(\res{f}{\xi})\]
            Thus for any $\be \in \al$, $\res{F_\al}{\be} = \res{H}{\be}$
            \[F_\al(\be) = G(\res{F_\al}{\be}) = G(\res{H}{\be}) = H(\be)\]
            and so $F_\al = H$.
    \end{itemize}
    We define $F$ to map any ordinal $\al$ to $G(F_\al)$.
    By \linkto{transfinite_induction}{strong induction} on 
    $\al$ we show that $F_\al = \res{F}{\al}$:
    suppose for any $\be < \al$, $F_\be = \res{F}{\be}$,
    then for any $\be < \al$
    \[F_\al(\be) = G(\res{F_\al}{\be}) = G(F_\be) = F(\be) = \res{F}{\al}(\be)\]
    and so $F_\al = \res{F}{\al}$ and for any $\al$,
    \[F(\al) = G(F_\al) = G(\res{F}{\al})\]
    Lastly $F$ is unique: if $H$ also satisfies the conditions then suppose
    $F(\be) = H(\be)$ for all $\be < \al$.
    This implies $\res{F}{\al} = \res{H}{\al}$ and
    \[F(\al) = G(\res{F}{\al}) = G(\res{H}{\al}) = H(\al)\]
    thus again by strong induction on $\al$ we have $F = H$.
\end{proof}