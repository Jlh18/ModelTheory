\section{Basics}
These first two sections follow Marker's book on Model Theory 
\cite{marker} with more 
emphasis on where things are happening, i.e. what signature we are working in,
and some more general statements such as working with embeddings rather than 
subsets.

\subsection{Signatures}
\begin{dfn}[First order language]
    We assume we have a tuple
    $\LL = (\CC,\FF,\RR,\VV,\set{\NOT,\OR,\forall,=,\top})$ such that 
    \begin{itemize}
        \item $|\CC|,|\FF|,|\RR|$ each sufficiently large 
            (say they have cardinality $\aleph_5$ or something).
        \item $|\VV| = \aleph_0$. 
            We index $\VV = \set{v_0, v_1, \dots }$ using $\N$.
        \item $\CC$,$\FF$,$\RR$, $\VV$, $\set{\NOT,\OR,\forall,=}$ 
            do not overlap.
    \end{itemize}
    We call $\LL$ the language and only really 
    use it to get symbols to work with.
    Whenever we introduce new symbols to create larger signatures, 
    we are pulling them out of this box.
\end{dfn}

\begin{dfn}[Signature]
    In a language $\LL$, 
    a tuple $\Si = (C,F,n_\star ,R, m_\star)$ is a signature when 
    \begin{itemize}
        \item $C \subs \CC$. 
            We call $C$ the set of constant symbols.
        \item $F \subs \FF$ and 
            $n_{\star} : F \to \N$, 
            which we call the function arity. 
            We call $F$ the set of function symbols.
        \item $R \subs \RR$
            and $m_{\star} : R \to \N$,
            which we call the relation arity.
            We call $T$ the set of relation symbols.
    \end{itemize}
    Given a signature $\Si$, we may refer to its constant, 
    function and relation symbol sets as $\const{\Si},\func{\Si},\rel{\Si}$.
    We will always denote function arity using $n_\star$ 
    and relation arity using $m_\star$.
\end{dfn}

\begin{dfn}[$\Si$-terms]
    Given $\Si$ a signature, its set of terms
    $\term{\Si}$ is inductively defined using three constructors:
    \begin{itemize}
        \item[$\vert$] If $c \in \const{\Si}$
            then $c \in \term{\Si}$.
        \item[$\vert$] If $v_i \in \VV$, 
            $v_i$ is a term.
        \item[$\vert$] Given $f \in \func{\Si}$ and 
            $t  \in (\term{\Si})^{n_f}$
            then there is the string $f(t) \in \term{\Si}$.
    \end{itemize}
\end{dfn}
The terms $v_i$ are called variables and will be 
referred to as elements of $\var{\Si}$ instead of $\VV$.

\begin{prop}[Terms have finitely many variables]
    \link{term_fin_var}
    For any term $t \in \term{\Si}$ there exists a finite subset 
    $S \subs \N$ such that for all $i \in \N$, $i \in S$ if and only if 
    $v_i$ occurs in $t$.
\end{prop}
\begin{proof}
    If there exists a finite subset $T$ of $\N$ such that for all 
    $i \notin T$, 
    $v_i$ does not occur in $t$, 
    then we can take the intersection of all such
    sets and still have a finite set $S$.
    
    We first prove existence of such a 
    $T$ using the inductive definition of $t$.
    \begin{itemize}
        \item If $t = c$ a constant symbol, 
        then $T = \nothing$ satisfies the above.
        \item If $t = v_i$ a variable symbol, 
        then $T = \set{i}$ satisfies the above.
        \item If $t = f(t_0, \dots, t_{n_f})$, 
        then by our induction hypothesis we have 
        a $T_i$ satisfying the condition for each $t_i$. 
        Then $\cup_{i} T_i$ satisfies the condition for t.
    \end{itemize}
    
    Since one such $T$ exists, 
    we can obtain a finite set $S$ by intersecting all 
    sets satisfying the condition on $T$. 
    Let $i \in \N$.
    Suppose $v_i$ occurs in $t$, 
    then $i$ is in any $T$ with the above property. 
    Hence $i \in S$. 
    Hence $S$ also satisfies the above condition.
    Suppose $v_i$ does not occur in t. 
    Then $S \setminus \set{i}$ also satisfies the above condition,
    hence $S \subseteq S \setminus \set{i}$ by minimality of $S$.
    Thus $i\notin S$.
\end{proof}

\begin{dfn}[$\Si$-formula, free variable]
    Given $\Si$ a signature, 
    its set of formulas $\form{\Si}$ is inductively defined:
    \begin{itemize}
        \item[$\vert$] $\top$ is an element of $\form{\Si}$.
        \item[$\vert$] Given $t, s \in \term{\Si}$, 
        there is the string $(t = s) \in \form{\Si}$
        \item[$\vert$] Given $r \in \rel{\Si}, t \in (\term{\Si})^{m_r}$, 
        there is the string $r(t) \in \form{\Si}$ \vspace{1em}
        \item[$\vert$] Given $\phi \in \form{\Si}$, 
        the string $(\NOT \phi) \in \form{\Si}$ 
        \item[$\vert$] Given $\phi, \psi \in \form{\Si}$, the string 
        $(\phi \lor \psi) \in \form{\Si}$
        \item[$\vert$] Given $\phi \in \form{\Si}$ 
        and $v_i \in \var{\Si}$, we take the replace all occurrences of
        $v_i$ with an unused symbol such as $z$ in the string $(\forall v_i, \phi)$
        and call this a new element of $\form{\Si}$.
    \end{itemize}
    Shorthand for some strings in $\form{\Si}$ include 
    \begin{itemize}
        \item $\bot := \NOT \top$
        \item $\phi \land \psi := \NOT(\NOT\phi \lor \NOT\psi)$
        \item $\phi \to \psi := (\NOT\phi) \lor \psi$
        \item $\exists v, \phi := \NOT(\forall v, \NOT\phi)$
    \end{itemize}
    
    The symbol $z$ is meant to be a `bounded variable', 
    and will not be considered when we want to evaluate
    variables in formulas.
    Variables that we do want to evaluate we call `free variables'.
\end{dfn}
\begin{rmk}
    There are two different uses of the symbol `$=$' from now on, 
    and context will allow us to tell them apart.
    Similarly, logical symbols might be used in our `higher' 
    language and will not be confused with symbols from formulas.
\end{rmk}

\begin{prop}[Formulas have finitely many \emph{free} variables]
    \link{form_fin_var}
    Given $\phi \in \form{\Si}$,
    there exists a finite $S \subs \N$ such that
    for all $i \in \N$, $i \in S$ if and only if $v_i$ occurs freely in $\phi$.
    
\end{prop}
\begin{proof}
    Similarly to the \linkto{term_fin_var}{previous proposition}, 
    we only need to show that there exists a $T$
    such that for all $i \notin T$,
    $v_i$ does not occur in $\phi$.
    We can do this by using the inductive definition of $\phi$:
    \begin{itemize}
        \item If $\phi$ is $\top$ then it has no variables.
        \item If $\phi$ is $t = s$, 
        then we have $S_t, S_s$ indexing the variables of $t$ and $s$ 
        \linkto{term_fin_var}{by the previous proposition},
        and so we can pick $T = S_t \cup S_s$.
        \item If $\phi$ is $r(t_0, \dots, t_{m_r})$, 
        then for each $t_i$ we have $S_i$ indexing the variables of $t_i$.
        Hence we can pick $T = \cup_i S_i$.
        \item If $\phi$ is $\NOT \psi$, 
        then by the induction hypothesis we have $T$ 
        satisfying the above conditions for $\psi$. 
        Pick this $T$ for $\phi$.
        \item If $\phi$ is $\psi \lor \chi$
        then by the induction hypothesis we have 
        $T_{\psi}, T_{\chi}$ satisfying the above conditions for $\psi$ and $\chi$.
        We take $T$ to be the union of indexing sets for $\psi$ and $\chi$.
        \item If $\phi$ is $\forall v_i, \psi$ with $v_i$ substituted for $z$,
        then by the induction hypothesis we have 
        $T_{\psi}$ satisfying the above conditions for $\psi$.
        Take $T = T_{\psi} \setminus \{v_i\}$.
    \end{itemize}
\end{proof}

\begin{nttn}[Substituting Terms for Variables]
    If a formula has a free variable $v_i$ 
    then to remind ourselves of the variable we can write 
    $\phi = \phi(v_i)$ to mean the same thing.

    If $t \in (\term{\Si})^S$, 
    then we write $\phi(t)$ to mean $\phi$ with 
    $t_i$ substituted for each $v_i$.
    We can show by induction on terms and formulas that this is still a formula.
\end{nttn}

\begin{eg}[The empty signature and theory]
    $\Si_\nothing = (\nothing, \nothing, n_\star, \nothing, m_\star)$ 
    is the empty signature.
    (We pick the empty functions for $n_\star, m_\star$.)
    The empty $\Si_\nothing$-theory is $\nothing$.
    Notice that any set is a model of $\Si_\nothing$.
\end{eg}
Note that in set theory the carrier set of models are indeed sets,
hence any set is a model of the empty theory.

\begin{dfn}[$\Si$-structure, interpretation]
    Given a signature $\Si$, a set $M$ and a tuple 
    $\intp{\Si}{\MM} = 
    \brkt{\intp{\const{\Si}}{\MM}, 
    \intp{\func{\Si}}{\MM}, \intp{\rel{\Si}}{\MM}}$, 
    we say that $\MM := (M, \intp{\Si}{\MM})$ is a $\Si$-structure when 
    \begin{itemize}
        \item $\intp{\const{\Si}}{\MM} : \const{\Si} \to M$
        \item $\intp{\func{\Si}}{\MM} : \func{\Si} \to (M^{n_\star} \to M)$
        \item $\intp{\rel{\Si}}{\MM} : \rel{\Si} \to \PP(M^{m_{\star}})$
    \end{itemize}
    The latter two are dependant types since the powers
    $n_\star, m_\star$ 
    depend on the function and relation symbols given.
    The class of $\Si$-structures is denoted $\struc{\Si}$.
    Given only the $\Si$-structure $\MM$, 
    we call its underlying carrier set $M$ as $\carrier{\MM}$.
\end{dfn}
We can think of $\intp{\Si}{\MM}$
as a single entity interpreting constant symbols as elements of $\carrier{\MM}$,
function symbols as $n_{\star}$-ary functions on $\carrier{\MM}$,
and relation symbols as $m_{\star}$-ary relations on $\carrier{\MM}$.
When the context is clear,
given $c \in \const{\Si}, f \in \func{\Si}, r \in \rel{\Si}$, 
we may write 
$\subintp{\const{\Si}}{\MM}{c} = \subintp{\Si}{\MM}{c} = \modintp{\MM}{c}, 
\subintp{\func{\Si}}{\MM}{f} = \subintp{\Si}{\MM}{f}  = \modintp{\MM}{f}, 
\subintp{\rel{\Si}}{\MM}{r} = \subintp{\Si}{\MM}{r}  = \modintp{\MM}{r}$.
We call these interpretations.

\begin{dfn}[Interpretation of terms]
    Given a signature $\Si$, 
    a $\Si$-structure $\MM$ and a $\Si$-term $t$, 
    There exists a unique induced map 
    $\subintp{T}{\MM}{t} : \carrier{\MM}^S \to \carrier{\MM}$, 
    such that this commutes with the interpretation of constants and
    functions\footnote{Precisely:
        For all constant symbols $c$, 
        $\subintp{T}{\MM}{\iota(c)} = \subintp{\const{\Si}}{\MM}{c}$,
        where $\iota$ is the inclusion $\const{\Si} \to \term{\Si}$.
        For all $f \in \func{\Si}, s \in \term{\Si}, 
        \subintp{T}{\MM}{(f(s))}(\star) = 
        \subintp{\func{\Si}}{\MM}{f}(\subintp{T}{\MM}{s}(\star))$.
        }.
    Here $S$ is the unique set indexing the variables of $t$.
    We then refer to this map as \emph{the} interpretation of the term $t$.
    This in turn defines a dependant
    $\Pi$-type 
    \[\intp{T}{\MM}: \term{\Si} \to (\carrier{\MM}^{S_*} \to \carrier{\MM})\]
\end{dfn}
\begin{proof}
    Given a signature $\Si$, 
    a $\Si$-structure $\MM = (M, \intp{\Si}{\MM})$ and a $\Si$-term $t$, 
    \linkto{term_fin_var}{there exists a finite 
    $S \subseteq \N$ indexing the variables of t}.
    To define a map $\subintp{T}{\MM}{t} : M^S \to M$ for each
    $t$ we use the inductive definition of $t \in \term{\Si}$.
    If $M$ is empty we define $\subintp{T}{\MM}{t}$ as the empty function.
    Otherwise let $a \in M^S$:
    \begin{itemize} 
        \item If $t = c \in \const{\Si}$ then define 
        $\subintp{T}{\MM}{t} : a \mapsto \subintp{C}{\MM}{c}$, 
        the constant map.
        This type checks since $S = \nothing$ 
        therefore $\subintp{T}{\MM}{t} : M^0 \to M$.
        \item If $t = v_i \in \var{\Si}$
        then define $\subintp{T}{\MM}{t} : a \mapsto a$, the identity.
        This type checks since $|S| = 1$.
        \item If $t = f(s)$ for some $f \in \func{\Si}$ and 
        $s \in (\var{\Si})^{n_f}$ then define 
        $\subintp{T}{\MM}{t} : a \mapsto \subintp{F}{\MM}{f}(\subintp{T}{\MM}{s}(a))$.
        This type checks since $s$ has the same number of variables as $t$.
    \end{itemize}
    By definition, 
    this map commutes with the interpretation of constants and functions.
    Furthermore, 
    it is unique since it is constructed to satisfy the commuting properties.
\end{proof}
Where there is no ambiguity, 
we write $\subintp{T}{\MM}{t} = \modintp{\MM}{t}$.
Furthermore, if we have a tuple 
$t \in (\term{\Si})^k$,
then we write $\subintp{T}{\MM}{t}:= 
(\subintp{T}{\MM}{t_0},\cdots,\subintp{T}{\MM}{t_k})$
    
\begin{dfn}[Satisfaction]
    Given $\Si$ a signature, $\MM \in \struc{\Si}$, 
    and a $\phi \in \form{\Si}$,
    \linkto{form_fin_var}{there exists a subset $S \subs \N$ 
    indexing the free variables of $\phi$}.
    Given $a \in (\carrier{\MM})^S$, 
    we want to define $\MM \model{\Si} \phi(a)$.
    If $\carrier{\MM}$ is empty then we write 
    $\MM \model{\Si} \phi(a)$.
    Otherwise, using the inductive definition of $\phi$:
    \begin{itemize}
        \item If $\phi$ is $\top$ then $\MM \model{\Si} \phi$.
            (We can omit the $a$ when there are no free variables.)
        \item If $\phi$ is $t = s$ then
            $\MM \model{\Si} \phi(a)$ when 
            $\modintp{\MM}{t}(a) = \modintp{\MM}{s}(a)$.
            \item If $\phi$ is $r(t)$, 
            where $r \in \rel{\Si}$ and 
            $t \in (\term{\Si})^{m_r}$,
            then $\MM \model{\Si} \phi(a)$ when 
            $\modintp{\MM}{t}(a) \in \modintp{\MM}{r}$.
            \vspace{1em}
        \item If $\phi$ is the string 
            $\NOT\psi$ for some $\psi \in \form{\Si}$, 
            then $\MM \model{\Si} \phi(a)$ when $\MM \nodel{\Si} \psi(a)$
        \item If $\phi$ is the string $(\psi \lor \chi)$, 
            then $\MM \model{\Si} \phi(a)$ when 
            $\MM \model{\Si} \psi(a)$ or $\MM \model{\Si} \chi(a)$.
        \item If $\phi$ is the string 
            $(\forall v, \psi(a)) \in \form{\Si}$,
            then $\MM \model{\Si} \phi(a)$ 
            if for any $b \in \carrier{\MM}$,   
            $\MM \model{\Si} \psi(a)(b)$.
    \end{itemize}
    We say $\MM$ satisfies $\phi(a)$.
\end{dfn}
\begin{rmk}
    Note that $S$ is empty when there are no free variables.
    Note also by convention we have a rather nasty possible case of 
    \[\nothing \modelsi \exists v, v = v\]
    which is intuitively wrong.%
\end{rmk}
\begin{rmk}\link{top_prop}
    Any $\Si$-structure $\MM$ satisfies $\top$ by casing on if 
    the structure is empty or not.
\end{rmk}

\subsection{Theories and Models}
\begin{dfn}[$\Si$-sentence, $\Si$-theory]
    We say $\phi \in \form{\Si}$ is a $\Si$-sentence when 
    \linkto{form_fin_var}{the set indexing the free variables} 
    of $\phi$ is the empty set.
    
    $T$ is an $\Si$-theory when it is a subset of $\form{\Si}$
    such that all elements of $T$ are $\Si$-sentences.
    We denote the set of $\Si$-theories as $\theory{\Si}$
\end{dfn}

\begin{dfn}[$\Si$-Model, consistent, finitely consistent] 
    \link{consistent}
    Given an $\Si$-structure $\MM$ and $\Si$-theory $T$, 
    we write $\MM \model{\Si} T$ and say
    \emph{$\MM$ is a $\Si$-model of $T$} when 
    for all $\phi \in T$ we have $\MM \model{\Si} \phi$.
    
    A $\Si$-theory $T$ is consistent if there exists 
    $\MM \in \struc{\Si}$ such that $\MM \model{\Si} T$ and 
    $\carrier{\MM} \ne \nothing$. 
    It is finitely consistent if all 
    finite subsets of $T$ are consistent.

    Let $\phi$ be $\Si$-formula with variables indexed by $S$. 
    Then given $c \in (\const{\Si})^S$ we write 
    $\phi(c)$ to mean the formula with its 
    variables substituted for the constant symbols.
    In this case $\MM \model{\Si} \phi(c)$ would be the same thing as 
    $\MM \model{\Si} \phi(\mmintp{c})$.
    Similarly we write $t(c)$ for terms.
\end{dfn}
Note that if $\phi$ is a $\Si$-formula with one variable $v$,
and $\MM \modelsi \phi(a)$ for some $a \in \carrier{\MM}$, 
we might not be able to say that there is a 
$c \in \const{\Si}$ such that $T \modelsi \phi(c)$.

\begin{dfn}[Consequence]
    Given a $\Si$-theory $T$ 
    and a $\Si$-sentence $\phi$,
    we say $\phi$ is a consequence of $T$
    and say $T \model{\Si} \phi$ 
    when for all models $\MM$ of $T$, 
    we have $\MM \model{\Si} \phi$.
\end{dfn}
\begin{rmk}
    We have to be a bit careful when we go from something like
    $\MM \model{\Si} \phi(a)$ to deducing something about $T$.
    This is because there might not exist a $\Si$-constant $c$ 
    such that $\modintp{\MM}{c} = a$,
    it only makes sense to write $T \model{\Si} \phi$ if $\phi$
    is a \emph{sentence}.
\end{rmk}


\begin{ex}[Empty model]
    \link{ex_empty_model}
    Show that for any $\Si$-formula $\phi$ the $\Si$-structure
    $\NN = (\nothing,\nothing)$ (if it is indeed a $\Si$-structure) satisfies
    $\NN \model{\Si} \phi$ and $\NN \model{\Si} \NOT \phi$.
    Does this lead to a contradiction?
\end{ex}

\begin{prop}[Inconsistent theory]
    \link{inconsistent}
    Given $T$ a $\Si$-theory, 
    show that $T$ not consistent if and only if there exists a 
    $\Si$-sentence $\phi$ such that
    $T \model{\Si} \phi$ and $T \model{\Si} \NOT \phi$.
\end{prop}
\begin{proof}
    \begin{forward}
        Take $\phi$ to be the $\Si$-sentence $\top$.
        Since the only model of $T$ is the empty $\Si$-structure 
        and $\nothing \model{\Si} \top$ and $\nothing \model{\Si} \bot$,
        $T \model{\Si} \top$ and $T \model{\Si} \bot$.
    \end{forward}
    \begin{backward}
        Suppose $T$ is consistent. 
        Let $\MM$ be the non-empty $\Si$-model of $T$.
        Then $\MM \model{\Si} \phi$ and $\MM \model{\Si} \NOT \phi$.
        Then since $\MM$ is non-empty this implies 
        $\MM \model{\Si} \phi$ and $\MM \nodel{\Si} \phi$,
        a contradiction.
    \end{backward}
\end{proof}
Thus the definition of consistent is thus intuitively 
`$T$ does not lead to a contradiction'.
