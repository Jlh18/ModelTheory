\subsection{Back and Forth}
\begin{nttn}
    We write the signature of a single relation symbol $r$ as
    \[\Si_r := \set{\nothing,\nothing,n_\star,\set{r},m_\star}\]
\end{nttn}

\begin{dfn}[Local isomorphisms]
    Let $\MM$ and $\NN$ be $\Si_r$-structures.
    Let $p \in \N$.
    We define the sets $\local_p(\modintp{\MM}{r},\modintp{\NN}{r})$
    inductively.
    \begin{itemize}
        \item $\local_0(\modintp{\MM}{r},\modintp{\NN}{r})$ is the set of 
            all $\Si_r$-isomorphisms $\io : A \to B$, where 
            $A$ and $B$ are substructures of $\MM$ and $\NN$.
        \item $\local_{k+1}(\modintp{\MM}{r},\modintp{\NN}{r})$ is the 
            set of $\io \in \local_{k}(\modintp{\MM}{r},\modintp{\NN}{r})$ 
            such that ($1.$) for any $a \in \MM$ there exists 
            $\la \in \local_{k}(\modintp{\MM}{r},\modintp{\NN}{r})$
            such that $\dom(\io) \cup \set{a} \subs \dom(\la)$
            and $\res{\la}{\dom(\io)} = \io$;
            ($2.$) for any $b \in \NN$ there exists 
            $\la \in \local_{k}(\modintp{\MM}{r},\modintp{\NN}{r})$
            such that $\codom(\io) \cup \set{b} \subs \codom(\la)$.
    \end{itemize}
    The first requirement in the inductive part of the definition is called 
    `forth' condition and the second is called the `back' condition.
    The first set $\local_0(\modintp{\MM}{r},\modintp{\NN}{r})$ is called 
    the set of local isomorphisms on $r$ and elements of 
    $\local_p(\modintp{\MM}{r},\modintp{\NN}{r})$ are often referred to 
    as $p$-isomorphisms on $r$.

    Note that as $\Si_r$ has no constant symbols or function symbols,
    subsets are naturally substructures by taking the interpretation of 
    the relation symbol to be the intersection.
\end{dfn}

\begin{prop}[Basic facts about local isomorphisms]
    \link{basic_facts_local_isomorphisms}
    Let $\MM$ and $\NN$ be $\Si_r$-structures.
    The sets of $p$-isomorphisms on $r$ form a chain
    \[\cdots \subs \local_p(\modintp{\MM}{r},\modintp{\NN}{r}) \subs \cdots 
    \subs \local_0(\modintp{\MM}{r},\modintp{\NN}{r})\]
    The following are useful facts:
    \begin{itemize}
        \item If $p \leq q \in \N$ then 
            $\local_q(\modintp{\MM}{r},\modintp{\NN}{r}) \subs
            \local_p(\modintp{\MM}{r},\modintp{\NN}{r})$.
        \item $p$-isomorphisms are preserved under substructure:
            if $\io \in \local_p(\modintp{\MM}{r},\modintp{\NN}{r})$
            then for any substructure $A \subs \dom(\io)$, 
            $\res{\io}{A} : A \to \io(A)$ is an element of 
            $\local_p(\modintp{\MM}{r},\modintp{\NN}{r})$.
        \item The inverse of a $p$-isomorphims is a $p$-isomporphism:
            if $\io \in \local_p(\modintp{\MM}{r},\modintp{\NN}{r})$
            then $\io^{-1} \in \local_p(\modintp{\NN}{r},\modintp{\MM}{r})$.
        \item The composition of $p$-isomorphisms is a $p$-isomorphism:
            if $\io \in \local_p(\modintp{\MM_0}{r},\modintp{\MM_1}{r})$,
            and $\la \in \local_p(\modintp{\MM_1}{r},\modintp{\MM_2}{r})$
            and $\dom(\la) = \codom(\la)$ then 
            \[  
                \la \circ \io \in 
                \local_p(\modintp{\MM_0}{r},\modintp{\MM_2}{r})
            \]
        \item If $\local_p(\modintp{\MM}{r},\modintp{\NN}{r}) = 
            \local_{p+1}(\modintp{\MM}{r},\modintp{\NN}{r})$
            then the chain is constant from $p$ onwards:
            if $p \leq q$ then $\local_p(\modintp{\MM}{r},\modintp{\NN}{r})
            = \local_q(\modintp{\MM}{r},\modintp{\NN}{r})$.
        \item If $\MM$ is finite and $\io : \MM \to \NN$ 
            is a $\Si_r$-isomorphism then $\io$ 
            is a $p$-isomorphism for any $p$.
    \end{itemize}
\end{prop}
\begin{proof}
    We prove the last claim:
    Suppose $\MM$ is finite and $\io : \MM \to \NN$ 
    is a $\Si_r$-isomorphism.
    Induct on $p \in \N$.
    If $p = 0$ then $\io \in \local_p$ as 
    $\io : \MM \to \NN$ is itself a $0$-isomorphism
    Suppose $\io \in \local_{k}$ then 
    for any element $a \in \MM$, 
    $\io$ extends itself with domain containing $a$,
    since $\dom(\io) = \MM$.
    Similarly the backwards condition is satisfied.
    Hence $\io \in \local_{k+1}$.
\end{proof}

\begin{dfn}[$\om$-isomorphism]
    Let $\MM$ and $\NN$ be $\Si_r$-structures.
    The set of $\om$-isomorphisms 
    \[\local_\om(\modintp{\MM}{r},\modintp{\NN}{r})\]
    is defined as the projective limit of the chain of $p$-isomorphisms.
    This is equivalent to it being 
    \[\bigcap_{p \in \N} \local_p(\modintp{\MM}{r},\modintp{\NN}{r})\]
\end{dfn}
\begin{rmk}
    If for every $p \in \N$, the set of $p$ isomorphisms is non-empty then 
    each contains the empty set 
    (\linkto{basic_facts_local_isomorphisms}{by restrction}).
    Hence
    $\local_\om(\modintp{\MM}{r},\modintp{\NN}{r})$ contains the empty set.
\end{rmk}

\begin{prop}[Intuitive property of $p$-isomorphisms]
    \link{intuitive_prop_of_p_iso}
    Let $\MM$ and $\NN$ be infinite $\Si_r$-structures.
    Suppose $a \subs \MM, b \subs \NN$ are finite subsets
    and $\io : a \to b$ is in $\local_{q}(\mmintp{r},\nnintp{r})$ then 
    for every finite subset $A \subs \MM$ such that $a \subs A$
    and $\abs{A \setminus a} = q$
    there exists an extension $\la : A \to B$ of $\io$ 
    that is a local isomorphism.
\end{prop}
\begin{proof}
    We show by induction on $p \in \N$ that 
    for any $p$, for any $\io \in \local_p$, for any subset $A \subs \MM$.
    such that $\abs{A} = p$,
    there exists a local isomorphism extending $\io$ 
    with domain $\dom(\io) \cup A$.
    Suppose $p = 0$ then we pick the extension 
    to be $\io$ itself and we are done.

    Suppose $p = k + 1$.
    Then let $\io \in \local_p$ and $A \subs \MM$ be such that 
    $\abs{A} = p$.
    Then as $p \ne 0$, 
    $A$ is non-empty containing $A_0$.
    By definition of $\local_p$ there exists $\la \in \local_k$ 
    extending $\io$ such that $A_0 \in \dom(\la)$.
    Then by the induction hypothesis 
    (taking $\la \in \local_k$, 
    $A \setminus \set{A_0} \subs \MM$)
    we have $\la' : A \cup \dom(\la) \to B$ a local isomorphism.
    We can restrict the domain of $\la'$ 
    to be $A \cup \dom(\io)$ as 
    \linkto{basic_facts_local_isomorphisms}{local isomorphisms 
    are preserved under substructure},
    giving us the extension that we want.
    Thus we are finished with the induction.

    Let $A \subs \MM$ be finite and containing $a$.
    Take $p$ to be $q$.
    By assumption $\io \in \local_{p}(\mmintp{r},\nnintp{r})$.
    By what we showed above there exists an extension 
    $\la : A \to B$ of $\io$ 
    that is a local isomorphism.
\end{proof}

\begin{prop}[Equivalent characterisation of $\om$-isomorphisms]
    \link{extending_by_arbitrary_subets_gives_om_iso}
    Let $\MM$ and $\NN$ be infinite $\Si_r$-structures.
    Suppose $a \subs \MM, b \subs \NN$ are finite subsets
    and $\io : a \to b$ is a bijection.
    Then the following are equivalent:
    \begin{itemize}
        \item $\io \in \local_\om(\mmintp{r},\nnintp{r})$
        \item For every finite subset $A \subs \MM$ such that $a \subs A$,
            there exists an extension $\la : A \to B$ of $\io$ 
            that is a local isomorphism.
    \end{itemize}
    
    Note that we only need the `forth' version.
    A similar `back' version is equivalent as well.
\end{prop}
\begin{proof}
    \begin{forward}
        Follows from picking $p = \abs{A \setminus a}$ and applying 
        \linkto{intuitive_prop_of_p_iso}{the previous proposition}.
    \end{forward}

    \begin{backward}
    We induct on $p \in \N$ to show that $\io \in \local_p$ for each $p$.
    For $p = 0$,
    take $A$ to be $a$,
    then by assumption there exists $\la : A \to B$ a local isomorphism
    that agrees with $\io$ upon restriction.
    Hence $\io = \la \in \local_0$.

    Suppose $p = k + 1$. 
    By the induction hypothesis $\io \in \local_k$.
    By the axiom of choice we can 
    find $A \subs \MM$ such that 
    $\abs{A \setminus a} = p$.
    By assumption there exists $\la : A \to B$ a local isomorphism
    extending $\io$.
    We have that $p \ne 0$ 
    hence the well-ordering principle allows us to write $A$ as a disjoint union
    \[A = a \sqcup \set{A_1,\dots,A_p}\]
    with $1 \leq r$.
    Note that \linkto{basic_facts_local_isomorphisms}{by restriction}
    $\res{\la}{A\setminus \set{A_1}} \in \local_1$.
    By induction we get that for any $1 \leq s \leq r$,
    \[\res{\la}{A\setminus \set{A_1,\dots, A_s}} \in \local_s\]
    In particular 
    \[\io = \res{\la}{A \setminus \set{A_1,\dots, A_p}} \in \local_p\]
    \end{backward}
\end{proof}

\begin{dfn}[Fraïssé rank]
    Let $\MM$ and $\NN$ be $\Si_r$-structures.
    Let $\io \in \local_0(\modintp{\MM}{r},\modintp{\NN}{r})$ be a local
    isomorphism. 
    We define the Fraïssé rank of $\io$ to be an element of $\N \cup {\om}$.
    If there exists $p \in \N$ such that
    \[
        \io \in \local_p(\modintp{\MM}{r},\modintp{\NN}{r}) 
        \setminus \local_{p+1}(\modintp{\MM}{r},\modintp{\NN}{r})
    \]
    Then it is said to have rank $p$,
    Otherwise it has rank $\om$.
\end{dfn}

\begin{dfn}[$p$-equivalence]
    Let $\MM$ and $\NN$ be $\Si_r$-structures,
    let $a \in \MM^{m_r}$ and $b \in \NN^{m_r}$.
    Let $p \in \N$.
    We say that $a$ and $b$ are $p$-equivalent when 
    \[\io : \set{a_1,\dots,a_{m_r}} \to \set{b_1,\dots,b_{m_r}}
    := a_i \mapsto b_i\]
    is an element of $\local_p(\modintp{\MM}{r},\modintp{\NN}{r})$.
    We write $(a,\modintp{\MM}{r}) \sim_p (b, \modintp{\NN}{r})$.
    (From now on we just write $a \in \dom(\io)$ to mean 
    $a_1,\dots,a_{m_r} \in \dom(\io)$.)

    If for all $p \in \N$ we have 
    $(a,\modintp{\MM}{r}) \sim_p (b, \modintp{\NN}{r})$
    then $a$ and $b$ are said to be $\om$-equivalent.
    For this we write $(a,\modintp{\MM}{r}) \sim_\om (b, \modintp{\NN}{r})$.
    and we say that $a$ and $b$ have the same type 
    (this is justified in the following proposition).
    One should check that these are equivalence relations.    
\end{dfn}

\begin{ex}
    Let $\MM$ and $\NN$ be infinite $\Si_r$-structures,
    with $A \subs \MM$ and $B \subs \NN$.
    Show by induction on formulas that if $\la : A \to B$ is a local isomorphism
    then there exists a bijective induced map 
    $\bar{\la} : \form{\Si_r(A)} \to \form{\Si_r(B)}$
    such that $\bar{\la} \psi(A) = \psi(\la(A))$.
\end{ex}

\begin{prop}[Elements having the same type]
    Let $\MM$ and $\NN$ be infinite $\Si_r$-structures,
    let $a \in \MM^{m_r}$ and $b \in \NN^{m_r}$.
    Let $\io : \set{a_1,\dots,a_{m_r}} \to \set{b_1,\dots,b_{m_r}}$
    be the map $a_i \mapsto b_i$.
    Then the following are equivalent:
    \begin{enumerate}
        \item $(a,\modintp{\MM}{r}) \sim_\om (b, \modintp{\NN}{r})$.
        \item $\io \in \local_1$ and 
            \[\subintp{\nothing,m_r}{\MM}{\tp}(a) = 
            \subintp{\nothing,m_r}{\NN}{\tp}(b)\]
    \end{enumerate}
\end{prop}
\begin{proof}
    \begin{forward}
        Clearly $\io$ is a $1$-isomorphism.
        Let $\phi \in \subintp{\nothing,m_r}{\MM}{\tp}(a) \subs F(\Si_r,n)$.
        By definition of $\subintp{\nothing,m_r}{\MM}{\tp}(a)$ we have 
        \begin{align*}
            &\phi \in \subintp{\nothing,m_r}{\MM}{\tp}(a) \iff 
            \MM \model{\Si_r} \phi(a) \\
            \iff A \model{\Si_r} \psi(A)
        \end{align*}
        As $A \subs \MM$ is elementary (similarly $B \subs \NN$).
        Furthermore, $\io$ being an isomorphism tells us this is if and only if
        \[
            \NN \model{\Si_r} (\phi)(b)
            \iff \phi \in \subintp{\nothing,m_r}{\NN}{\tp}(b)
        \]
        Thus 
        \[
            \subintp{\nothing,m_r}{\MM}{\tp}(a) = 
            \subintp{\nothing,m_r}{\NN}{\tp}(b)
        \]
    \end{forward}

    \begin{backward}
        We first show by induction on $p \in \N$
        that for any $p$ and any local isomorphism $\la \in \local_1$ 
        extending $\io$, if the induced map 
        $\bar{\la} : \form{\Si_r(A)} \to \form{\Si_r(B)}$
        restricts to a bijection
        \[\subintp{A,m_r}{\MM}{\tp}(a) \to 
        \subintp{B,m_r}{\NN}{\tp}(b)\]
        then $\la \in \local_p$.
        The base case is clear.

        For the induction, 
        let $\la : A \to B$ be a $1$-isomporphism extending
        $\io$ such that $\bar{\la} : \form{\Si_r(A)} \to \form{\Si_r(B)}$
        restricts to a bijection
        \[\subintp{A,m_r}{\MM}{\tp}(a) \to 
        \subintp{B,m_r}{\NN}{\tp}(b)\]
        By the induction hypothesis $\la \in \local_p$ and we show that 
        it is in the next one.
        Let $\al \in \MM$.
        By definition of $\local_p$ there exists $\mu$ a $p - 1$-isomorphism
        such that $\al \in \dom(\mu)$ and $\res{\mu}{a} = \io$.
        We show that the induced map 
        $\bar{\mu} : \form{\Si_r(A)} \to \form{\Si_r(B)}$
        restricts to a bijection
        \[\subintp{\dom(\mu),m_r}{\MM}{\tp}(a) \to 
        \subintp{\im(\mu),m_r}{\NN}{\tp}(b)\]
        Indeed, let $\phi \in F(\Si_r(\dom(\mu)),n)$
        (writing it is $\psi(\dom(\mu))$ for $\psi \in \form{\Si_r}$),
        \begin{align*}
            &\phi \in \subintp{\dom(\mu),m_r}{\MM}{\tp}(a) \iff 
            \MM \model{\Si_r(\dom(\mu))} \phi(a) \iff
            \MM \model{\Si_r} \psi(\dom(\mu))\\
            \iff & \dom{\mu} \model{\Si_r} \psi(\dom(\mu)) \quad 
            \text{as subsets are elementary embeddings}\\
            \iff & \im{\mu} \model{\Si_r} \psi(\im(\mu)) \quad
            \text{as $\mu$ is a 
            \linkto{iso_imp_elem_equiv}{$\Si_r$-isomorphism}}\\ 
            \iff & \NN \model{\Si_r} \psi(\im(\mu)) \iff 
            \NN \model{\Si_r(\im(\mu))} \bar{\mu}(\phi)(\mu(a))\\
            \iff & \bar{\mu}(\phi) \in \subintp{\im(\mu),m_r}{\NN}{\tp}(b)
        \end{align*}
        Thus by the induction hypothesis $\mu \in \local_p$
        and the `forth' condition is satisfied.
        The `back' condition is similar, 
        hence we have that $\la \in \local_{p+1}$.

        The induction is done and we can conclude that any 
        $1$-isomorphism $\la$ that extends $\io$ and induces a bijection 
        on the types of elements is an $\om$-isomorphism.
        By assumption $\io$ satisfies these conditions and therefore 
        $\io \in \local_\om$.
        (Note that the induced map
        $\bar{\io} : \form{\Si_r} \to \form{\Si_r}$
        is actually the identity
        \[\subintp{\nothing,m_r}{\MM}{\tp}(a) \to 
        \subintp{\nothing,m_r}{\NN}{\tp}(b)\]
        which is why it is phrased as equality in the statement of the theorem.)
    \end{backward}
\end{proof}


%%%%%%%%%%%%%%%%%%%%%%%%%%%%

%second attempt
\begin{dfn}[Partial isomorphisms]
    Let $\MM$ and $\NN$ be $\Si$-structures.
    A partial isomorphism from $\MM$ to $\NN$ is a bijective function $p$
    with finite domain in $\MM$ and codomain in $\NN$ such that it commutes with 
    interpretation:
    \begin{itemize}
        \item For all constant symbols $c$ such that $\mmintp{c} \in \dom(p)$,
            $p(\mmintp{c}) = \nnintp{c}$.
        \item For all function symbols $f$ and $a \in \dom(p)^{n_f}$, 
            \[p(\mmintp{f}(a)) =
             \nnintp{f}(p(a))\]
        \item For all relation symbols $r$ and $a \in \dom(p)^{m_f}$,
            \[a \in \mmintp{r} \iff p(a) \in \nnintp{r}\]
    \end{itemize}
\end{dfn}

\begin{prop}[Basic facts about partial isomorphisms]
    \link{basic_facts_partial_isomorphisms}
    Let $\MM$ and $\NN$ be $\Si$-structures.
    \begin{itemize}
        \item The inverse of a partial isomorphism is a partial isomorphism.
        \item The restriction of a partial isomorphism is a partial isomorphism.
        \item The composition of partial isomorphisms is a partial isomorphism.
    \end{itemize}
\end{prop}

\begin{dfn}[Partially isomorphic structures]
    Let $\MM$ and $\NN$ be $\Si$-structures.
    $\MM$ and $\NN$ are partially isomorphic via $P$
    when $P$ is a non-empty set
    of partial isomorphisms from $\MM$ to $\NN$ such that 
    \begin{itemize}
        \item (Forth) For each $p \in P$ and $a \in \MM$
            there exists $q \in P$ such that 
            $q$ extends $p$ and $a \in \dom{p}$.
        \item (Back) For each $p \in I$ and $b \in \NN$
            there exists $q \in I$ such that 
            $q$ extends $p$ and $b \in \codom{q}$.
    \end{itemize}
    We write $\MM \iso_P \NN$.
\end{dfn}

\begin{prop}[Countable partially isomorphic structures are isomorphic]
    Let $\MM$ and $\NN$ be $\Si$-structures with
    cardinality less than or equal to $\om$.
    If $\MM$ and $\NN$ are partially isomorphic then 
    $\MM$ and $\NN$ are isomorphic. 
\end{prop}
\begin{proof}
    Write $\MM = \set{a_i}_{i \in \N}$ and $\NN = \set{b_i}_{i \in \N}$
    (if $\MM$ and $\NN$ are finite then after some $i$ all the elements are 
    equal).
    $\MM \iso_P \NN$ by some non-empty $P$.
    Inductively define $p_\star : \N \to P$:
    \begin{itemize}
        \item Take $p_0$ to be the element of non-empty $P$.
        \item If $n + 1$ is odd then ensure $a_{n/2}$ 
            is in the domain:
            by the `forth' property of $P$ there exists $p_{n+1}$ 
            extending $p_n$ such that $a_{n/2} \in \dom(p_{n+1})$.
        \item If $n + 1$ is even then ensure $b_{(n+1)/2}$ is in the codomain:
            by the `back' property of $P$ there exists $p_{n+1}$ 
            extending $p_n$ such that $b_{(n-1)/2} \in \codom(p_{n+1})$.
    \end{itemize}
    We claim that $p$, the union of the partial isomorphisms 
    $p_n$ for each $n \in \N$, is an isomorphism.
    Note that it is well-defined and has image $\NN$ as the $p_i$ are nested and
    for any $a_i \in \MM$ and $b_i \in \NN$, 
    $a_i \in \dom(p_{2i+1})$ and $b_i \in \dom(p_{2i+2})$.
    It is injective: if $a_i, a_j \in \MM$ and $p(a_i) = p(a_j)$ then 
    $p_{2i+2}(a_i) = p_{2i+2}(a_j)$ and so $a_i = a_j$ as $p_{2i+2}$ is a 
    partial isomorphism.
    One can easily show that it is an isomorphism.
\end{proof}

\begin{dfn}[$p$-equivalence, $P$-equivalence]%??
    Let $\MM$ and $\NN$ be $\Si_R$-structures.
    Let $a \in \MM^{n}$ and $b \in \NN^{n}$.
    We say that $a$ and $b$ are $p$-equivalent when
    $p$ is a partial isomorphism with domain containing
    $\set{a_1,\dots,a_n}$ and codomain containing $\set{b_1,\dots,b_n}$
    and maps $a_i \mapsto b_i$.
    We write $(a, \MM) \sim_p (b, \NN)$.

    %?If $\MM \iso_P \NN$ and
    %?for all $p \in P$ we have 
    %?$(a,\MM) \sim_p (b, \NN)$
    %?then $a$ and $b$ are said to be $P$-equivalent (usually $\om$-equivalent).
    %?For this we write $(a,\MM) \sim_P (b, \NN)$.
    %?and we say that $a$ and $b$ have the same type 
    %?%?
    %?One should check that these are equivalence relations.
\end{dfn}

%\begin{dfn}[Quantifier rank]
%    The quantifier rank of a $\Si$-formula $\phi$,
%    $\qr(\phi)$ is defined inductively:
%    \begin{itemize}
%        \item If $\phi$ is atomic then it has rank $0$.
%        \item If $\phi$ is $\psi \OR \chi$ then it has rank 
%            $\max(\qr(\psi),\qr(\chi))$
%        \item If $\phi$ is $\forall v, \psi$ then it has rank $\qr(\psi) + 1$
%            if $v$ is a free-variable of $\psi$.
%            Otherwise it has rank $\qr(\psi)$.
%    \end{itemize}
%\end{dfn}

\begin{nttn}
    We write
    \[\Si_R := \set{\nothing,\nothing,n_\star,R,m_\star}\]
    as a signature with only relation symbols.
\end{nttn}

\begin{prop}[$p$-equivalent elements have the same type]%??
    \link{p_equiv_elements_are_same_type}
    Let $\MM$ and $\NN$ be $\Si_R$-structures.
    Let $a \in \MM^{n}$ and $b \in \NN^{n}$.
    If $a$ and $b$ are $p$-equivalent for some $p \in P$ then
    \[\subintp{\nothing,n}{\MM}{\tp}(a) = 
    \subintp{\nothing,n}{\NN}{\tp}(b)\]
    Thus $a$ and $b$ `have the same type'.
\end{prop}
\begin{proof}
    \begin{forward}
    First note that $\Si_R$ terms can only be variable symbols as there are no 
    constant symbols or function symbols.
    Let $\phi$ be a $\Si_R$-formula with up to $n$ variables.
    We show by induction on $\phi$ that 
    \[
        \MM \model{\Si_R} \phi(a)
        \iff \NN \model{\Si_R} \phi(p(a))
        \iff \NN \model{\Si_R} \phi(b)
    \]
    \begin{itemize}
        \item If $\phi$ is $\top$ it is trivial.
        \item If $\phi$ is $t = s$ then they are variable symbols $x,y$
            and $\phi$ has $2$ free-variables. 
            \[
                \mmintp{t}(a_i) = \mmintp{s}(a_j)
                \iff a_i = a_j \iff p(a_i) = p(a_j) 
                \iff \nnintp{t}(p(a_i)) = \nnintp{s}(p(a_j))
            \]
        \item If $\phi$ is $r(t)$ then again noting that $t$ are all variable
            symbols
            \[  
                (a_{i_1},\dots,a_{i_j}) \in \mmintp{r} \iff 
                p(a_{i_1},\dots,a_{i_j}) \in \nnintp{r} \iff 
            \]
            as $p$ is a partial isomorphisms.
        \item If $\phi$ is $\NOT \psi$ or it is $\psi \OR \chi$ then it is 
            clear by induction.
        \item If $\phi$ is $\forall x, \psi$ 
            then suppose for every $\al \in \MM$ we have that 
            $\MM \model{\Si_R} \psi(\al)(a)$.
            Let $\be \in \NN$. 
            By the `back' property of $p \in P$, 
            there exists $q \in P$ such that $q$ extends $p$ and 
            $\be \in \codom(q)$, thus by the induction hypothesis
            \[
                \MM \model{\Si_R} \psi(q^{-1}(\be))(a) 
                \implies \NN \model{\Si_R} \psi(\be)(q(a))
                \implies \NN \model{\Si_R} \psi(\be)(p(a))
            \]
            thus $\NN \model{\Si_R} \forall x, \phi(a)$.
            The other direction is similar, 
            using the `forth' condition.
    \end{itemize}
    Hence $\MM \model{\Si_R} \phi(a)\iff \NN \model{\Si_R} \phi(b)$ and
    \[
        \subintp{\nothing,n}{\MM}{\tp}(a) = 
        \subintp{\nothing,n}{\NN}{\tp}(b)
    \]
    \end{forward}

    \begin{backward}
        If
        \[
            \subintp{\nothing,n}{\MM}{\tp}(a) = 
            \subintp{\nothing,n}{\NN}{\tp}(b)
        \]
        Then define $p : a \to b$ mapping $a_i \mapsto b_i$.
        This is a bijection and for any $\phi \in \form{\Si}$,
        \[\MM \model{\Si} \phi(a,\al) \iff 
        \phi \in \subintp{\nothing,n}{\MM}{\tp}(a) = 
        \subintp{\nothing,n}{\NN}{\tp}(b)
        \iff \NN \model{\Si} \phi(b,\be)\]
        In particular relation symbols are preserved under $p$,
        hence $p$ is a local isomorphism.
    \end{backward}
\end{proof}%%??

\begin{prop}[Elementary equivalence]
    Let $\MM$ and $\NN$ be $\Si_R$-structures.
    Then if $\MM$ and $\NN$ are partially isomorphic via some $P$
    then they are elementarily equivalent.
\end{prop}
\begin{proof}
    Let $p \in P$.
    Since \linkto{p_equiv_elements_are_same_type}{$p$-equivalent
        elements have the same type},
    we can take $a \in \MM^0,b \in \NN^0$ and
    note that $a$ and $b$ are trivially $p$-equivalent 
    (they are both $\nothing$).
    Hence 
    \[\Theory_\MM(\nothing) = \subintp{\nothing,n}{\MM}{\tp}(a) = 
    \subintp{\nothing,n}{\NN}{\tp}(b) = \Theory_\NN(\nothing)\]
    Thus $\MM \equiv_{\Si_R} \NN$.
\end{proof}

\begin{dfn}[$\infty$-equivalence]
    Let $\MM$ and $\NN$ be $\Si$-structures.
    We say $\MM$ and $\NN$ are $\infty$-equivalent if 
    for any $a \in \MM^n$ and $b \in \NN^n$ such that 
    \[\subintp{\nothing,n}{\MM}{\tp}(a) = 
    \subintp{\nothing,n}{\NN}{\tp}(b)\]
    we have 
    \begin{itemize}
        \item Forth: 
        For any $\al \in \MM$ there exists a $\be \in \NN$ such that 
        \[\subintp{\nothing,n+1}{\MM}{\tp}(a,\al) = 
        \subintp{\nothing,n+1}{\NN}{\tp}(b,\be)\]
        \item Back:
        For any $\be \in \NN$ there exists a $\al \in \MM$ such that 
        \[\subintp{\nothing,n+1}{\MM}{\tp}(a,\al) = 
        \subintp{\nothing,n+1}{\NN}{\tp}(b,\be)\]
    \end{itemize}
\end{dfn}

\begin{prop}
    Let $\MM$ and $\NN$ be $\Si$-structures.
    If $\MM$ and $\NN$ are $\om$-saturated %?and elementarily equivalent
    then they are $\infty$-equivalent.
\end{prop}
\begin{proof}
    We only prove `forth'.
    Let $a \in \MM^n$ and $b \in \NN^n$ such that 
    \[\subintp{\nothing,n}{\MM}{\tp}(a) = 
    \subintp{\nothing,n}{\NN}{\tp}(b)\]
    Let $\al \in \MM$ and consider 
    \[p(a,v) := \subintp{a,1}{\MM}{\tp}(\al) \in S_1(\Theory_\MM(a))\]
    Any formula in $p(a,v)$ can be written as a 
    $\Si$-formula $\phi(w,v)$ with variables $w$
    replaced with elements of $a$ 
    ($v$ represents a single variable to be replaced by $\al$). 
    Let 
    \[p(w,v) := \set{\phi(w,v) \st \phi(a,v) \in p(a,v)}\]
    We claim that 
    \[p(b,v) := \set{\phi(b,v) \st \phi \in p(w,v)} \in S_1(\Theory_\NN(b))\]
    To this end, we note that it is indeed a subset of 
    and it is a maximal subset of it since for any $\phi(b) \in F(\Si(b),1)$
    \[\phi(a) \in p(a,v) \text{ or } \NOT \phi(a) \in p(a) \implies 
    \phi(b) \in p(b,v) \text{ or } \NOT \phi(b) \in p(b)\]
    We just need to show that it is consistent with $\Theory_\NN(b)$.

    By \linkto{compactness_for_types}{compactness for types} 
    and noting that $\NN$ is a $\Si(b)$-model of $\Theory_\NN(b)$,
    it suffices to show
    that for any finite subset $\De(w,v) \subs p(w,v)$ 
    there exists $\be \in \NN^m$ such that 
    $\NN \model{\Si(b)} \De(b,\be)$.
    \begin{align*}
        &\MM \model{\Si(a)} \bigand{\phi \in \De}{} \phi(a,\al)\\
        \implies &\MM \model{\Si} \exists v, \bigand{\phi \in \De}{} \phi(a,v)\\
        \implies &\brkt{\exists v, \bigand{\phi \in \De}{} \phi(a,v)} \in 
        \subintp{\nothing,n}{\MM}{\tp}(a) = 
        \subintp{\nothing,n}{\NN}{\tp}(b)\\
        \implies &\NN \modelsi \exists v, \bigand{\phi \in \De}{} \phi(b,v)\\
        \implies &\exists \be \in \NN, 
        \NN \modelsi \bigand{\phi \in \De}{} \phi(b,\be)\\
        \implies &\exists \be \in \NN, \NN \model{\Si(b)} \De(b,\be)
    \end{align*}
    Thus $p(b,v)\in S_1(\Theory_\NN(b))$ and since $\NN$ is $\om$-saturated
    $p(b,v)$ is realised in $\NN$ by some $\be$.
    Thus by maximality, $p(b,v) = \subintp{b,1}{\NN}{\tp}(\be)$.

    Finally, for $\phi(v,w) \in F(\Si,n+1)$
    \begin{align*}
        &\phi(v,w) \in \subintp{\nothing,n+1}{\MM}{\tp}(a,\al)
        \iff &\MM \model{\Si} \phi(a,\al) \iff \MM \model{\Si(a)} \phi(a,\al)\\
        \iff &\phi(a,v) \in \subintp{a,1}{\MM}{\tp}(\al) = p(a,v)\\
        \iff &\phi(b,v) \in p(b,v) = \subintp{b,1}{\MM}{\tp}(\be)\\
        \iff &\NN \model{\Si(b)} \phi(b,\be) \iff \NN \modelsi \phi(b,\be)\\
        \iff &\phi(w,v) \in \subintp{\nothing,n+1}{\NN}{\tp}(b,\be)
    \end{align*}
\end{proof}

%\begin{dfn}[Quantifier rank]
%    The quantifier rank of a $\Si$-formula $\phi$,
%    $\qr(\phi)$ is defined inductively:
%    \begin{itemize}
%        \item If $\phi$ is atomic then it has rank $0$.
%        \item If $\phi$ is $\psi \OR \chi$ then it has rank 
%            $\max(\qr(\psi),\qr(\chi))$
%        \item If $\phi$ is $\forall v, \psi$ then it has rank $\qr(\psi) + 1$
%            if $v$ is a free-variable of $\psi$.
%            Otherwise it has rank $\qr(\psi)$.
%    \end{itemize}
%\end{dfn}

%\begin{nttn}
%    We write
%    \[\Si_R := \set{\nothing,\nothing,n_\star,R,m_\star}\]
%    as a signature with only relation symbols.
%\end{nttn}

%%%%%%%%%%%%%%%%%%%%%%%%%%%%%%%%%%%%%%%%%%%%%%%%%%%%%%%%%%