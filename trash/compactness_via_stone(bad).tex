\subsection{Ultraproducts and Łos's Theorem}
This section introduces ultrafilters and ultraproducts and uses Łos's Theorem
to prove the compactness theorem.
\begin{dfn}[Partially ordered set]
    The signature of partially ordred sets $\Si_\PO$ consists of 
    $(\nothing,\nothing ,n_f,\set{\leq},m_f)$,
    where $n_\leq = 2$.
    The theory of partially ordered sets $\PO$ consists of 
    \begin{itemize}
        \item[$\vert$] Reflexivity: 
            $\forall x, 
            x \leq x$ (this is just notation for $\leq(x,x)$)
        \item[$\vert$] Antisymmetry: 
            $\forall x \forall y, 
            (x \leq y \AND y \leq x) \to x = y$
        \item[$\vert$] Transitivity:
            $\forall x \forall y \forall z, 
            (x \leq y \AND y \leq z) \to x \leq z$
    \end{itemize}
\end{dfn}

\begin{dfn}[Boolean algebra]
    The signature of Boolean algebras $\Si_\BLN$ consists of 
    $(\set{1,0},\set{\leq,\cap,\cup,\NEG},n_f,\nothing,m_f)$,
    where $n_\leq = 2$, $n_\cap = n_\cup = 2$ and $n_\NEG = 1$.
    The theory of Boolean algebras $\BLN$ consists of the theory of 
    partially ordered sets $\PO$ together with the formulas
    \begin{itemize}
        \item[$\vert$] Assosiativity of adjunction: 
            $\forall x \forall y \forall z, 
            (x \cap y) \cap z = x \cap (y \cap z)$
        \item[$\vert$] Assosiativity of disjunction: 
            $\forall x \forall y \forall z, 
            (x \cup y) \cup z = x \cup (y \cup z)$
        \item[$\vert$] Identity for adjunction:
            $\forall x, x \cap 1 = x$ 
        \item[$\vert$] Identity for disjunction:
            $\forall x, x \cup 0 = x$ 
        \item[$\vert$] Commutativity of adjunction: 
            $\forall x \forall y, x \cap y = y \cap x$
        \item[$\vert$] Commutativity of disjunction: 
            $\forall x \forall y, x \cup y = y \cup x$
        \item[$\vert$] Distributivity of adjunction:
            $\forall x \forall y \forall z, 
            x \cap (y \cup z) = (x \cap y) \cup (x \cap z)$
        \item[$\vert$] Distributivity of disjunction:
            $\forall x \forall y \forall z, 
            x \cup (y \cap z) = (x \cup y) \cap x (\cup z)$
        \item[$\vert$] Negation on adjunction: 
            $\forall x, x \cap \NEG x = 0$ 
        \item[$\vert$] Negation on disjunction: 
            $\forall x, x \cup \NEG x = 1$ 
        \item[$\vert$] Order on adjunction:
            $\forall x \forall y, (x \cap y) \leq x$
        \item[$\vert$] Maximal property of adjunction:
            $\forall x \forall y \forall z, 
            (x \leq y) \AND (x \leq z) \to (x \leq y \cap z)$ 
        \item[$\vert$] Order on disjunction:
            $\forall x \forall y, x \leq (x \cup y)$ 
        \item[$\vert$] Minimal property of disjunction:
            $\forall x \forall y \forall z, 
            (x \leq z) \AND (y \leq z) \to (x \cup y \leq z)$ 
        \end{itemize}
    Often `absorption' is also included, 
    but it can be deduced from the other axioms.
    I have not used the usual logical symbols due to obvious clashes with 
    our notation, 
    and we will be using this in the context of sets anyway.
    If $B$ is a $\Si_\BLN$-model of $\BLN$ we call it a Boolean algebra.
\end{dfn}
\begin{dfn}[Filters and ultrafilters on Boolean algebras]
    Let $B$ be a Boolean algebra.
    A subset $\FF$ of $B$ is an filter on $B$ if
    \begin{itemize}
        \item $1 \in \FF$.
        \item For any two members $a,b \in \FF$ the adjunction
            $a \cap b$ is in $\FF$. (Closure under finite intersection.)
        \item If $a \in \FF$ then any $b \in B$ 
            such that $a \leq b$ is also a member of $\FF$. 
            (Closure under superset.)
    \end{itemize}
    We say a filter on $B$ is proper if it does not contain $0$.
    A proper filter $\FF$ on $B$ is an ultrafilter (maximal filter) when
    for any filter $\GG$ on $B$, if $\FF \subs \GG$ then $\GG = \FF$ 
    or $\GG = B$.
\end{dfn}

\begin{dfn}[Filters on sets]
    Let $X$ be a set. 
    The power set of $X$ is a Boolean algebra with $0$ interpreted as 
    $\nothing$, 
    $1$ interpreted as $X$ and $\leq$ interpreted as $\subs$.
    We say $\FF$ is an filter on $X$ if it is filter on the power set
    as this Boolean algebra.
    The definition translates to:
    \begin{itemize}
        \item $X \in \FF$.
        \item For any two members of $\FF$ their intersection is in $\FF$.
        \item If $a \in \FF$ then any $b$ in the power set of $X$ 
            such that $a \subs b$ is also a member of $\FF$.
    \end{itemize}
    A filter on $X$ is proper if and only if it does not contain the empty set,
    if and only if the filter is not equal to the power set.
    A proper filter $\FF$ is an ultrafilter if and only if
    for any filter $\GG$, 
    if $\FF \subs \GG$ then $\FF = \GG$ 
    or $\GG$ is the power set of $X$.
\end{dfn}

\begin{dfn}[Ultraproduct]
    Let $\FF$ be an ultrafilter on $X$.
    We define a relation on $\prod_{x \in X} x$ by 
    \[
        (a_x)_{x \in X} \sim (b_x)_{x \in X} 
        := \set{x \in X \st a_x = b_x} \in \FF
    \]
    This is an equivalence relation as 
    \begin{itemize}
        \item $(a_x)_{x \in X} \sim (a_x)_{x \in X} \iff 
            \set{x \in X \st a_x = a_x} = X \in \FF$ 
        \item Symmetry is due to symmetry of $=$.
        \item If $\set{x \in X \st a_x = b_x} \in \FF$ and 
            $\set{x \in X \st b_x = c_x} \in \FF$ then 
            $\set{x \in X \st a_x = b_x = c_x}$ is their intersection
            and so is in $\FF$.
            Thus its superset $\set{x \in X \st a_x = c_x}$ is in $\FF$.
    \end{itemize}
    We define the ultraproduct of $X$ over $\FF$:
    \[\prod X / \FF := \prod_{x \in X} x / \sim\]
\end{dfn}

%\begin{dfn}[Filter basis]
%    
%    Let $X$ be a set and $\BB$ a subset of its power set. 
%    We say $\BB$ is a proper filter basis on $X$ if 
%    $\nothing \notin \BB, \nothing \ne \BB$ and for any two elements 
%    $V_0,V_1 \in \BB$ there exists $U \in \BB$ such that $U \subs V_0 \cap V_1$.
%
%    Let \[\FF = V \subs X \st \exists U \in \BB, U \subs V\]
%    then 
%    \[\FF \text{ is a proper fiter on }X \iff 
%    \BB \text{ is a proper filter basis on }X\]
%
%    Clearly any filter that contains $\BB$ contains the filter $\FF$.
%\end{dfn}
%\begin{proof}
%    \begin{forward}
%        Suppose $\nothing \in \BB$ then $\FF$ is the power set of $X$, 
%        which is false.
%        Suppose $\nothing = \BB$ then $\FF$ is empty which is also false.
%        Let $V_0,V_1 \in \BB$ then each of them are in $\FF$ 
%        by taking themselves as the subsets from $\BB$.
%        Thus their intersection is in $\FF$, 
%        but then by definition of $\FF$
%        \[\exists U \in B, U \subs V_0 \cap V_1\]
%    \end{forward}
%    \begin{backward}
%        $\BB$ is non-empty and so $X \in \FF$.
%        Finite intersection: if $V_0,V_1 \in \FF$ then respectively there
%        exist $U_0,U_1 \in \BB$ that are subsets of $V_0,V_1$.
%        By the property of $\BB$ there exists a $U \in \BB$ such that 
%        $U \subs U_0 \cap U_1 \subs V_0 \cap V_1$ 
%        hence the intersection is in $\FF$.
%        Closure in superset is trivial.
%        Lastly it is proper since $\nothing \notin \BB$.
%    \end{backward}
%\end{proof}

\begin{lem}[Technical detail]
    \link{technical_lemma_w_le_neg_u}
    Let $B$ be a boolean algebra.
    If $a,b \in B$ are such that $a \cap b = 0$ then $b \leq \NEG a$.
\end{lem}
\begin{proof}
    \begin{align*}
        &a \cup \NEG a = 1\\
        \implies & b \cap (a \cup \NEG a) = b \cap 1 = b\\
        \implies & (b \cap a) \cup (b \cap \NEG a) = b\\
        \implies & 0 \cup (b \cap \NEG a) = b\\
        \implies & b = b \cap \NEG a \leq \NEG a
    \end{align*}
\end{proof}

\begin{prop}[Equivalent definition of ultrafilter]
    \link{negation_is_in_ultrafilter}
    Let $B$ be a boolean algebra. 
    Let $\FF$ be a proper filter on $B$.
    $\FF$ is an ultrafilter over $B$ if and only if for every element 
    $a$ of $B$,
    \emph{either} $a \in \FF$ or $(\NEG a) \in \FF$.
\end{prop}
\begin{proof}
    \begin{forward}
        Let $a \in B$ then take 
        \[\GG_a = \set{b \in B \st \exists c \in \FF, c \cap a \leq b}\]
        Then clearly $\GG_a$ is a filter containing $a$.
        Since $\FF$ is an ultrafilter, 
        we can case on whether $\GG_a$ is $\FF$ or $B$.
        If $\GG_a = B$ set then it contains $0$ and so 
        there exists $c \in \FF$ such that 
        $c \cap a = 0$.
        \linkto{technical_lemma_w_le_neg_u}{Hence $c \leq \neg a$,}
        and so $(\NEG a) \in \FF$.
        If $\GG_a$ is $\FF$ then $a \in \FF$.

        This is an `exclusive or' since if both $a$ 
        and its negation were in $\FF$ then $\FF$ would not be proper.
    \end{forward}
    \begin{backward}
        Let $\GG$ be a proper filter such that $\FF \subs \GG$.
        It suffices to show that $\GG = \FF$.
        Then let $a \in \GG$.
        Suppose $a \notin \FF$.
        Then $\NEG a \in \FF$ and so $\NEG a \in \GG$.
        Thus $0 = (a \cap \NEG a) \in \GG$ 
        thus $\GG = \FF$, a contradiction.
    \end{backward}
\end{proof}

\begin{prop}[Equivalent definition of ultrafilter (translated to the power set)]
    \link{complement_is_in_ultrafilter}
    Let $X$ be a set. 
    Let $\FF$ be a proper filter on $X$.
    $\FF$ is an ultrafilter over $X$ if and only if for every subset 
    $U \subs X$, 
    \emph{either} $U \in \FF$ or $X \setminus U \in \FF$.
\end{prop}
\begin{proof}
    Follows immediately from \linkto{negation_is_in_ultrafilter}{the 
    equivalent definition of ultrafilter}.
\end{proof}

\begin{prop}[Łos's Theorem]
    \link{los_theorem}
    Let $\f{M} \subs \struc{\Si}$ where $\Si$ is a signature
    such that \emph{each carrier set is non-empty}.
    Let \[X = \set{{\MM} \st \MM \in \f{M}}\]
    Write $\MM = {\MM}$ and $\f{M} = X$ to make things look nicer.
    Suppose $\FF$ is an ultrafilter on $\f{M}$ 
    (i.e. an ultrafilter on the Boolean algebra $P(\f{M})$).
    Then we want to make $\NN := \prod \f{M} / \FF$ into a $\Si$-structure.
    Let $\pi$ be the natural surjection 
    $\prod_{\MM \in \f{M}} \MM \to \prod \f{M} / \FF$.
    If $a = (a_1, \dots, a_n) \in \prod_{\MM \in \f{M}} \MM$ then
    write $a_\MM := ((a_1)_\MM, \dots, (a_n)_\MM)$.
    \begin{itemize}
        \item Constant symbols $c \in \const{\Si}$ are interpreted as 
            \[\modintp{\NN}{c} := \pi(\modintp{\MM}{c})_{\MM \in \f{M}}\]
        \item Any function symbol $f \in \func{\Si}$ is interpreted as the 
            function
            \[\modintp{\NN}{f} : 
            \brkt{\prod \f{M} / \FF}^n \to \prod \f{M} / \FF
            := \pi\brkt{(a_\MM)_{\MM \in \f{M}}} \mapsto 
            \pi(f^\MM (a_\MM))_{\MM \in \f{M}}\]
            where $\pi\brkt{(a_\MM)_{\MM \in \f{M}}} = 
            \brkt{\pi((a_i)_\MM)_{\MM \in \f{M}}}_{i=1}^n$.
        \item Any relation symbols 
            $r \in \rel{\Si}$ is interpreted as the subset such that 
            \[\pi(a) \in \modintp{\NN}{r}
            \iff \set{\MM \in \f{M} \st 
            a_\MM \in \modintp{\MM}{r}} \in \FF\]
            where $a = a_1,\dots, a_m$ and $\pi(a) = \brkt{\pi(a_i)}_{i=1}^m$.
    \end{itemize}
    Then for any $\Si$-formula $\phi$ with free variables indexed by $S$,
    If $a = (a_1, \dots, a_n) \in \prod_{\MM \in \f{M}} \MM$ then 
    \[\NN \model{\Si} \phi(\pi(a)) \iff 
    \set{\MM \in \f{M} \st \MM \model{\Si} \phi(a_\MM)} \in \FF\]
\end{prop}
\begin{proof}
    We show that the interpretation of functions is well defined.
    Let $a, b \in \brkt{\prod_{\MM \in \f{M}}}^{n_f}$.
    Suppose for each $i \in \set{1,\dots,n_f}$, 
    $a_i \sim b_i \in \prod_{\MM \in \f{M}} \MM$.
    Then 
    \begin{align*}
        &\text{for each } i, 
        \set{\MM \in \f{M} \st (a_i)_\MM = (b_i)_\MM} \in \FF \\
        &\implies
        \set{\MM \in \f{M} \st \bigand{i = 1}{n} (a_i)_\MM = (b_i)_\MM} \in \FF
        \text{ by closure under finite adjunction}\\
        &\implies
        \set{\MM \in \f{M} \st a_\MM = b_\MM} \in \FF\\
        &\implies
        \set{\MM \in \f{M} \st 
            \modintp{\MM}{f}(a_\MM) = \modintp{\MM}{f}(b_\MM)} \in \FF
        \text{ by closure under superset}\\
        &\implies \pi(\mmintp{f}(a_\MM)) = \pi(\mmintp{f}(b_\MM))
        \text{ by definition of the quotient}
    \end{align*}

    We use the following claim:
    If $t$ is a term with variables $S$ and 
    $a \in (\prod_{\MM \in \f{M}} \MM)^S$
    then there exists $b \in \prod_{\MM \in \f{M}} \MM$ such that 
    \[\modintp{\NN}{t}\circ \pi(a) = \pi(b) \text{ and } 
    \forall \MM \in \f{M}, \modintp{\MM}{t}(a_\MM) = b_\MM\]
    We prove this by induction on $t$:
    \begin{itemize}
        \item If $t$ is a constant symbol $c$ then pick 
            $b := (\modintp{\MM}{c})_{\M \in \f{M}}$.
        \item If $t$ is a variable then let 
            $a \in \prod_{\MM \in \f{M}} \MM$ (only one varible), 
            pick $b := a$.
        \item If $t$ is a $f(s)$ then let $a \in (\prod_{\MM \in \f{M}} \MM)^S$
            by the inducition hypothesis there exists 
            $c \in \prod_{\MM \in \f{M}} \MM$ such that 
            \[\modintp{\NN}{s} \circ \pi(a) = \pi(c) \text{ and } 
            \forall \MM \in \f{M}, \modintp{\MM}{s}(a_\MM) = c_\MM\]
            Then we can take 
            $b = \brkt{\modintp{\MM}{f}(c_\MM)}_{\MM \in \f{M}}$. 
            Thus 
            \[\nnintp{t}\circ \pi(a) = \nnintp{f}(\nnintp{s}\circ (\pi(a)))
            = \nnintp{f}(\pi(c)) = \pi(\mmintp{f}(c_\MM))_{\MM \in \f{M}}
            = \pi(b)\]
            and for any $\MM \in \f{M}$,
            \[\mmintp{t}(a_\MM) = \mmintp{f} \circ \mmintp{s} (a_\MM)
            = \mmintp{f}(c_\MM) = b_\MM\]
    \end{itemize}
    We now induct on $\phi$ to show that for any appropriate $a$,
    \[\NN \model{\Si} \phi(\pi(a)) \iff 
    \set{\MM \in \f{M} \st \MM \model{\Si} \phi(a_\MM)} \in \FF\]
    \begin{itemize}
        \item The case where $\phi$ is $\top$ is trivial 
            (noting that anything models $\top$ and $\f{M} \in \FF$).
        \item If $\phi$ is $s = t$ then it suffices to show that 
        \[  
            \modintp{\NN}{s}\circ \pi(a) = \modintp{\NN}{t}\circ \pi(a)
            \iff 
            \set{\MM \in \f{M} \st \MM \model{\Si} \modintp{\MM}{t}(a_\MM) = 
            \modintp{\MM}{s}(a_\MM)} \in \FF
        \]
        \begin{forward}
            If for two terms $s,t$ we have
            $\modintp{\NN}{s}\circ \pi(a) = \modintp{\NN}{t}\circ \pi(a)$
            then by the claim
            there exists $b \in \prod_{\MM \in \f{M}} \MM$ such that
            \[  
                \f{M} = 
                \set{\MM \in \f{M} \st 
                \modintp{\MM}{s}(a_\MM) = b_\MM = \modintp{\MM}{t}(a_\MM)} 
                = 
                \set{\MM \in \f{M} \st 
                    \modintp{\MM}{t}(a_\MM) = \modintp{\MM}{s}(a_\MM)} 
            \]
            which is therefore in the filter $\FF$.
        \end{forward}
        \begin{backward}
            If for two terms $s,t$ we have
            $\modintp{\NN}{s}\circ \pi(a) \ne \modintp{\NN}{t}\circ \pi(a)$
            then by the claim there exist 
            $b \ne c \in \prod_{\MM \in \f{M}} \MM$ such that
            \[  
                \set{\MM \in \f{M} \st 
                    \modintp{\MM}{t}(a_\MM) = \modintp{\MM}{s}(a_\MM)} =
                \set{\MM \in \f{M} \st b_\MM = c_\MM} = 
                \nothing
            \]
            which is not in the filter $\FF$ as it is proper.
        \end{backward}
        \item If $\phi$ is $r(t)$ then by the claim we have 
            $b \in \prod_{\MM \in \f{M}} \MM$ with the desired properties.
            It suffices to show that 
            \[\pi(b) \in \nnintp{r} \iff 
            \set{\MM \in \f{\MM} \st b_\MM \in \mmintp{r}} \in \FF\]
            This follows from our definition of 
            interpretation of relation symbols.
        \item If $\phi$ is $\NOT \psi$ then 
            $\NN \model{\Si} \phi(\pi(a))$ if and only if 
            $\set{\MM \in \f{M} \st \MM \model{\Si} \psi(a_\MM)} \notin \FF$
            by induction.
            This holds if and only if its 
            \linkto{complement_is_in_ultrafilter}{complement is in the filter}:
            $\set{\MM \in \f{M} \st \MM \nodel{\Si} \psi(a_\MM)} \in \FF$
            which is if and only if 
            $\set{\MM \in \f{M} \st \MM \model{\Si} \phi(a_\MM)} \notin \FF$
        \item Without loss of generality we can use $\AND$ instead of $\OR$
            to make things simpler 
            (replacing this comes down to dealing with a couple of $\NOT$
            statements).
            If $\phi$ is $\psi \AND \chi$ then one direction 
            follows filters being closed under intersection:
            \begin{align*}
                &\NN \model{\Si} \phi(\pi(a))\\
                &\iff 
                \set{\MM \in \f{M} \st \MM \model{\Si} \psi(a_\MM)} \in \FF
                \text{ and }
                \set{\MM \in \f{M} \st \MM \model{\Si} \chi(a_\MM)}\in \FF\\
                &\implies 
                \set{\MM \in \f{M} \st \MM \model{\Si} \psi(a_\MM)}
                \cap 
                \set{\MM \in \f{M} \st \MM \model{\Si} \chi(a_\MM)}\in \FF\\
                &\iff
                \set{\MM \in \f{M} \st \MM \model{\Si} 
                \psi(a_\MM) \AND \chi(a_\MM)}\in \FF
            \end{align*}
            To make second implication a double implication we note that 
            each of the two sets 
            \[\set{\MM \in \f{M} \st \MM \model{\Si} \psi(a_\MM)} \in \FF
            \text{ and }
            \set{\MM \in \f{M} \st \MM \model{\Si} \chi(a_\MM)}\in \FF\]
            are supersets of the intersection which is in $\FF$.
        \item Without loss of generality we can use $\exists$ instead of 
            $\forall$ to make things simpler.
            \begin{forward}
                Suppose $\NN \model{\Si} \exists v, \psi(\pi(a),v)$.
                Then there exists $b \in \prod_{\MM \in \f{M}} \MM$ such that 
                $\NN \model{\Si} \psi(\pi(a),\pi(b))$.
                Then by induction 
                \[\set{\MM \in \f{M} \st \MM \model{\Si} \psi(a_\MM,b_\MM)}
                \in \FF\]
                This is a subset of 
                \[  
                    \set{\MM \in \f{M} \st \exists c \in \MM, \MM \model{\Si} 
                    \psi(a_\MM,c)} = 
                    \set{\MM \in \f{M} \st \MM \model{\Si} 
                    \exists v, \psi(a_\MM,v)}
                \]
            \end{forward}
            \begin{backward}
                Suppose $Y:=\set{\MM \in \f{M} \st \MM \model{\Si} 
                \exists v, \psi(a_\MM,v)} \in \FF$.
                Then by the axiom of choice we have for each $\MM$
                \[\begin{cases}
                    b_\MM \in \MM, \MM \model{\Si} &\text{if } \MM \in Y\\
                    b_\MM \in \MM   &\MM \notin Y
                \end{cases}\]
                since each $\MM$ is non-empty.
                By induction we have 
                $\NN \model{\Si} \psi(\pi(a),\pi(b))$
                and so $\NN \model{\Si} \exists v, \psi(\pi(a),v)$
            \end{backward}
    \end{itemize}
\end{proof}

\begin{cor}[The Compactness Theorem]
    A $\Si$-theory is consistent if and only if it is finitely consistent.
\end{cor}
\begin{proof}
    Suppose $T$ is finitely consistent.
    For each finite subset $\De \subs T$ we let $\MM_\De$ be the given 
    non-empty model of $\De$,
    which exists by finite consistency.
    We generate an ultrafilter $\FF$ on 
    $\f{M} := \set{\MM_\De \st \De \in I}$ and use 
    \linkto{los_theorem}{Łos's Theorem} to show that
    $\prod \f{M} / \FF$ is a model of $T$.
    Let
    \[I = \set{\De \subs T \st \De \text{ finite}}
    \quad \text{ and } [\star] : I \to \PP(I) := 
    \De \mapsto \set{\Ga \in I \st \De \subs \Ga}\]
    Writing $[I]$ for the image of I,
    we claim that $\FF := \set{U \in \PP(I) \st \exists V \in [I], V \subs U}$
    forms an ultrafilter on $I$ 
    (i.e. an ultrafilter on the Boolean algebra $\PP(I)$).
    Indeed 
    \begin{itemize}
        \item $\nothing \in I$ thus $I = \set{[\nothing]  \in [I] \subs \FF}$.
        \item Suppose $\nothing \in \FF$ then $\nothing \in [I]$ and so there 
            exists $\De \in I$ such that $[\De] = \nothing$.
            This is a contradiction as $\De \in [\De]$.
        \item If $U, V \in \FF$ then there exist $\De_U, \De_V \in I$ 
            such that $[\De_U] \subs U$ and $[\De_V] \subs V$. 
            \begin{align*}
                [\De_U] \cap [\De_V] &
                = \set{\Ga \in I \st \De_0 \subs \Ga 
                \text{ and } \De_1 \subs \Ga}\\
                &= \set{\Ga \in I \st \De_0 \cup \De_1 \subs \Ga}\\
                &= [\De_0 \cup \De_1] \in [I] \subs \FF
            \end{align*}
        \item Closure under superset is clear.
    \end{itemize}
    We identify each $\MM_\De \in \f{M}$ with $\De \in I$ and generate
    the same filter (which we will still call $\FF$) on $\f{M}$ 
    (this is okay as the power sets are isomorphic as Boolean algebras.)
    By \linkto{los_theorem}{Łos's Theorem} $\prod \f{M} / \FF$ is 
    a well-defined $\Si$-structure such that for any $\Si$-sentence $\phi$
    \[\prod \f{M} / \FF \model{\Si} \phi \iff 
    \set{\MM \in \f{M} \st \MM \model{\Si} \phi} \in \FF\]
    Let $\phi \in T$, 
    then $\set{\De \in I \st \set{\phi} \subs \De} \in \FF$ and so
    \[
        \set{\De \in I \st \set{\phi} \subs \De} \subs 
        \set{\De \in I \st \phi \in \De} \subs
        \set{\De \in I \st \MM \model{\Si} \phi} \in \FF
    \]
    The image of this under the isomorphism is 
    $\set{\MM_\De \in X \st \MM \model{\Si} \phi}$ thus is in $\FF$ and so 
    $\prod \f{M} / \FF \model{\Si} \phi$.
\end{proof}

\subsection{Boolean algebras and Stone Duality}
\begin{dfn}
    The category of Boolean algebras consists of Boolean algebras as the objects
    and for any two Boolean algebras $B, C$ morphisms $f:B \to C$ such that
    for any $a,b \in B$
    \[f(0) = 0, f(1) = 1, f(a \cup b) = f(a) \cup f(b), 
    f(a \cap b) = f(a) \cap f(b)\] 
\end{dfn}

\begin{lem}[Facts about Boolean algebras]
    \link{bare_boolean_facts}
    Let $B$ be a Boolean algebra, let $\FF$ be an ultrafilter
    on $B$, let $a,b \in B$ and let $f : B \to C$ be a morphism.
    \begin{itemize}
        \item $a \cap a = a$ and $a \cup a = a$.
        \item $a \cup 1 = 1$ and $a \cap 0 = 0$.
        \item If $a \cap b = 0$ and $a \cup b = 1$ then $a = \NEG b$
            (negations are unique.)
        \item $\NEG (a \cup b) = (\NEG a) \cap (\NEG b)$ and its dual. 
            (De Morgan)
        \item ($a \in \FF$ or $b \in \FF$) if and only if $a \cup b \in \FF$.
        \item Morphisms are order preserving.
        \item Morphisms commute with negation.
        \item $a \cap b = 1$ if and only if $a = 1$ and $b = 1$.
            Similarly for $\cup$ with $0$.
    \end{itemize}
\end{lem}
\begin{proof}~
    \begin{itemize}
        \item We prove only $a \cap a = a$ as the other has the same proof.
            \[a \cap a = (a \cap a) \cup 0 = (a \cap a) \cup (a \cap \NEG a)
            = a \cap (a \cup \NEG a) = a \cap 1 = a\]
        \item Again, we prove only $a \cup 1 = 1$. Using the previous part,
            \[a \cup 1 = a \cup (a \cup \NEG a) = (a \cup a) \cup \NEG a
            = a \cup \NEG a = 1\]
        \item \begin{align*}
            a &= a \cup 0 = a \cup (b \cap \NEG b) = 
            (a \cup b) \cap (a \cup \NEG b) = 1 \cap (a \cup \NEG b)\\
            &= 0 \cup (a \cup \NEG b) = (b \cap \NEG b) \cup (a \cup \NEG b)
            = (b \cup a) \cup \NEG b = 0 \cup \NEG b \\
            &= \NEG b
        \end{align*} 
        \item By the previous part it suffices to show that 
            \[\brkt{(\NEG a) \cap (\NEG b)} \cap (a \cup b) = 0 \text{ and }
            \brkt{(\NEG a) \cup (\NEG b) } \cup (a \cup b) = 1\]
            These are clear.
        \item \begin{forward}
            In the case that $a \in \FF$ we have $a \leq a \cup b \in \FF$ 
            as $\FF$ is under superset. 
            The other case is the same.
        \end{forward}
        \begin{backward}
            Suppose $a \notin \FF$ and $b \notin \FF$ then
            \linkto{negation_is_in_ultrafilter}{$\NEG a \in \FF$ and 
                $\NEG b \in \FF$} hence $\NEG a \cap \NEG b \in \FF$
                by closure under intersection. 
                By the previous part this is equal to $\NEG (a \cup b)$,
                which implies $(a \cup b) \notin \FF$ as $\FF$ 
                is an \linkto{negation_is_in_ultrafilter}{ultrafilter}.
        \end{backward}
        \item Suppose $b \leq a$. 
            We show that $f(a) \leq f(b)$.
            By the minimal property of disjunction,
            \[b \leq a \AND a \leq a \implies b \cup a \leq a\]
            Clearly $a \leq b \cup a$ and so $a = b \cup a$
            Hence $f(b) \leq f(b) \cup f(a) = f(b \cup a) = f(a)$.
        \item Suppose $f(a) = b$.
            Then $f(a) \AND f(\NEG a) = f(a \AND \NEG a) = f(0) = 0$
            and $f(a) \OR f(\NEG a) = f(a \OR \NEG a) = f(1) = 1$.
            As negations are unique (shown above) 
            this gives us that $f(\NEG a) = \NEG f(a)$.
        \item Suppose $a \cap b = 1$ then $1 \leq a \cap b \leq a$ hence $1 = a$
            and similarly $1 = b$.
    \end{itemize}
\end{proof}

\begin{dfn}
    The category of Stone spaces has
    $0$-dimensional, 
    compact and Hausdorff topological spaces as objects and 
    continuous maps as morphisms.
\end{dfn}

\begin{prop}[Contravariant functor from Boolean algebras to Stone spaces]
    \link{bool_alg_to_stone_functor}
    The map
    $S(\star)$ that sends a Boolean algebra $B$ to the Stone space
    \[S(B) := \set{\FF \subs B \st \FF \text{ is an ultrafilter}}\]
    and a Boolean algebra morphism
    $f: A \to B$ to a continuous map of Stone spaces:
    \[S(f) := f^-1(\star) : S(B) \to S(A)\]
    is a contravariant functor from Boolean algebras to the category of Stone
    spaces.
\end{prop}
\begin{proof}
    We give the topology on $S(B)$ the topology generated by the clopen sets:
    Let $b \in B$, 
    then an element of the clopen basis is given by
    \[[b] := \set{\FF \subs B \st b \in \FF}\]
    Thus $S(B)$ is $0$-dimensional by definition.
    It is Hausdorff because if $\FF, \GG \in S(B)$ are not equal then 
    without loss of generality there exists $b \in \FF$ such that 
    $b \notin \GG$. 
    As ultrafilters are proper, $[b] \cap [\NEG b] = \nothing$ and so
    we have obtained disjoint open sets such that 
    $\FF \in [b]$ and $\GG \in [\NEG b]$.
    
    Next we show that it is compact. 
    To do this, 
    we will look at $S(B)$ as a subspace of the power set of $B$
    which is isomorphic to
    $2^B$, 
    the set of functions from $B$ to $2$.
    We endow $2 = \set{0,1}$ with the discrete topology and note that it is 
    compact.
    Then $2^B = \prod_{a \in B} 2$ has an induced product topology which is 
    compact by Tychonoff's Theorem.
    The isomorphism from $2^B$ to the power set of $B$ is given by 
    \[2^B \to \PP(B) := f \mapsto f^{-1}(1)\]
    We take the topology on $\PP(B)$ induced by this isomorphism.
    We must show that 
    \begin{itemize}
        \item $S(B)$ is the image of 
            $\mor{B}{2}{} := 
            \set{f \in 2^B \st f \text{ is a Boolean algebra morphism}}$
            under this isomorphism.
        \item $\mor{B}{2}{}$ is a closed subset of $2^B$ and therefore
            compact.
        \item The induced topology on $S(B)$ 
            is the same as its original topology.
    \end{itemize}
    With these facts we see that $S(B)$ is also compact.

    Let $f : B \to 2$ be a Boolean algebra morphism. 
    We must show that $\FF := f^{-1}(1)$ is an ultrafilter.
    Since $f(1) = 1$, $1 \in \FF$. 
    If $a,b \in \FF$ then $f(a) = f(b) = 1$ so 
    $f(a \cap b) = f(a) \cap f(b) = 1$ thus $\FF$ is closed under intersection.
    If $a \subs b$ and $a \in \FF$ then as $f$ is 
    \linkto{bare_boolean_facts}{order preserving} $f(a) \subs f(b)$.
    It is a proper filter as $f(0) = 0 \ne 1$.
    To show it is an ultrafilter we use the
    \linkto{negation_is_in_ultrafilter}{equivalent definition}:
    let $a \in \PP(B)$. 
    If $f(a) = 1$ then we are done,
    otherwise
    \[f(\neg a) = \neg f(a) = \neg 0 = 1 \implies \neg a \in \FF\]
    and we are done.
    To show that this is a surjection we use the inverse:
    \[\FF \mapsto \brkt{a \to \begin{cases}
        1, a \in \FF\\
        0, a \notin \FF
    \end{cases}}\] 
    To show that this is a morphism we note that $\FF$ 
    is proper and contains $1$ thus $f(0) = 0$ and $f(1) = 1$. 
    Also 
    \begin{align*}
        & f(a \cap b) = 1 \iff a \cap b \in \FF\\ 
        &\iff a \in \FF \text{ and } b \in \FF 
        \quad \text{ by closure under finite intersection and superset}\\
        &\iff f(a) = 1 \text{ and } f(b) = 1 \\
        &\iff f(a) \cap f(b) = 1 
        \quad \text{ \linkto{bare_boolean_facts}{ as proven before}}
    \end{align*}
    Hence $f(a \cap b) = f(a) \cap f(b)$.
    Lastly since 
    \[\NEG f(a) = 1 \iff f(a) = 0 \iff a \notin \FF \iff \NEG a \in \FF 
    \iff f(\NEG a) = 1\]
    thus by \linkto{bare_boolean_facts}{De Morgan} and 
    \linkto{bare_boolean_facts}{uniqueness of negations} we have
    \[f(a \cup b) = f(\NEG(\NEG a \cap \NEG b)) = 
    \NEG \sqbrkt{\NEG f(a) \cap \NEG f(b)} = 
    \brkt{\NEG \NEG f(a)} \cup \NEG \NEG f(b) = f(a) \cup f(b)\]
    Thus the inverse map gives back a Boolean algebra morphism and 
    $\mor{B}{2}{} \iso S(B)$ under the isomorphism.

    To show that $\mor{B}{2}{}$ is a closed subset we write it as 
    \[\mor{B}{2}{} = \set{f \st f(0) = 0} \cap \set{f \st f(1) = 1}
        \cap \set{f \st f \text{ commutes with $\cap$}} \cap 
        \set{f \st f \text{ commutes with $\cup$}}\]
    and show that all of these four sets are closed.
    Call them $\min,\max,C_{\cap},C_\cup$ respectively and for each $a \in B$
    call the projection map $\pi_a : 2^B \to 2$ (these send $f \mapsto f(a)$
    such that $\pi_a(f) = f(a)$)
    and note that by definition of the product topology each 
    $\pi_a$ is continuous; 
    the closed sets of the product are generated by preimages of closed sets.
    \[f(0) = 0 \iff \pi_0(f) = 0 \iff f \in \pi_0^{-1}(0)\]
    Hence $\min = \pi_0^{-1}(0)$ is closed as $\set{0}$ 
    is closed in the discrete topology on $2$.
    Similarly $\max = \pi_1^{-1}(1)$ is closed.
    \[C_\cap = \bigcap_{a,b \in B} \set{f \st f(a \cap b) = f(a) \cap f(b)}
    = \bigcap_{a,b \in B} \set{f \st \pi_{a \cap b}^{-1}(f(a) \cap f(b))}\]
    Thus $C_\cap$ is an arbitrary intersection of preimages of closed sets 
    since each ${f(a) \cap f(b)}$ is closed in the discrete toipology on $2$,
    hence $C_\cap$ is closed. 
    Similarly 
    \[C_\cup = \bigcap_{a,b \in B} 
    \set{f \st \pi_{a \cup b}^{-1}(f(a) \cup f(b))}\]
    is closed and so $\mor{B}{2}{}$ is closed.

    With regards to compactness it remains to show that the topologies 
    on $S(B)$ are the same.
    It suffices to show that any (closed) basis element of each can be written 
    as a closed set in the other.
    Let $[b]$ be an element of the basis for $S(B)$ under the Stone topology.
    Then this is the image of the closed subset 
    $\pi_b^{-1}(1) \subs \mor{B}{2}{}$ under the isomorphism:
    \[\mathrm{iso}(\pi_b^{-1}(1)) = \set{f^{-1}(1) \st f(b) = 1}
    = \set{\FF \text{ ultrafilter} \st b \in \FF}\]
    Conversely, 
    any element of the closed basis for $\mor{B}{2}{}$ is of the form 
    $\pi_b^{-1}(X)$ where $b \in B$ and $X \subs 2$.
    Hence 
    \[\mathrm{iso}(\pi_b^{-1}(X)) = \set{f^{-1}(1) \st f(b) \in X}\]
    We can case on if $X = \nothing, \set{0}, \set{1}, 2$ and deduce that 
    respectively $\mathrm{iso}(\pi_b^{-1}(X))$ 
    becomes $\nothing, [\NEG b], [b], S(B)$, 
    all of which are closed in the Stone topology.
    Thus the topologies are the same under this isomorphism and hence 
    $S(B)$ is compact.

    To show that $S(\star)$ 
    is a contravariant functor we need to check that the morphism map
    \[S(f) := f^-1(\star) : S(B) \to S(A)\]
    is a well-defined, respects the identity and composition. 
    We show that $S(f)$ is continuous: 
    it suffices that preimages of clopen elements are clopen.
    Let $[b] \subs S(A)$ be clopen. 
    \begin{align*}
        & S(f)^{-1}[b]\\
        =& \set{\FF \in S(B) \st f^{-1}(\FF) \in [b]}\\
        =& \set{\FF \in S(B) \st f(b) \in \FF}\\
        =& [f(b)]
    \end{align*}
    which is clopen.
\end{proof}

\begin{prop}[Stone Duality]
    There is an equivalence between the category of Stone
    spaces and the category of Boolean algebras.
    Given by the functor $\BB_\star$ 
    sending any topological space $X$ to the set of 
    its clopen subsets (this is a basis of $X$ as it is $0$ dimensional):
    \[\BB_X := \set{a \subs X \st a \text{ is clopen}}\]
    and its inverse \linkto{bool_alg_to_stone_functor}{$S(\star)$} .
\end{prop}
\begin{proof}
    Let $X$ be a $0$-dimensional compact Hausdorff topological space.
    There is an obvious Boolean algebra to take on 
    \[\BB_X := \set{a \subs X \st a \text{ is clopen}}\]
    which is interpreting $0$ to be $\nothing$, $1$ to be $X$, 
    $\leq$ as $\subs$,
    adjunction as intersection, disjunction as union and negation to be 
    taking the complement in $X$.
    One can check that this is a Boolean algebra.

    We make this a contravariant functor by taking any continuous map 
    $f : X \to Y$
    to an induced map $f^{\diamond}:= f^{-1}(\star):\BB_Y \to \BB_X$.
    One can check that this is a well-defined functor.

    Lastly we prove the equivalence of categories by giving 
    natural transformations $S(\BB_{\star}) \to \id{\star}$ in
    the category of topological spaces
    and $\id{\star} \to \BB_{S(\star)}$ in the category of Stone spaces.
    %?incomplete
\end{proof}

%?This end this section by using this general version of Stone spaces 
%?to provide an alternative way of proving the compactness theorem.

\begin{dfn}[A Boolean algebra on $F(\Si,n)$]
    Let $T$ be a $\Si$-theory.
    We quotient out $F(\Si,n)$ by the equivalence relation
    \[  
        \phi \sim \psi \quad := 
        \quad \phi \text{ and } \psi \text{ equivalent modulo } T
        := T \model{\Si} \forall v, (\phi \IFF \psi)
    \]
    Call the projection into the quotient $\pi$
    and the quotient $F(\Si,n) / T$.
    We make $F(\Si,n) / T$ into a Boolean algebra by 
    interpreting $0$ as $\pi(\bot)$, $1$ as $\pi(\top)$, 
    $\pi(\phi) \cap \pi(\psi)$ as $\pi(\phi \AND \psi)$,
    $\pi(\phi) \cup \pi(\psi)$ as $\pi(\phi \OR \psi)$,
    $\NEG \pi(\phi)$ as $\pi(\NOT \phi)$ and 
    $\pi(\phi) \leq \pi(\psi)$ as 
    \[\set{(\pi(\phi),\pi(\psi)) \st T \model{\Si} \forall v, (\phi \to \psi)}\]
    One can verify that these are well-defined and satisfy the axioms of 
    a Boolean algebra.
    Notice we need $T$ (potentially chosen to be the empty set) 
    to make $\to$ look like $\leq$ and that we had to quotient modulo $T$ 
    to make $\leq$ satisfy antisymmetry. 
    Antisymmetry in this context looks very much like 
    `propositional extensionality'.
    Thus it makes sense to consider the Stone space of this Boolean algebra 
    $S(F(\Si,n) / T)$.
\end{dfn}

\begin{nttn}
    To avoid confusion, 
    we denote the clopen sets in the Stone space of a theory $S_n(T)$ as
    \[[\phi]_T = \set{p \in S_n(T) \st \phi \in p}\]
    We look for a way to assosiate $[\phi]_T$ with $[\pi(\phi)]$,
    where the second is the clopen subset of $S(F(\Si,n) / T)$.
\end{nttn}

\begin{lem}
    \link{technical_lem_max_subs_pullback}
    If $p \subs F(\Si,n)$ is a maximal subset 
    ($\forall \phi \in F(\Si,n), \phi \in p$ or $\NOT \phi \in p$) then 
    $\pi(\phi) \in \pi(p)$ in the quotient implies $\phi \in p$.
\end{lem}
\begin{proof}
    If $\pi(\phi) \in \pi(p)$ then there exists $\psi \in p$ such that 
    $\psi$ is equivalent to $\phi$ modulo $T$.
    By consistency with $T$ there exists a non-empty $\Si$-model $\MM$ of $T$
    and $b \in {\MM}^n$ 
    such that $\MM \model{\Si} p(b)$, 
    in particular $\MM \model{\Si} \psi(b)$.
    Equivalence modulo $T$ then gives us that $\MM \model{\Si} \phi(b)$.
    By maximality of $p$, 
    $\phi$ or $\NOT \phi$ is in $p$ but the latter would lead to 
    $\MM \nodel{\Si} \phi(b)$, 
    a contradiction.
\end{proof}

\begin{lem}
    Let $\FF \in S(F(\Si,n) / T)$.
    Then $\pi^{-1}(\FF)$ is consistent with $T$ if and only if 
    $T \subs \pi^{-1}(\FF)$. 
\end{lem}
\begin{proof}
    If $\pi^{-1}(\FF)$ is consistent with $T$ then as $\FF$ is an ultrafilter
    $\pi^{-1}$ is maximal and thus contains $T$.
    On the other hand suppose $T \subs \pi^{-1}(\FF)$.
    To show that $\pi^{-1}(\FF)$ is consistent it suffices that 
    \linkto{equiv_def_of_consistent_with_theory}{$\pi^{-1}(\FF(c) \cup T)$
        is consistent as a $\Si(c)$-theory}, 
    where $c$ is an $n$-tuple of constant symbols.
    This holds \linkto{inconsistent}{if and only if} 
    for any $\Si(c)$-sentence $\phi(c)$,
    \[\pi^{-1}(\FF)(c) \cup T \nodel{\Si(c)} \phi 
    \quad \text{ or } \quad
    \pi^{-1}(\FF)(c) \cup T \nodel{\Si(c)} \NOT \phi\]
    Let $\phi(c)$ be a $\Si(c)$-sentence and suppose 
    \[\pi^{-1}(\FF)(c) \cup T \model{\Si(c)} \phi 
    \quad \text{ and } \quad
    \pi^{-1}(\FF)(c) \cup T \model{\Si(c)} \NOT \phi\]
    As $T \subs \pi^{-1}(\FF)$ this implies 
    \[\pi^{-1}(\FF)(c) \model{\Si(c)} \phi 
    \quad \text{ and } \quad
    \pi^{-1}(\FF)(c) \model{\Si(c)} \NOT \phi\]
    Since $\FF$ is a proper filter, 
\end{proof}

\begin{prop}[The Stone space of a theory is compact]
    This theorem has problems!!!%?Wrong!!!
    \link{stone_of_theo_compact}
    The Stone space of a $\Si$-theory $T$ is homeomorphic 
    to the set of ultrafilters from $S(F(\Si,n) / T)$ 
    that have preimage consistent with $T$ with the subspace topology.
    In other words, if $\pi$ is the projection to the quotient then
    $S_n(T) \iso X$, where
    \[X := 
    \set{\FF \in S(F(\Si,n) / T) \st \pi^{-1}(\FF) \text{ consistent with } T}\]
    $X$ is a closed subspace of $S(F(\Si,n) / T)$ hence it is compact,
    giving us topological compactness for $S_n(T)$.
\end{prop}
\begin{proof}
    Warning: this proof uses $\pi$ to be three different things,
    the quotient $F(\Si,n) \to F(\Si,n) / T$, 
    the image map (of the quotient) $S_n(T) \to X$, 
    and the map of clopen sets (the image map of the image map) 
    $\PP(S_n(T)) \to \PP(X)$.
    The second will be a homeomorphism and the third will be a map between
    subsets of the topologies (in particular the clopen subsets).

    We show that sending $p \in S_n(T)$ to its image under the projection 
    to the quotient $\pi(p)$ is a homeomorphism.
    To show that it is well-defined it suffices to show that for any $p$, a 
    maximal $n$-type over $T$, $\pi(p)$ is an ultrafilter of $F(\Si,n) / T$
    with preimage consistent with $T$.
    Preimage being consistent with $T$ 
    follows from the definition of $n$-types over theories.
    To show that it is a proper filter:
    \begin{itemize}
        \item $\top \in p$ by consistency and maximality. 
            Hence $\pi(\top) \in \pi(p)$.
        \item If $\pi(\bot) \in \pi(p)$ then
            \linkto{technical_lem_max_subs_pullback}{$\bot \in p$} 
            which is a contradiction with consistency.
        \item If $\pi(\phi), \pi(\psi) \in \pi(p)$ then 
            \linkto{technical_lem_max_subs_pullback}{$\phi,\psi \in p$} and so
            $\phi \AND \psi \in p$ thus by definition of the Boolean algebra 
            $F(\Si,n) / T$, 
            \[\pi(\phi) \cap \pi(\psi) = \pi(\phi \AND \psi) \in \pi(p)\]
        \item If $\pi(\phi) \in \pi(p)$ and $\pi(\phi) \leq \pi(\psi)$ then 
            \linkto{technical_lem_max_subs_pullback}{$\phi \in p$} 
            and by definition of $\leq$,
            \[T \model{\Si} \forall v, (\phi \to \psi)\]
            Since $p$ is consistent with $T$ there exists a non-empty
            $\Si$-structure $\MM$ and $b \in {\MM}^n$ such that 
            $\MM \model{\Si} \phi(b) \to \psi(b)$ and $\MM \model{\Si} \phi(b)$.
            Hence $\MM \model{\Si} \psi(b)$ and by maximality of $p$ we have 
            $\psi \in p$ and $\pi(\psi) \in \pi(p)$.
    \end{itemize}
    The image $\pi(p)$ is an ultrafilter by the 
    \linkto{negation_is_in_ultrafilter}{equivalent definition}:
    if $\pi(\phi) \in F(\Si,n) / T$ then either 
    $\phi \in p$ or $\NEG \phi \in p$ by maximality of $p$, 
    hence $\pi(\phi) \in \pi(p)$ or $\pi(\NEG \phi) \in \pi(p)$.
    Thus we have $\pi$ is a map into \[
    \set{\FF \in S(F(\Si,n) / T) \st \pi^{-1}(\FF) \text{ consistent with } T}\]

    Injectivity: if $p,q \in S_n(T)$ and $\pi(p) = \pi(q)$ then 
    if $\phi \in p$, we have $\pi(\phi) \in \pi(p) = \pi(q)$ and by our 
    claim above $\phi \in q$.
    Surjectivity: let $\FF \in S(F(\Si,n) / T)$ 
    have its preimage consistent with $T$.
    Then its preimage is an $n$-type.
    If its preimage is a maximal $n$-type then we have surjectivity.
    Indeed since $\FF$ is an ultrafilter if $\phi \in F(\Si,n)$ then 
    $\pi(\phi) \in \FF$ or $\pi(\NOT \phi) = \NEG \pi(\phi) \in \FF$,
    hence $\phi \in \pi^{-1}(\FF)$ or $\NOT \phi \in \pi^{-1}(\FF)$.

    To show that the map is continuous in both directions,
    it suffices to show that images of clopen sets are clopen and preimages of 
    clopen sets are clopen,
    as each topology is generated by their clopen sets.
    For $\phi \in F(\Si,n)$ since 
    $\pi : S_n(T) \to X$ is a bijection we have that
    \[\pi([\phi]_T) = \set{\FF \in X \st \phi \in \pi^{-1}(\FF)} =
    \set{\FF \in F(\Si,n) / T \st \pi(\phi) \in \FF \text{ and } \FF \in X} = 
    [\pi(\phi)] \cap X\]
    and similarly $\pi^{-1}([\pi(\phi)] \cap X) = [\phi]_T$.
    Hence there is a correspondence between clopen sets.

    To show that $X$ is a closed subspace: 
    (the second equality is where the problem in this theorem lies)%?
    \begin{align*}
        X &= 
        \set{\FF \in S(F(\Si,n) / T) \st \pi^{-1}(\FF) 
            \text{ consistent with $T$}}\\
        &= \set{\FF \in S(F(\Si,n) / T) \st \forall \phi \in T, 
        \phi \in \pi^{-1}(\FF)} \quad \text{ as } \pi^{-1}(\FF) 
        \text{ is a maximal $n$-type}\\%? This is wrong!!!
        &= \bigcap_{\phi \in T}\set{\FF \in S(F(\Si,n) / T) 
        \st \pi(\phi) \in \FF}\\
        &= \bigcap_{\phi \in T}[\pi(\phi)]
    \end{align*}
    and an intersection of clopen sets is closed. 
    \linkto{bool_alg_to_stone_functor}{$S(F(\Si,n) / T)$} is compact and 
    closed in compact is compact so $X$ is compact.
    By the homeomorphism we have that $S_n(T)$ is compact.
\end{proof}

\begin{lem}[Topological consistency]
    \link{topological_consistency}
    Let $\FF \in S(F(\Si,n) / T)$ and $T$ be a $\Si$-theory.
    $\pi^{-1}(\FF)$ is consistent with $T$ if and only if 
    \[\pi^{-1}(\FF) \in \bigcap_{\phi \in \pi^{-1}(\FF)} [\phi]_T\]
    if and only if 
    \[\bigcap_{\phi \in \pi^{-1}(\FF)} [\phi]_T \text{ is non empty}\]
\end{lem}
\begin{proof}
    ($1. \implies 2. \implies 3.$)
        Suppose $\pi^{-1}(\FF)$ is consistent with $T$.
        Then $\pi^{-1}(\FF) \in S_n(T)$ thus for any $\phi \in \pi^{-1}(\FF)$, 
        $\pi^{-1}(\FF) \in [\phi]_T$. 
        Hence 
        \[\pi^{-1}(\FF) \in \bigcap_{\phi \in \pi^{-1}(\FF)} [\phi]_T\]
        and it is non-empty.

    ($3. \implies 1.$)
        Suppose 
        \[p \in \bigcap_{\phi \in \pi^{-1}(\FF)} [\phi]_T\]
        then $\forall \phi \in \pi^{-1}(\FF), \phi \in p$.
        As $\FF$ is an ultrafilter, for any $\phi \in p$, 
        \[\phi \notin \pi^{-1}(\FF) \implies \NOT \phi \in \pi^{-1}(\FF) 
        \implies \NOT \phi \in p \text{ a contradiction}\]
        Hence $p = \pi^{-1}(\FF)$.
        Hence $\pi^{-1}(\FF) \in S_n(T)$ and thus is consistent with $T$.
\end{proof}

\begin{prop}[Compactness theorem for types]
    \link{compactness_for_types_2}
    Let $\FF \in S(F(\Si,n)/T)$ and $T$ be a $\Si$-theory.
    Then $\pi^{-1}(\FF)$ is consistent with $T$ 
    if and only if $\pi^{-1}(\FF)$ if finitely consistent
    with $T$.
\end{prop}
\begin{proof}
    By definition $\pi^{-1}(\FF)$ is finitely consistent with $T$ if and only if 
    any finite subset of $\pi^{-1}(\FF)$ is consistent with $T$.
    Translating this to the topology, this is
    \linkto{topological_consistency}{if and only if} 
    for any finite subset $\De \subs \pi^{-1}(\FF)$,
    \[\bigcap_{\phi \in \De} [\phi]_T \text{ is non empty}\]
    By \linkto{stone_of_theo_compact}{topological compactness of $S_n(T)$} 
    this is if and only if 
    \[\bigcap_{\phi \in \pi^{-1}(\FF)} [\phi]_T \text{ is non empty}\]
    \linkto{topological_consistency}{Translating this back to model theory}
    this is if and only if $\pi^{-1}(\FF)$ is consistent with $T$.
\end{proof}

Notice that here we have by the fact that $\FF$ is an ultrafilter that
$\pi^{-1}(\FF)$ is maximal.
We can remove weaken this hypothesis via Zorn's lemma.

\begin{prop}[The Compactness Theorem]
    A $\Si$-theory is consistent if and only if it is finitely consistent.
\end{prop}
\begin{proof}
    Suppose $T$ is a finitely consistent $\Si$-theory.
    Then we can \linkto{make_max}{extend $T$} to $p$, 
    a maximal and finitely consistent 
    $\Si$-theory.
    Then the image of the quotient 
    $\pi(p)$ is an ultrafilter on $F(\Si,n) / T$,
    i.e. $\pi(p) \in S(F(\Si,n)/T)$ and
    \linkto{technical_lem_max_subs_pullback}{$\pi^{-1}\pi(p) = p$} 
    is finitely consistent.
    By \linkto{compactness_for_types_2}{compactness for types} we have that 
    $p = \pi^{-1}(\pi(p))$ is consistent.
    Hence $T \subs p$ is consistent.
\end{proof}
\begin{rmk}
    We can even extend this to consistency of arbitrary subsets of 
    $S(F(\Si,n) / T)$
    with respect to a theory via the 
    \linkto{equiv_def_of_consistent_with_theory}{equivalent definition} 
    of being consistent with a theory.
\end{rmk}
