\section{Morley Rank}

We begin by looking at $1$-types, 
as done in \linkto{infinite_infinite_classes1}{the example} 
with infinite infinite equivalence classes.

\begin{prop}[Classification of $1$-types on a model of $\ACF_p$]
    Let $\NN$ be an \linkto{om_sat_elem_ext_of_models}{
        $\om$-saturated extension of $\MM$}, 
    a $\Si_\RNG$-model of $\ACF_p$.
    \linkto{elems_of_stone_space_are_types_of_elements}{
        Any element of the Stone space $S_1(\eldiag(\Si_\RNG,\MM))$
        is of the form}
        $\subintp{\nothing,1}{\NN}{\tp}(a)$.
    There are two cases for what the types over $\MM$ can be:
    \begin{itemize}
        \item $a$ is in the image of $\MM$; 
            equivalently the type is an isolated point.
        \item $a$ is not in the image of $\MM$;
            equivalently its type is the unique non-isolated type.
    \end{itemize}
\end{prop}
\begin{proof}
    The first part is covered in 
    \linkto{type_of_an_element_from_the_model}{the lemma}.

    For the second part suppose $a$ is not in the image of $\MM$.
    It is not isolated by the first part.
    Note that for any $p \in \MM[x] \setminus {0}$, 
    $\io(p)(a) \ne 0$ because $\MM$ is algebraically closed. 
    (If $\io(p)(a) = 0$ then we can keep factoring roots of the polynomial
    from $\MM$ to find one that corresponds to $a$.)
    We show that any $b$ not in the image gives rise to the same type as $a$.

    Let $\phi \in \tp(a)$.
    As \linkto{ACF_has_quantifier_elimination}{
        $\ACF_p$ has quantifier elimination}
    we have \linkto{quant_elim_for_types}{equivalently}
    $\phi \in \qftp(a)$.
    Then \linkto{disjunctive_normal_form}{by the disjunctive normal form} 
    for formulas in $\Si_\RNG$ we have that 
    \[ 
        \NN \model{\Si_\RNG} \forall v, \phi(v) 
        \IFF \bigor{i \in I}{} 
        \brkt{\bigand{j \in J_{i0}}{} p_{ij}(v) = 0 \AND 
        \bigand{j \in J_{i1}}{} q_{ij}(v) \ne 0}
    \]
    Thus there exists $i \in I$ such that 
    \[ 
        \NN \model{\Si_\RNG} \forall v, \phi(v) \iff
        \bigand{j \in J_{i0}}{} p_{ij}(v) = 0 \AND 
        \bigand{j \in J_{i1}}{} q_{ij}(v) \ne 0
    \]
    If $J_{i0}$ contains a non-zero polynomial $p_{ij}$ then we would have a 
    contradiction by the above remark.
    Thus 
    \[ 
        \NN \model{\Si_\RNG} \forall v, \phi(v) \iff
        \bigand{j \in J_{i1}}{} q_{ij}(v) \ne 0
    \]
    Suppose for a contradiction
    that $\phi$ is not in $\tp(b)$, 
    in which case $\NN \nodel{\Si_\RNG} \phi(b)$
    and $\NN \model{\Si_\RNG} \bigor{j \in J_{i1}}{} q_{ij}(b) = 0$.
    Hence either $\NN \model{\Si_\RNG} \bot$, which is a contradiction,
    or $\NN \model{\Si_\RNG} q_{ij}(b) = 0$ for some $j \in J_{i1}$.
    By the above remark $q_{ij}$ must therefore be the zero polynomial,
    which also leads to a contradiction as 
    \[ 
        \NN \model{\Si_\RNG} q_{ij}(a) \ne 0
    \]
    Hence $\tp(a) = \tp(b)$.
\end{proof}


\subsection{Dimension}
In this section we will define Model-theoretic dimension
and see how it aligns with transcendence degree and other geometric notions.
\begin{dfn}[Algebraic, algebraic closure]
    Let $\MM$ be a $\Si$-structure and let $D$ be a subset of $\MM$
    (often \linkto{strongly_minimal}{strongly minimal}).
    Let $A$ be a subset of $D$, 
    $a \in \MM$ is algebraic over $A$ if $a$ belongs to a finite 
    $\Si(A)$-definable set .
    Define the algebraic closure of $A$ over $D$ to be
    \[\acl_{\Si,D}(A) := \set{a \in D \st a \text{ is algebraic over } A}\]
    We drop the subscripts $\Si$ and $D$ when it is sufficiently obvious.
\end{dfn}

\begin{lem}[Some definable sets]
    \link{some_definable_sets}
    Let $\MM$ be a $\Si$-structure and let $B,C \subs \MM$ be $\Si$-definable
    set (i.e. $\Si(\nothing)$-definable).
    Let $\phi(x)$ be a $\Si$-formula with $n$ free variables.
    For $b \in B^n$, let $\psi(x,b)$ be a $\Si(B)$-formula 
    with $m$ free variables ($\psi(x,y)$
    is a $\Si$-formula with $n+m$ free variables).

    Then the following sets are definable by a $\Si$-formula:
    \begin{itemize}
        \item The intersection of $B$ and $C$, the union of $B$ and $C$ and
            the complement of $B$.
        \item The set of $b \in B^n$ that satisfy $\phi(x)$:
            \[\set{b \in B^n \st \MM \model{\Si} \phi(b)}\]
        \item The elements $b \in \MM^n$ such that $\psi(x,b)$ defines a set of 
            at most $k$ elements.
        \item The elements of $\MM^n$ such that $\psi(x,b)$ defines a set of 
            at least $k$ elements.
        \item The elements of $\MM^n$ such that $\psi(x,b)$ defines a set of 
            cardinality $k$.
            \[\set{b \in \MM^n \st \abs{\psi(\MM,b)} = k}\]
            and even $\set{b \in B^n \st \abs{\psi(\MM,b)} = k}$ by 
            taking the intersection of two definable sets.
    \end{itemize}
    We will become lazier when dealing with definable sets as we gain an idea 
    of what should and should not be definable.
\end{lem}
\begin{proof}
    \begin{itemize}
        \item This is clear.
        \item 
        Since $B$ is $\Si$-definable we can take $\chi(x)$ as the 
        $\Si$-formula defining $B$
        and consider the $\Si$-formula 
        \[\phi(x_1,\dots,x_n) \AND \bigand{i = 1}{n} \chi(x_i)\]
        Clearly this defines $\set{b \in B^n \st \MM \model{\Si} \phi(b)}$.
        \item 
        To make $\set{b \in \MM^n \st \abs{\psi(\MM,b)} \leq k}$
        we take the $\Si$-formula $\chi(x)$:
        \[
            \chi(x) = \bigforall{i = 1}{k + 1} x_i,
            \bigand{i = 1}{k + 1} \psi(x_i,y) \to \bigor{i \ne j}{x_i = x_j}
        \]
        where potentially $x_i$ represents $m$ variables, which we can 
        quantify over as it is finite.
        \item 
        To make $\set{b \in \MM^n \st k \leq \abs{\psi(\MM,b)}}$
        we take the $\Si$-formula $\chi(x)$:
        \[
            \chi(x) = \bigexists{i = 1}{k} x_i, {x_i \ne x_j}
        \]
    \end{itemize}
\end{proof}

\begin{dfn}[Minimal, strongly minimal \cite{marker}]
    \link{strongly_minimal}
    Let $\MM$ be a $\Si$-structure.
    Let $D$ be an infinite $\Si(\MM)$-definable subset of $\MM^n$.
    $D$ is minimal in $\MM$ if any $\Si(\MM)$-definable subset of $D$
    if finite or cofinite.
    $D$ is strongly minimal if it is minimal in 
    $\NN$ for any elementary extension $\NN$ of $\MM$.
    A $\Si$-theory $T$ is strongly minimal if any $\Si$-model of $T$
    is strongly minimal 
    (note that any $\Si$-structure is definable by the formula $v = v$).
\end{dfn}

\begin{prop}[Algebraic closure is a \linkto{pregeometry_dfn}{pregeometry}]
    \link{acl_d_is_pregeometry}
    Let $\MM$ be a $\Si$-structure. 
    Let $D$ be a minimal subset of $\MM$.
    Then $(D,\acl_{\Si,D})$ is a pregeometry.
\end{prop}
\begin{proof}\cite{fandom0}
    The signature we work in will always be $\Si$ and the 
    strongly minimal subset will always be $D$ so we drop the subscript here.
    Preserves order: 
    \emph{if $A \subs B \subs D$ then $\acl(A) \subs \acl(B)$.}
    Let $a \in \acl(A)$. 
    Then there exists a finite $\Si(A)$-definable set containing $a$.
    Any $\Si(A)$-formula is naturally a $\Si(B)$-formula thus $a \in \acl(B)$.

    Idempotence: \emph{for any $A \subs D$, $\acl(A) = \acl(\acl(A))$.}
    \begin{forward}
        We first show that for any subset $A \subs D$, $A \subs \acl(A)$.
        Let $a \in A$ then $a = x$ is a $\Si(A)$-formula that is defines a 
        finite set. 
        Thus $a \in \acl(A)$.
        Directly we have the corollary $\acl(A) \subs \acl(\acl(A))$.
    \end{forward}

    \begin{backward}
        We show that $\acl(\acl(A)) \subs \acl(A)$.
        Let $a \in \acl(\acl(A))$.
        Then there exists $\phi(x,v) = \phi(x,v_0,\dots,v_n)$ a $\Si$-formula 
        and $b_0,\dots, b_n \in \acl(A)$ such that $\phi(x,b)$
        defines a finite subset of $\MM$ containing $a$.
        Let $k$ be the finite cardinality of $\phi(\MM,b)$.
        \linkto{some_definable_sets}{There exists a $\Si$-formula $\psi(v)$} 
        that defines the set 
        $\set{b \in B \st \abs{\phi(\MM,b)} \leq n}$
        \[
            \phi'(x,v) := \phi(x,v) \AND \psi(v)
        \]
        We have that $a \in \phi(\MM,b) = \phi'(\MM,b)$
        and for any $c \in \MM^n$, $\phi'(\MM,c)$ is finite.
    
        For each $b_i$ appearing in $b$
        there exists a $\Si(A)$-formula $\psi_i(v_i)$ such that 
        $b_i \in \psi_i(\MM)$ and this definable set is finite.
        Define the $\Si(A)$-formula
        \[
            \phi''(x) := \bigexists{i = 1}{n} v_i,
            \phi'(x,v_0,\dots,v_n) \AND \bigand{i = 1}{n} \psi_i(v_i)
        \]
        Then $a \in \phi''(\MM)$ by taking the $v_i$ to be $b_i$ and 
        \begin{align*}
            d \in \phi''(\MM) 
            &\implies \exists c \in \MM^n, \MM \model{\Si}\phi'(d,c) 
            \text{ and for each $i$,} \MM \model{\Si}\psi_i(c_i)\\
            &\implies \text{there exist for each $i$ } c_i \in \psi_i(\MM),
            \MM \model{\Si}\phi'(d,c) \\
            &\implies d \in 
            \bigcup_{i = 0}^n \bigcup_{c_i \in \psi_i(\MM)} \phi'(\MM,c)
        \end{align*}
        The last expression is a finite union of finite sets which is finite.
        Hence $\phi''(\MM)$ is finite and $a \in \acl(A)$
    \end{backward}
    
    Finite character: \emph{if $A \subs D$ and $a \in \acl(A)$ then 
    there exists a finite subset $F \subs A$ such that $a \in \acl(F)$.}
    Take the $\Si(A)$-formula defining the finite set containing $a$.
    Pick out the (finitely many) constant symbols from $A$, 
    forming a finite subset $F \subs A$.
    Then $a \in \acl(F)$.

    Exchange: \emph{if $A \subs D$ and $a,b \in D$ such that 
    $a \in \acl(A,b)$ (shorthand for $A,\set{a}$)
    then $a \in \acl(A)$ or $b \in \acl(A,a)$.}
    Since $a \in \acl(A,b)$ there exists a $\Si(A)$-formula $\phi(v,w)$ such 
    that $a \in \phi(\MM,b)$ and $\phi(\MM,b)$ is finite - 
    say it has cardinality $n$ 
    (if $b$ does not appear in the formula then we immediately have 
    $a \in \acl(A)$).
    \linkto{some_definable_sets}{There exists a $\Si(A)$-formula} $\psi(w)$
    defining the set
    \[\psi(\MM) = \set{b' \in D \st n = \abs{\phi(\MM,b')}}\]
    As $\psi(\MM) \subs D$ and $D$ is minimal, 
    $\psi(\MM)$ is finite or cofinite.
    If it is finite then $b \in \psi(\MM)$ and so 
    $b \in \acl(A) \linkto{acl_d_is_pregeometry}{\subs} \acl(A,a)$.

    If it is $\psi(\MM)$ then consider the $\Si(A)$-formula 
    $\phi(v,w) \AND \psi(w)$.
    For each $a' \in D$ let $X(a')$ be the subset of $D$ defined by 
    $\phi(a',w) \AND \psi(w)$.
    Consider $b \in X(a)$, and case on whether it is finite or cofinite.
    If it is finite then $b \in \acl(A,a)$ as $\phi(a,w) \AND \psi(w)$
    is a $\Si(A)$-formula defining a finite set.

    If $X(a)$ is cofinite then let $m = \abs{D \setminus X(a)} \in \N$.
    \linkto{some_definable_sets}{There exists a $\Si(A)$-formula} $\chi(v)$
    defining the set
    \[\chi(\MM) = \set{a' \in D \st m = \abs{D \setminus X(a')}}\]
    If $\chi(\MM)$ is finite then $a \in \chi(\MM)$ and so $a \in \acl(A)$.
    If $\chi(\MM)$ is confinite then there exist $n + 1$ distinct elements 
    $a_i \in \chi(\MM)$ since $D$ is infinite by definition.
    Take the (finite) intersection of the cofinite $X(a_i)$,
    producing a non-empty (infinite) set.
    Take 
    \[b' \in \bigcap_{i = 1}^{n+1} X(a_i) = 
    \bigcap_{i = 1}^{n+1} \phi(a_i,\MM) \cap \psi(\MM)\] 
    Then for each $i$, $\MM \model{\Si} \phi(a_i,b')$, 
    hence $n + 1 \leq \abs{\phi(\MM,b')}$.
    However $\MM \model{\Si} \psi(b')$ implies $n = \abs{\phi(\MM,b')}$,
    a contradiction.
\end{proof}

The definition of \linkto{dimension_dfn}{dimension} 
    for pregeometries thus carries through for 
    subsets of $D$.
\begin{dfn}
    Let $\MM$ be a $\Si$-structure and let $X \subs D \subs \M$,
    where $D$ is minimal.
    We write $\dim_{\Si,D}(X)$ to mean the 
    \linkto{dimension_dfn}{dimension} 
    of $X$ in the pregeometry $(D,\acl_{\Si,D})$.
    We call this the $\Si$-dimension of $X$ in $D$.
\end{dfn}

\begin{lem}[$\acl$ preserves dimension]
    \link{acl_preserves_dimension}
    Let $\MM$ be a $\Si$-structure and let $X \subs D \subs \M$,
    where $D$ is minimal. 
    Then $\dim_{\Si,D}(X) = \dim_{\Si,D}(\acl_{\Si,D}(X))$.
\end{lem}
\begin{proof}
    Let $S \subs X$ be a basis of $X$. 
    Then it is an independent subset of $X \subs \acl(X)$ such that 
    \[\acl(X) \subs \acl(S) 
    \quad \text{ and by \linkto{acl_d_is_pregeometry}{idempotence} } \quad
    \acl(\acl(X)) \subs \acl(X) \subs \acl(S)\]
    Hence $S$ is a basis for $\acl(X)$.
\end{proof}

\subsection{The theory of algebrically closed fields is strongly minimal}
\begin{dfn}[Strongly minimal theory]
    A $\Si$-theory $T$ is (strongly) minimal if any $\Si$-model 
    of $T$ is (strongly) minimal.
\end{dfn}

\begin{lem}[Disjunctive normal form of definable sets]
    \link{dnf_for_definable_sets}
    Let $K$ be an algebraically closed field.
    Any definable set in $K^n$ can be written in the form 
    \[\bigcup_{i \in S}\brkt{V_i \cap U_i}\]
    where $V_i$ is a variety and $U_i$ is the complement of a variety in $K^n$.
\end{lem}
\begin{proof}
    Let $X$ be a definable set:
    \[X = \set{a \in K^m \st K \model{\Si_\RNG} \phi(a,b)}\]
    where $b \in K^n$ and 
    $\phi$ is some $\Si_\RNG$-formula with $n+m$ free variables.
    Then by \linkto{ACF_has_quantifier_elimination}{quantifier elimination in 
    $\ACF$}
    we have a quantifier free $\Si_\RNG$-formula $\psi$ such that 
    \[X = \set{a \in K^m \st K \model{\Si_\RNG} \psi(a,b)}\]
    We can find the `disjunctive normal form' of $\psi$ as 
    \linkto{disjunctive_normal_form}{it is quantifier free}.
    Hence for $a \in K^m$
    \begin{align*}
        &a \in X\\
        &\iff \bigor{i \in I}{} 
        \brkt{\bigand{j \in J_{i0}}{} p_{ij}(a,b) = 0 \AND 
        \bigand{j \in J_{i1}}{} q_{ij}(a,b) \ne 0}\\
        &\iff a \in \bigcup_{i \in I}
        \brkt{\bigcap_{j \in J_{i0}} \V_K(p_{ij}(x,b)) \cap 
        \bigcap_{j \in J_{i1}} K^m \setminus \V_K(q_{ij}(x,b))}\\
    \end{align*}
\end{proof}

\begin{lem}[Non-trivial vanishings are finite]
    \link{vanishing_finite_or_cofinite}
    If $K$ is a field and $S \subs K[x]$
    then $\V_K(S)$ is finite or $S = \set{0}$.
    In particular $\V_K(S)$ is either finite or cofinite in $K$.
\end{lem}
\begin{proof}
    If $\V(f)$ is finite we are done.
    If $\V(f)$ is infinite
    then each $f \in S$ has infinitely many distinct roots
    so $f = 0$ by the division algorithm.
    
    In particular, if $S = \set{0}$ then $\V(S)$ is $K$ and it is cofinite.
\end{proof}

\begin{prop}
    \link{ACF_strong_min}
    $\ACF$ is strongly minimal.
\end{prop}
\begin{proof}
    Let $K$ be an algebraically closed field.
    
    Let $D \subs K$ be definable.
    Any elementary extension of $K$ is also algebraically closed so 
    without loss of generality we only need to show minimality rather than 
    strong minimality.
    Then \linkto{dnf_for_definable_sets}{there exist 
    $p_{ij}(x,b),q_{ij}(x,b) \in K[x]$} 
    (note that the polynomials are in only one variable) such that 
        \[D = \bigcup_{i \in I}
        \brkt{\bigcap_{j \in J_{i0}} \V_K(p_{ij}(x,b)) \cap 
        \bigcap_{j \in J_{i1}} K \setminus \V_K(q_{ij}(x,b))}\]
    Which is a finite union and intersection of 
    \linkto{vanishing_finite_or_cofinite}{finite and cofinite sets},
    which is finite or cofinite.
    Hence $D$ is finite or cofinite and $\ACF$ is strongly minimal.
\end{proof}

\subsection{Dimension and transcendence degree}

Our first aim is to show that our definition of algebraic closure 
aligns with the usual notion of algebraic closure of fields.
\begin{lem}
    \link{adjoining_elements_is_adding_constant_symbols}
    If $K$ is a field and $A$ is a subset of an extension field then
    $\term{\Si(K(A))} = \term{\Si(K,A)}$ and
    $\form{\Si(K(A))} = \form{\Si(K,A)}$.
    ($K(A)$ is the minimal subfield of the extension containing 
    $A$ and the image of $K$)
\end{lem}
\begin{proof}
    %?
\end{proof}

\begin{lem}[Formulas defining finite sets and polynomials]
    \link{formulas_defining_finite_sets_give_poly}
    Suppose $K$ is an algebraically closed field, $S$ is a set of constant
    symbols and 
    $\phi$ is a $\Si_\RNG(K,S)$-formula with exactly $1$ free variable.
    If $\phi$ defines a finite set containing $a \in K(S)$ 
    then there exists a non-zero polynomial $p \in K(S)[x]$ such that 
    $p(a) = 0$.
\end{lem}
\begin{proof}
    First take the \linkto{disjunctive_normal_form}{
        disjunctive normal form of $\phi$}
    \[K(S) \model{\Si(K,S)} \forall v, \phi \IFF \bigor{i \in I}{} 
        \brkt{\bigand{j \in J_{i0}}{} p_{ij}(v) = 0 \AND 
        \bigand{j \in J_{i1}}{} q_{ij}(v) \ne 0}\]
    where $\Si(K,S)$-terms $p_{ij},q_{ij}$ are naturally 
    \linkto{adjoining_elements_is_adding_constant_symbols}{$\Si(K(S))$-terms} 
    and thus polynomials in $K(S)[x]$.
    We see that there is some $i \in I$ such that 
    \[K(S) \model{\Si_\RNG(K,S)} 
        \bigand{j \in J_{i0}}{} p_{ij}(a) = 0 \AND 
        \bigand{j \in J_{i1}}{} q_{ij}(a) \ne 0
    \]
    The set defined by $\bigand{j \in J_{i0}}{} p_{ij}(v) = 0$ in $K(S)$
    is $\V_{K(S)}(\set{p_{ij} \st j \in J_{i0}})$,
    hence is \linkto{vanishing_finite_or_cofinite}{either finite
    (there exists a non-zero polynomial) 
    or all of $K(S)$ (all polynomials are zero)}.
    In the first case we obtain a non-zero polynomial $p \in K(S)[x]$ 
    such that $p(a) = 0$ and we are done.

    Assume for a contradiction the second case holds.
    For each $j \in J_{i1}$ 
    consider the set defined by $q_{ij}(v) \ne 0$ in $K(S)$,
    which is \linkto{vanishing_finite_or_cofinite}{cofinite or empty}.
    If it is empty then it is not satisfied by $a$ which is a contradiction.
    Hence 
    \[
        \bigand{j \in J_{i0}}{} p_{ij}(v) = 0 \AND 
        \bigand{j \in J_{i1}}{} q_{ij}(v) \ne 0
    \]
    defines a cofinite subset of $\phi(K(S))$, and so $\phi(K(S))$ is 
    \linkto{card_of_alg_closed_fields}{infinite}, a contradiction.
\end{proof}

%? Do we use this?
\begin{prop}[Algebraic closure in $\ACF$ is an field theoretic 
    algebraic closure]
    Let $K \to M$ be a field extension with $M$ algebraically closed, 
    then as \linkto{ACF_strong_min}{$\ACF$ is strongly minimal} 
    we have that $M$ is strongly minimal.
    Consider a subset $A \subs M$.
    Then $\acl_{\Si(K),M}(A)$ is an algebraic closure of $K(A)$.
\end{prop}
\begin{proof}
    We write $\acl$ instead of $\acl_{\Si(K),M}$.
    We must show that $\acl(A)$ is a field that contains $K(A)$, 
    is algebraically closed and is an algebraic extension of 
    $K(A)$.

    To show that it is a field we only show closure for addition and leave
    the rest as an exercise. 
    Let $a, b \in \acl(A)$.
    Then there exist $\Si(K,A)$-formulas $\phi_a,\phi_b$ that define 
    finite subsets of $M$ containing $a$ and $b$ respectively.
    Then their sum is in the finite set defined by the formula with one free
    variable $z$:
    \[
        \exists x, \exists y, \phi_a(x) \AND \phi_b(y) \AND z = x + y
    \]

    It contains $K(A)$ if and only if it contains $A$ and the image of $K$.
    It contians $K$ since for any element $k \in K$ we can take the 
    $\Si(K,A)$-formula $x = k$. 
    Similarly for $A$.

    To show that it is algebraically closed let 
    $p \in \acl(A)[x]$ be a polynomial.
    Then we write $p$ out in terms of its coefficients $a_i \in \acl(A)$
    \[p = \sum_{i = 1}^m a_i x^i\]
    Each coefficient is in a finite subset of $M$
    defined by a $\Si(K,A)$-formula $\phi_i$.
    The formula with free variable $x$
    \[
        \bigexists{i = 1}{m} v_i, \bigand{i = 1}{m} \phi_i(v_i) \AND 
        \sum_{i = 1}^m v_i x^i
    \]
    defines the set of roots of $p$ in $M$.
    Since $M$ is algebraically closed this is all the roots and since
    $p$ only has finitely many roots it is finite.
    Hence all the roots of $p$ are in $\acl(A)$.

    To show that it is an algebraic extension of $K(A)$ we take $a \in \acl(A)$
    and obtain a formula defining a finite subset of $M$ containing $a$.
    Thus 
    \linkto{formulas_defining_finite_sets_give_poly}{there exists a polynomial}
    $p \in K(A)[x]$ with $a$ as a root, and so $a$ is algebraic over $K(A)$.
\end{proof}

%?Now we look at the connection between dimension and transcendence degree.

\begin{prop}[Transcendence degree is dimension]
    Let $K \to L$ be a field extension.
    Suppose $S \subs L$ and 
    $M$ is an algebraically closed field extension of $L$.
    We consider the pregeometry $(M,\acl_{\Si_\RNG(K),M})$.
    \begin{enumerate}
        \item $S$ is algebraically independent over $K$ if and only if 
            $S$ is independent in the pregeometry.
        \item $K(S) \to L$ is an algebraic extension if and only if $S$ 
            spans $L$ in the pregeometry.
        \item $S$ is a transcendence basis for the extension $K \to L$ 
            if and only if 
            $S$ is a basis for $L$ in the pregeometry.
        \item Transcendence degree is the same thing as dimension:
            \[\tdeg(K \to K(S)) = \dim_{\Si(K),M}(L)\]
    \end{enumerate}
\end{prop}
\begin{proof}~
    We will write $\acl$ to mean $\acl_{\Si_\RNG(K),M}$.
    \begin{enumerate}
        \item \begin{forward}
            Let $a \in S$ and suppose for a contradiction that 
            $a \in \acl(S \setminus \set{a})$.
            Then there exists a $\Si_\RNG(K,S \setminus \set{a})$-formula $\phi$
            defining a finite subset of $\M$ containing $a$.
            Then \linkto{formulas_defining_finite_sets_give_poly}{
                there exists a non-zero polynomial} 
            $p \in K(S \setminus \set{a})[x]$ such that 
            $p(a) = 0$.
            This contradicts algebraic independence of $S$.
        \end{forward}

        \begin{backward}
            Suppose $p \in K[x_0,\dots,x_n]$ with distinct $s_0,\dots,s_n \in S$
            such that $p(s_0,\dots,s_n) = 0$.
            Then we have a polynomial in one variable
            \[p(x_0,s_1,\dots,s_n) \in K(S \setminus \set{s_0})[x_0]\]
            with $s_0$ as a root.
            Call $\phi(x_0)$ 
            the \linkto{adjoining_elements_is_adding_constant_symbols}{
                $\Si_\RNG(K,S \setminus \set{s_0})$}-formula 
            `$p(x_0,s_1,\dots,s_n) = 0$'.
            Since 
            \linkto{vanishing_finite_or_cofinite}{
                non-zero polynomials have finitely many roots} $\phi(M)$
            is finite or $p = 0$; since $s_0$ is a root, 
            \[M \model{\Si_\RNG (K,S \setminus \set{s_0})} \phi(s_0)\]
            and so if it is finite then $s_0 \in \acl(S \setminus \set{s_0})$,
            contradicting pregeometrical independence.
            Hence $p = 0$ and so $S$ is algebraically independent.
        \end{backward}
        \item \begin{forward}
            By \linkto{acl_d_is_pregeometry}{idempotence} of $\acl$, 
            for $S$ to be spanning in the pregeometry 
            it suffices to show that $L \subs \acl(S)$ 
            (take $\acl$ of both sides).
            Let $a \in L$.
            Since $K(S) \to L$ is an algebraic extension 
            there exists a non-zero polynomial 
            $p \in K(S)[x]$ such that $p(a) = 0$.
            The \linkto{adjoining_elements_is_adding_constant_symbols}{
                $\Si_\RNG(K,S)$}-formula `$p = 0$',
            defines a finite set containing $a$.
        \end{forward}
        
        \begin{backward}
            Let $a \in L$.
            We want to show that $a$ is algebraic over $K(S)$
            Since $S$ spans $L$
            \[a \in L \subs \acl(L) \subs \acl(S)\]
            and so there exists a $\Si(K,S)$-formula defining a finite set 
            containing $a$.
            \linkto{formulas_defining_finite_sets_give_poly}{Hence there exists
            a non-zero polynomial} $p \in K(S)[x]$ with $a$ as a root.
            Hence the extension is algebraic.
        \end{backward}

        \item
            $S$ is a transcendence basis of the extension $K \to L$ 
            \linkto{transcendence_bases_algebraic_extensions}{if and only if}
            $K(S) \to L$ is algebraic and $S$ is 
            algebraically independent over $K$.
            By the last two parts this is if and only if $S$ is independent 
            and spanning $L$ in the pregeometry, 
            which is if and only if $S$ is pregeometrical basis for $L$.
        
        \item This is clear.
    \end{enumerate}
\end{proof}

\subsection{Morley degree}
\begin{prop}[Strong minimality in terms of Morley rank and degree]
    Let $X$ be a definable subset of $\MM$, a $\Si$-structure.
    Then $X$ is strongly minimal if and only if $\MR{}{X} = \MD{X} = 1$.
\end{prop}
\begin{proof}
    \begin{forward}
        Suppose $X$ is strongly minimal.
        Then $X$ is infinite 
        \linkto{basic_facts_morley_rank_of_dfnbl_set}{hence} $1 \leq \MR{}{X}$.
        Let $\M$ be an 
        \linkto{om_sat_elem_ext_of_models}{$\om$-saturated extension of $\MM$}.
        If $2 \leq \MR{}{X}$ then there would be infinitely many 
        disjoint $\Si(\M)$ definable subsets of $X$ of Morley rank $1$
        (hence they are infinite).
        By strong minimality the only 
        $\Si(\M)$-definable subsets of $X$ are finite or cofinite.
        Thus these subsets must all be cofinite
        and so any two will intersect, a contradiction.
        Thus $1 = \MR{}{X}$.

        Since $\MR{}{X} = 1 \in \ord$ we have that $\MD{X} \in \N_{>0}$.
        Again if we have two disjoint 
        $\Si(\M)$-definable subsets of $X$ of Morley rank $1$ 
        we have a contradiction, hence $\MD{X} \leq 1$ and so $\MD{X} = 1$.
    \end{forward}

    \begin{backward}
        Suppose $X$ has Morley rank and degree $1$.
        Let $\NN$ be an elementary extension of $\MM$. 
        Let $A \subs X$ be a $\Si(\NN)$-definable subset of $X$;
        its complement is also $\Si(\NN)$-definable.
        Suppose both are infinite then they both have Morley rank greater 
        than or equal to $1$ and are disjoint,
        thus $2 \leq \MD{}{X} = 1$, which is a contradiction.
        Hence one is finite and the other cofinite.
    \end{backward}
\end{proof}

\subsection{Krull dimension and Morley rank}
We will show in this section that Krull dimension is the same thing as Morley
rank.

\begin{prop}[Morley rank of types and dimension]
    Suppose $\M$ is strongly minimal. %?Theorem hypotheses might be wrong
    Let $A \subs \M$ such that $\abs{\acl_{\Si(A),\M}(A)} < \abs{\M}$ 
    and let $a \in \M^k$. Then 
    \[\MR{}{\subintp{A}{\M}{\tp}(a)} = \dim_{\Si(A),\M}(\{a_1,\dots,a_k\})\]
\end{prop}
\begin{proof}
    We work in the pregeometry $(\M,\acl_{\Si(A),\M})$ and write 
    $\dim$ for $\dim_{\Si(A),\M}$ and $\acl$ for $\acl_{\Si(A),\M}$.
    We also write $\MR{}{a}$ for $\MR{}{\subintp{A}{\M}{\tp}(a)}$
    and $a$ for $\{a_1,\dots,a_k\}$.

    Let us first show that without loss of generality 
    $\{a_1,\dots,a_k\}$ is independent in the pregeometry.
    Concretely, we prove by induction on $k$ that there exists an independent
    subset $s \subs \set{a_1,\dots,a_k}$ such that 
    \[\MR{}{s} = \MR{}{a} \quad \text{ and } \quad \dim(s) = \dim(a)\]
    If $k = 0$ then it is okay as the empty set is trivially independent.
    If $0 < k$ then we case on if $\set{a_1,\dots,a_k}$ is independent
    or not.
    If it is then we are done. 
    Otherwise remove a dependent element $a_i$.
    By the theorem on \linkto{morley_rank_of_extended_types}{Morley 
        rank of extended types} we have that 
    \[\MR{}{a \setminus \set{a_i}} = \MR{}{a}\]
    Since \linkto{acl_preserves_dimension}{$\acl$ preserves dimension}
    and $a_i$ is dependent
    \[
        \dim(a \setminus \set{a_i}) = \dim(\acl(a \setminus \set{a_i}))
        = \dim(\acl(a)) = \dim(a)
    \]
    Hence by induction there is a subset $s \subs a \setminus \set{a_i} \subs a$
    such that 
    \[\MR{}{s} = \MR{}{a \setminus \set{a_i}} = \MR{}{a}
    \text{ and } \dim(s) = \dim(a \setminus \set{a_i}) = \dim(a)\]
    Hence $a$ can be replaced by an independent subset.

    Now we show that for independent $a_1,\dots,a_k$, $\MR{}{a} = k$.
    This will complete the proof since independent sets 
    are bases for themselves, which implies $\dim(a) = k = \MR{}{a}$.
    We prove this by induction on $k$.
    If $k = 0$ then $\subintp{A}{\M}{\tp}(a)$ 
    is just sentences satisfied by $\M$, 
    which \linkto{definable_set}{by convension} each define the set 
    $\set{\nothing}$. 
    Hence everything in $\tp(a)$ has Morley rank $0$ and so $\MR{}{a} = 0$.

    Suppose 
    
    %?Incomplete
\end{proof}

\begin{prop}[Morley rank is Krull dimension for algebraically closed fields]
    Let $K$ be an algebraically closed field and $X \subs K^n$ a variety.
    Then the Morley rank of $X$ is equal to the 
    \linkto{dfn_krull_dimension}{Krull dimension} of $X$.
\end{prop}
\begin{proof}
    
\end{proof}