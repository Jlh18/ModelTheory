\documentclass{book}
\usepackage[left=1in,right=1in,tmargin = 25mm,bmargin = 25mm]{geometry}
\usepackage[left=1in,right=1in]{geometry}
\usepackage{subfiles}
\usepackage{amsmath, amssymb, stmaryrd, verbatim} % math symbols
\usepackage{amsthm} % thm environment
\usepackage{mdframed} % Customizable Boxes
\usepackage{hyperref,nameref,cleveref,enumitem} % for references, hyperlinks
\usepackage[dvipsnames]{xcolor} % Fancy Colours
\usepackage{mathrsfs} % Fancy font
\usepackage{tikz, tikz-cd, float} % Commutative Diagrams
\usepackage{perpage}
\usepackage{parskip} % So that paragraphs look nice
\usepackage{ifthen,xargs} % For defining better commands
\usepackage{anyfontsize}
\usepackage[T1]{fontenc}
\usepackage[utf8]{inputenc}
\usepackage{tgpagella}

% Bibliography 
\usepackage{url}
\usepackage{biblatex}

\addbibresource{refs.bib}

% % Misc
\newcommand{\brkt}[1]{\left(#1\right)}
\newcommand{\sqbrkt}[1]{\left[#1\right]}
\newcommand{\dash}{\text{-}}

% % Logic
\renewcommand{\implies}{\Rightarrow}
\renewcommand{\iff}{\Leftrightarrow}
\newcommand{\IFF}{\leftrightarrow}
\newcommand{\limplies}{\Leftarrow}
\newcommand{\NOT}{\neg\,}
\newcommand{\AND}{\, \land \,}
\newcommand{\OR}{\, \lor \,}
\newenvironment{forward}{($\implies$)}{}
\newenvironment{backward}{($\limplies$)}{}
% General way of making larger symbols with limits above and below
\makeatletter
\DeclareRobustCommand\bigop[1]{%
  \mathop{\vphantom{\sum}\mathpalette\bigop@{#1}}\slimits@
}
\newcommand{\bigop@}[2]{%
  \vcenter{%
    \sbox\z@{$#1\sum$}%
    \hbox{\resizebox{
      \ifx#1\displaystyle.7\fi\dimexpr\ht\z@+\dp\z@}{!}{$\m@th#2$}}% symbol size
  }%
}
\makeatother
\newcommand{\bigforall}[2]{\DOTSB\bigop{\forall}_{#1}^{#2}}
\newcommand{\bigexists}[2]{\DOTSB\bigop{\exists}_{#1}^{#2}}
\newcommand{\bigand}[2]{\DOTSB\bigop{\mbox{\Large$\land$}}_{#1}^{#2}}
\newcommand{\bigor}[2]{\DOTSB\bigop{\mbox{\Large$\lor$}}_{#1}^{#2}}

% % Sets
\DeclareMathOperator{\supp}{supp}
\newcommand{\set}[1]{\left\{#1\right\}}
\newcommand{\st}{\,|\,}
\newcommand{\minus}{\setminus}
\newcommand{\subs}{\subseteq}
\newcommand{\ssubs}{\subsetneq}
\DeclareMathOperator{\im}{Im}
\newcommand{\nothing}{\varnothing}
\newcommand\res[2]{{% we make the whole thing an ordinary symbol
  \left.\kern-\nulldelimiterspace 
  % automatically resize the bar with \right
  #1 % the function
  \vphantom{\big|} 
  % pretend it's a little taller at normal size
  \right|_{#2} % this is the delimiter
  }}

% % Greek 
\newcommand{\al}{\alpha}
\newcommand{\be}{\beta}
\newcommand{\ga}{\gamma}
\newcommand{\de}{\delta}
\newcommand{\ep}{\varepsilon}
\newcommand{\io}{\iota}
\newcommand{\ka}{\kappa}
\newcommand{\la}{\lambda}
\newcommand{\om}{\omega}
\newcommand{\si}{\sigma}

\newcommand{\Ga}{\Gamma}
\newcommand{\De}{\Delta}
\newcommand{\Th}{\Theta}
\newcommand{\La}{\Lambda}
\newcommand{\Si}{\Sigma}
\newcommand{\Om}{\Omega}

% % Mathbb
\newcommand{\N}{\mathbb{N}}
\newcommand{\M}{\mathbb{M}}
\newcommand{\Z}{\mathbb{Z}}
\newcommand{\Q}{\mathbb{Q}}
\newcommand{\R}{\mathbb{R}}
\newcommand{\C}{\mathbb{C}}
\newcommand{\F}{\mathbb{F}}
\newcommand{\V}{\mathbb{V}}
\newcommand{\U}{\mathbb{U}}

% % Mathcal
\renewcommand{\AA}{\mathcal{A}}
\newcommand{\BB}{\mathcal{B}}
\newcommand{\CC}{\mathcal{C}}
\newcommand{\DD}{\mathcal{D}}
\newcommand{\EE}{\mathcal{E}}
\newcommand{\FF}{\mathcal{F}}
\newcommand{\GG}{\mathcal{G}}
\newcommand{\HH}{\mathcal{H}}
\newcommand{\II}{\mathcal{I}}
\newcommand{\JJ}{\mathcal{J}}
\newcommand{\KK}{\mathcal{K}}
\newcommand{\LL}{\mathcal{L}}
\newcommand{\MM}{\mathcal{M}}
\newcommand{\NN}{\mathcal{N}}
\newcommand{\OO}{\mathcal{O}}
\newcommand{\PP}{\mathcal{P}}
\newcommand{\QQ}{\mathcal{Q}}
\newcommand{\RR}{\mathcal{R}}
\renewcommand{\SS}{\mathcal{S}}
\newcommand{\TT}{\mathcal{T}}
\newcommand{\UU}{\mathcal{U}}
\newcommand{\VV}{\mathcal{V}}
\newcommand{\WW}{\mathcal{W}}
\newcommand{\XX}{\mathcal{X}}
\newcommand{\YY}{\mathcal{Y}}
\newcommand{\ZZ}{\mathcal{Z}}

% % Mathfrak
\newcommand{\f}[1]{\mathfrak{#1}}

% % Mathrsfs
\newcommand{\s}[1]{\mathscr{#1}}

% % Category Theory
\newcommand{\obj}[1]{\mathrm{Obj}\left(#1\right)}
\newcommand{\Hom}[3]{\mathrm{Hom}_{#3}(#1, #2)\,}
\newcommand{\mor}[3]{\mathrm{Mor}_{#3}(#1, #2)\,}
\newcommand{\End}[2]{\mathrm{End}_{#2}#1\,}
\newcommand{\aut}[2]{\mathrm{Aut}_{#2}#1\,}
\newcommand{\CAT}{\mathbf{Cat}}
\newcommand{\SET}{\mathbf{Set}}
\newcommand{\TOP}{\mathbf{Top}}
%\newcommand{\GRP}{\mathbf{Grp}}
\newcommand{\RING}{\mathbf{Ring}}
\newcommand{\MOD}[1][R]{#1\text{-}\mathbf{Mod}}
\newcommand{\VEC}[1][K]{#1\text{-}\mathbf{Vec}}
\newcommand{\ALG}[1][R]{#1\text{-}\mathbf{Alg}}
\newcommand{\PSH}[1]{\mathbf{PSh}\brkt{#1}}
\newcommand{\map}[2]{ \yrightarrow[#2][2.5pt]{#1}[-1pt] }
\newcommand{\op}{^{op}}
\newcommand{\darrow}{\downarrow}
\newcommand{\LIM}[2]{\varprojlim_{#2}#1}
\newcommand{\COLIM}[2]{\varinjlim_{#2}#1}

% % Algebra
\newcommand{\iso}{\cong}
\newcommand{\nsub}{\trianglelefteq}
\newcommand{\id}[1]{\mathrm{id}_{#1}}
\newcommand{\inv}{^{-1}}
\DeclareMathOperator{\dom}{dom}
\DeclareMathOperator{\codom}{codom}
\DeclareMathOperator{\coker}{Coker}
\DeclareMathOperator{\spec}{Spec}

% % Analysis
\newcommand{\abs}[1]{\left\vert #1 \right\vert}
\newcommand{\norm}[1]{\left\Vert #1 \right\Vert}
\renewcommand{\bar}[1]{\overline{#1}}
\newcommand{\<}{\langle}
\renewcommand{\>}{\rangle}
\renewcommand{\hat}[1]{\widehat{#1}}
\renewcommand{\check}[1]{\widecheck{#1}}

% % Galois
\newcommand{\Gal}[2]{\mathrm{Gal}_{#1}(#2)}
\DeclareMathOperator{\Orb}{Orb}
\DeclareMathOperator{\Stab}{Stab}
\newcommand{\emb}[3]{\mathrm{Emb}_{#1}(#2, #3)}
\newcommand{\Char}[1]{\mathrm{Char}#1}

% % Model Theory
\newcommand{\intp}[2]{
    \star_{\text{\scalebox{0.7}{$#1$}}}^{
    \text{\scalebox{0.7}{$#2$}}}}
\newcommand{\subintp}[3]{
    {#3}_{\text{\scalebox{0.7}{$#1$}}}^{
    \text{\scalebox{0.7}{$#2$}}}}
\newcommand{\modintp}[2]{#2^\text{\scalebox{0.7}{$#1$}}}
\newcommand{\mmintp}[1]{\modintp{\MM}{#1}}
\newcommand{\nnintp}[1]{\modintp{\NN}{#1}}
\newcommand{\const}[1]{{#1}_\mathrm{con}}
\newcommand{\func}[1]{{#1}_\mathrm{fun}}
\newcommand{\rel}[1]{{#1}_\mathrm{rel}}
\newcommand{\term}[1]{{#1}_\mathrm{ter}}
\newcommand{\struc}[1]{{#1}_\mathrm{str}}
\newcommand{\form}[1]{{#1}_\mathrm{for}}
\newcommand{\var}[1]{{#1}_\mathrm{var}}
\newcommand{\theory}[1]{{#1}_\mathrm{the}}
\newcommand{\carrier}[1]{{#1}_\mathrm{car}}
\newcommand{\model}[1]{\vDash_{#1}}
\newcommand{\nodel}[1]{\nvDash_{#1}}
\newcommand{\modelsi}{\model{\Si}}
\newcommand{\eldiag}[2]{\mathrm{ElDiag}(#1,#2)}
\newcommand{\atdiag}[2]{\mathrm{AtDiag}(#1,#2)}
\newcommand{\Theory}{\mathrm{Th}}
\newcommand{\unisen}[1]{{#1}_\mathrm{uni}}
\newcommand{\lift}[2]{\uparrow_{#1}^{#2}}
\newcommand{\fall}[2]{\downarrow_{#1}^{#2}}
\newcommand{\Mod}[1]{\M \mathbf{od}(#1)}
\DeclareMathOperator{\GRP}{GRP}
\newcommand{\RNG}{\mathrm{RNG}}
\newcommand{\ER}{\mathrm{ER}}
\DeclareMathOperator{\FLD}{FLD}
\DeclareMathOperator{\ID}{ID}
\DeclareMathOperator{\ZFC}{ZFC}
\DeclareMathOperator{\ACF}{ACF}
\newcommand{\BLN}{\mathrm{BLN}}
\newcommand{\PO}{\mathrm{PO}}
\DeclareMathOperator{\tp}{tp}
\DeclareMathOperator{\qftp}{qftp}
\DeclareMathOperator{\qf}{qf}
\DeclareMathOperator{\eqzero}{eqzero}
\newcommand{\MR}[2]{\mathrm{MR}^{#1}(#2)}
\DeclareMathOperator{\acl}{\mathrm{acl}}
\DeclareMathOperator{\cl}{\mathrm{cl}}

% % Set theory
\DeclareMathOperator{\ord}{\mathrm{Ord}}

% % Boolean algebra
\newcommand{\NEG}{\smallsetminus}
\newcommand{\upa}[1]{#1^{\uparrow}}

% % Field theory
\DeclareMathOperator{\tdeg}{t.deg}
\newcommand{\zmo}[2][p]{\Z/#1^{#2}\Z}

%% code from mathabx.sty and mathabx.dcl to get some symbols from mathabx
\DeclareFontFamily{U}{mathx}{\hyphenchar\font45}
\DeclareFontShape{U}{mathx}{m}{n}{
      <5> <6> <7> <8> <9> <10>
      <10.95> <12> <14.4> <17.28> <20.74> <24.88>
      mathx10
      }{}
\DeclareSymbolFont{mathx}{U}{mathx}{m}{n}
\DeclareFontSubstitution{U}{mathx}{m}{n}
\DeclareMathAccent{\widecheck}{0}{mathx}{"71}

% Arrows with text above and below with adjustable displacement
% (Stolen from Stackexchange)
\newcommandx{\yaHelper}[2][1=\empty]{
\ifthenelse{\equal{#1}{\empty}}
  % no offset
  { \ensuremath{ \scriptstyle{ #2 } } } 
  % with offset
  { \raisebox{ #1 }[0pt][0pt]{ \ensuremath{ \scriptstyle{ #2 } } } }  
}

\newcommandx{\yrightarrow}[4][1=\empty, 2=\empty, 4=\empty, usedefault=@]{
  \ifthenelse{\equal{#2}{\empty}}
  % there's no text below
  { \xrightarrow{ \protect{ \yaHelper[ #4 ]{ #3 } } } } 
  % there's text below
  {
    \xrightarrow[ \protect{ \yaHelper[ #2 ]{ #1 } } ]
    { \protect{ \yaHelper[ #4 ]{ #3 } } } 
  } 
}

% xcolor
\definecolor{darkgrey}{gray}{0.10}
\definecolor{lightgrey}{gray}{0.30}
\definecolor{slightgrey}{gray}{0.80}
\definecolor{softblue}{RGB}{30,100,200}

% hyperref
\hypersetup{
      colorlinks = true,
      linkcolor = {softblue},
      citecolor = {blue}
}

\newcommand{\link}[1]{\hypertarget{#1}{}}
\newcommand{\linkto}[2]{\hyperlink{#1}{#2}}

% Theorems

% % custom theoremstyles
\newtheoremstyle{definitionstyle}
{5pt}% above thm
{0pt}% below thm
{}% body font
{}% space to indent
{\bf}% head font
{\vspace{1mm}}% punctuation between head and body
{\newline}% space after head
{\thmname{#1}\thmnote{\,\,--\,\,#3}}

% % custom theoremstyles
\newtheoremstyle{propositionstyle}
{5pt}% above thm
{0pt}% below thm
{}% body font
{}% space to indent
{\bf}% head font
{\vspace{1mm}}% punctuation between head and body
{\newline}% space after head
{\thmname{#1}\thmnote{\,\,--\,\,#3}}

\newtheoremstyle{exercisestyle}%
{5pt}% above thm
{0pt}% below thm
{\it}% body font
{}% space to indent
{\scshape}% head font
{.}% punctuation between head and body
{ }% space after head
{\thmname{#1}\thmnote{ (#3)}}

\newtheoremstyle{remarkstyle}%
{5pt}% above thm
{0pt}% below thm
{}% body font
{}% space to indent
{\it}% head font
{.}% punctuation between head and body
{ }% space after head
{\thmname{#1}\thmnote{\,\,--\,\,#3}}

% % Theorem environments

\theoremstyle{definitionstyle}
\newmdtheoremenv[
    linewidth = 2pt,
    leftmargin = 20pt,
    rightmargin = 0pt,
    linecolor = darkgrey,
    topline = false,
    bottomline = false,
    rightline = false,
    footnoteinside = true
]{dfn}{Definition}
\newmdtheoremenv[
    linewidth = 2 pt,
    leftmargin = 20pt,
    rightmargin = 0pt,
    linecolor = darkgrey,
    topline = false,
    bottomline = false,
    rightline = false,
    footnoteinside = true
]{prop}{Proposition}
\newmdtheoremenv[
    linewidth = 2 pt,
    leftmargin = 20pt,
    rightmargin = 0pt,
    linecolor = darkgrey,
    topline = false,
    bottomline = false,
    rightline = false,
    footnoteinside = true
]{cor}{Corollary}
\newmdtheoremenv[
    linewidth = 2 pt,
    leftmargin = 20pt,
    rightmargin = 0pt,
    linecolor = darkgrey,
    topline = false,
    bottomline = false,
    rightline = false,
    footnoteinside = true
]{lem}{Lemma}


\theoremstyle{exercisestyle}
\newmdtheoremenv[
    linewidth = 0.7 pt,
    leftmargin = 20pt,
    rightmargin = 0pt,
    linecolor = darkgrey,
    topline = false,
    bottomline = false,
    rightline = false,
    footnoteinside = true
]{ex}{Exercise}
\newmdtheoremenv[
    linewidth = 0.7 pt,
    leftmargin = 20pt,
    rightmargin = 0pt,
    linecolor = darkgrey,
    topline = false,
    bottomline = false,
    rightline = false,
    footnoteinside = true
]{eg}{Example}
\newmdtheoremenv[
    linewidth = 0.7 pt,
    leftmargin = 20pt,
    rightmargin = 0pt,
    linecolor = darkgrey,
    topline = false,
    bottomline = false,
    rightline = false,
    footnoteinside = true
]{nttn}{Notation}

\theoremstyle{remarkstyle}
\newtheorem{rmk}{Remark}

% % footnotes
\renewcommand{\thempfootnote}{$\dagger$}
\MakePerPage{footnote}

% % tikzcd diagram 
\newenvironment{cd}{
    \begin{figure}[H]
    \centering
    \begin{tikzcd}
}{
    \end{tikzcd}
    \end{figure}
}

% tikzcd
% % Substituting symbols for arrows in tikz comm-diagrams.
\tikzset{
  symbol/.style={
    draw=none,
    every to/.append style={
      edge node={node [sloped, allow upside down, auto=false]{$#1$}}}
  }
}


\begin{document}
\title{Model Theory}
\author{JLH}
\maketitle






\subsection{Monster model}
\begin{nttn}
    We write $\Si(\dots,r)$ to mean the signature 
    with an extra relation symbol thrown in with the rest.
\end{nttn}

\begin{dfn}[$\ka$-monster \cite{hodges}]
    Let $\ka$ be a cardinal and $r$ 
    a relation symbol not in the signature $\Si$.
    We say a $\Si$-structure $\MM$ is a $\ka$-monster
    (also called big or splendid by Hodges) if it satisfies:
    for any subset $A \subs \MM$ 
    of cardinality $< \ka$ and any $\Si(A,r)$ structure $\NN$ that 
    such that $\NN$ and $\MM$ are elementarily equivalent \emph{in $\Si$}, 
    there is an interpretation of $\MM$ as a $\Si(A,r)$-structure 
    such that $\NN$ and $\MM$ are elementarily equivalent \emph{in $\Si(r)$}.
    \[\MM \equiv_\Si \NN \rightsquigarrow \MM \equiv_{\Si(r)} \NN\]
\end{dfn}

We want monsters to have the properties of being saturated, 
homogeneous and universal, 
which we define soon.
We show that monsters are 
saturated and that saturated structures are homogeneous and universal.

\begin{cd}
    & & \text{homogeneous}\\
    \text{monster} \ar[r]& \text{saturated} \ar[ur] \ar[dr]& \\
    & & \text{universal}\\
\end{cd}

\begin{ex}
    Let $\phi$ be a $\Si$-formula with $n$ free variables.
    There exists a $\Si$-formula that means 
    `there exists a unique tuple $x = (x_1,\dots,x_n)$ such that $\phi(x)$'.
    We write this as 
    \[\exists ! x, \phi(x)\]
\end{ex}

\begin{lem}[Monsters are saturated]
    Let the $\Si$-structure $\MM$ be a $\ka$-monster.
    Then it is $\ka$-saturated.
\end{lem}
\begin{proof}
    Let $A$ be a subset of $\MM$ with cardinality $< \ka$.
    Let $p \in S_n(\Th_\MM(A))$ be a maximal $n$-type.
    Since \linkto{finite_realisation_and_embeddings}{$p$ 
        is realised by $b \in \NN^n$ in some elementary 
        $\Si(A)$-extension $\NN$}
    we can make $\NN$ a $\Si(A,r)$-structure that interprets 
    $r$ as the singleton set $\set{b}$.
    We have $\MM \equiv_\Si \NN$ as $\NN$ is an elementary extension.
    Since $\MM$ is a $\ka$-monster and $\MM \equiv_\Si \NN$ 
    we have an interpretation of $r$ in $\MM$
    such that $\MM \equiv_{\Si(A,r)} \NN$.
    Hence 
    \[\NN \model{\Si(A,r)} \exists ! x, r(x) \implies 
    \MM \model{\Si(A,r)} \exists ! x, r(x) \]
    Taking this unique element $a \in \MM^n$ we see that 
    \[  
        p = \subintp{A}{\NN}{\tp}(b) = 
        \subintp{A}{\NN}{\tp}(\io(a)) 
        = \subintp{A}{\MM}{a}
    \]
    The last equality using the fact that $\NN$ is an elementary 
    $\Si(A)$-extension of $\MM$. 
    (Note that we might not have that it is an elementary 
    $\Si(A,r)$-extension)
    Hence $p$ is realised in $\MM$ and $\MM$ is $\ka$-saturated.
\end{proof}

\begin{dfn}[$\ka$-saturation (strong $\ka$-homogeneity)]
    \link{ka_saturation_dfn}
    Let $\ka$ be a cardinal and $\MM$ be a $\Si$-structure. 
    $\MM$ is $\ka$-saturated if for subset $A \subs \MM$
    such that $\abs{A} < \ka$
    every $n \in \N$ and every $p \in S_n(\Theory_\MM(A))$,
    $p$ is realised in $\MM$.

    Equivalently $\MM$ is $\ka$-saturated
    if for any $A \subs \NN$ in any $\Si$-structure $\NN$ 
    satisfying $\abs{A} < \ka$ either of the equivalent
    following hold:
    \begin{itemize}
        \item For any partial elementary embedding $f : A \to \MM$ and 
            any $b \in \NN$, 
            $f$ can be extended to a partial elementary $\Si$-embedding 
            $A \cup \set{b} \to \MM$.
        \item If $\MM \equiv_{\Si(A)} \NN$ then for any $b \in \NN$ we have 
            $\MM \equiv_{\Si(A,c)} \NN$ for some constant symbol $c$ that 
            is interpreted as $b$ in $\NN$.
    \end{itemize}

    The latter two definitions are equivalent due to the fact that 
    $\MM \equiv_{\Si(A)} \NN$ if and only if 
    there is a way to interpret symbols from $A$ in $\MM$ such that 
    \[\MM \model{\Si(A)} \phi \iff \NN \model{\Si(A)} \phi\] 
    for any $\Si(A)$-sentence $\phi$, 
    which is equivalent to the existance of a partial embedding $A \to \MM$. 
    The strong version implies the weak version by taking $\NN$ to be 
    $\MM$. 
    It remains to prove the equivalence with the first definition:
\end{dfn}
\begin{proof}
    \begin{forward}
        Let $f : A \to \MM$ be a partial elementary embedding,
        where $A \subs \NN$ and $\abs{A} < \ka$; let $b \in \NN$.
        We show that $\MM \equiv_{\Si(A,c)} \NN$
        for a constant symbol $c$ that is interpreted as $b$ in $\NN$
        which gives a way to extend $f$ over $A \cup \set{b}$.

        By \linkto{amalgamation}{amalgamation} there exists $\LL$ and 
        commuting elementary $\Si(A)$-embeddings $\io_\MM,\io_\NN$into 
        $\LL$ from $\MM$ and $\NN$.
        First we show that $\subintp{A}{\LL}{\tp}(\io_\NN(b))$ is consistent 
        with $\Theory_\MM(A)$.
        It suffices to show \linkto{compactness_for_types}{finite consistency:} 
        let $\De$ be a finite subset of $\Theory_\MM(A)$. 
        The formula
        \[\exists x, \bigand{\psi \in \De}{} \psi(x)\]
        is satisfied in $\LL$ by taking $\io_\NN(b)$ hence is satisfied in 
        $\MM$ as $\io_\MM$ is $\Si(A)$-elementary.
        Thus $\subintp{A}{\LL}{\tp}(\io_\NN(b))$ is consistent 
        with $\Theory_\MM(A)$.

        Hence, as $\MM$ is $\ka$-saturated we can find $a \in \MM$ such that 
        \[
            \subintp{A}{\LL}{\tp}(\io_\NN(b)) = 
            \subintp{A}{\MM}{\tp}(a)
        \]

        Add a constant symbol $c$ to the signature such that 
        $\nnintp{c} = b$ and $\mmintp{c} = a$.
        Then $x = c$ is in both types and so $\io_\NN(b) = \io_\MM (a)$.
        Finally we have 
        \[\NN \equiv_{\Si(A,c)} \LL \equiv_{\Si(A,c)} \MM\]
        since $\NN \to \LL$ and $\MM \to \MM$ are elementary.
    \end{forward}

    \begin{backward}
        If $p \in S_n(\Theory_\MM(A))$ then \linkto{}{it is realised 
        in some elementary $\Si$-extension} $\io : \MM \to \NN$
        by $b \in \NN$.
        Hence we have a partial elementary $\Si$-embedding 
        $\io^{-1} : \io(\MM) \to \MM$ that can be extended to have domain 
        $\io^{-1} \cup \set{b}$ by assumption.
        Hence we have the image of $b$ under this map as the element of $\MM$ 
        realising the type $p$.
    \end{backward}
\end{proof}

A special case of the second definition of 
saturation comes in the form of homogeneity.
\begin{dfn}[$\ka$-homogeneity]
    Let $\ka$ be a cardinal and $\MM$ a $\Si$-structure.
    $\MM$ is $\ka$-homogeneous 
    if for any $A \subs \MM$, 
    any \linkto{partial_morph_dfn}{partial elementary embedding} 
    $f : A \to \MM$ and any $b \in \MM$, 
    if $\abs{A} < \ka$ then 
    $f$ can be extended to a partial elementary embedding 
    $A \cup \set{b} \to \MM$.
\end{dfn}

\begin{dfn}[$\ka$-universality]
    Let $T$ be a $\Si$-theory and let $\MM \model{\Si} T$.
    Then $\MM$ is a $\ka$-universal $\Si$-model of $T$ if 
    any $\NN \model{\Si} T$ such that $\abs{\NN} < \ka$ has an elementary 
    embedding into $\MM$.

    We replace $\ka$ with $\ka^+$ when the $<$ is improved to $\leq$.
    %?Maybe for each of them
\end{dfn}

\begin{lem}[Saturated structures are universal \cite{marker}]
    Let $\MM$ be a $\ka$-saturated $\Si$-structure.
    Then $\MM$ is a $\ka^+$-universal $\Si$-structure.
\end{lem}
\begin{proof}
    Let $\NN_\be$ be a $\Si$-structure such that 
    $\abs{\NN_\be} = \be \leq \ka$.
    Then enumerate $\NN_\be = \set{n_\ga \st \ga \leq \be}$
    and define nested subsets $\NN_\al := \set{n_\ga \st \ga \leq \al}$
    for each $\al \leq \be$.
    We then use \linkto{transfinite_induction}{transfinite induction} to define 
    partial elementary embeddings $f_\al : \NN_\al \to \MM$ 
    for $\al \leq \be$ such that they mutually agree upon restriction.

    If $\al$ is a limit ordinal and for each $\ga \leq \al$ we have 
    a well-defined partial elementary embedding $f_\ga : \NN_\ga \to \MM$ 
    that agree upon restriction
    then $f_\al := \bigcup_{\ga < \al} f_\ga$ is well-defined.

    To define $f_{\al+1}$ using $f_\al$ we first define 
    \[\Ga = \set{\phi(v,f_\al(a)) \in \form{\Si(f_\al(\NN_\al))} \st
    \exists a \in \NN^\star, \NN \model{\Si} \phi(n_\al,a))}\]
    $\Ga$ is finitely consistent with $\Theory_\MM(f_\al(\NN_\al))$:
    Gor any finite subset $\De \subs \Ga$,
    we can take the conjuction,
    \[\exists v, \bigand{\phi(v,f_\al(a)) \in \De}{} \phi(v,a)\]
    Note that this is satisfied by $\NN$ therefore by the partial elementary 
    $\Si$-embedding $f_\al$:
    \[\MM \model{\Si} \exists v, 
    \bigand{\phi(v,f_\al(a)) \in \De}{} \phi(v,f_\al(a))\]
    Hence $\MM \model{\Si} \De(b)$ for some $b \in \MM$.
    By \linkto{compactness_for_types}{compactness for types}
    $\Ga$ is consistent with $\Theory_\MM(f_\al(\NN_\al))$.
    This implies that it 
    \linkto{extend_to_maximal_type_zorn}{can be extended} to a maximal $1$-type 
    $p \in S_1(\Theory_\MM(\NN_\al))$.
    By $\ka$-saturation, $\Ga(v)$ is realised by some 
    $b \in \MM$.
    Define $f_{\al + 1} : n_\al \mapsto b$ and agreeing with $f_\al$ otherwise.
    By definition $f_{\al + 1}$ is a partial elementary $\Si$-embedding.

    Hence $f_\be : \NN_\be \to \MM$ is an elementary $\Si$-embedding.
\end{proof}

\begin{dfn}[Monster model]
    Let $T$ be an infinitely modelled $\Si$-theory.
    Then for any infinite cardinal $\ka$ there exists a `Monster model' of $T$: 
    a $\Si$-model $\M_{\ka^+}$ such that 
    \begin{itemize}
        \item $\ka \leq \abs{\M_{\ka^+}}$
        \item $\M_{\ka^+}$ is $\ka$-saturated
        \item For any $\NN \model{\Si} T$ such that $\abs{\NN} < \ka$ there is 
            an elementary $\Si$-embedding $\NN \to \M_{\ka^+}$.
            For any model $\MM$ of $T$ and elementary $\Si$-embeddings 
            $\MM \to \M$ and $\MM \to \NN$ the following commutes 
            \begin{cd}
                \MM \ar[r] \ar[dr] &\NN \ar[d]\\
                &\M
            \end{cd}
    \end{itemize}
\end{dfn}
\begin{proof}
    %?
\end{proof}



%? Strongly minimal, algebraic closure
\begin{dfn}[Algebraic, algebraic closure]
    Let $\MM$ be a $\Si$-structure and let $D$ be a subset of $\MM$
    (often \linkto{strongly_minimal}{strongly minimal}strongly minimal).
    Let $A$ be a subset of $D$, 
    $a \in \MM$ is algebraic over $A$ if $a$ belongs to a finite 
    $\Si(A)$-definable set .
    Define the algebraic closure of $A$ over $D$ to be
    \[\acl_D(A) := \set{a \in D \st a \text{ is algebraic over } A}\]
    We drop the subscript $D$ when it is sufficiently obvious.
\end{dfn}

\begin{lem}[Some definable sets]
    \link{some_definable_sets}
    Let $\MM$ be a $\Si$-structure and let $B,C \subs \MM$ be a $\Si$-definable
    set.
    Let $\phi(x)$ be a $\Si$-formula with $n$ free variables.
    For $b \in B^n$, let $\psi(x,b)$ be a $\Si(B)$-formula 
    with $m$ free variables ($\psi(x,y)$
    is a $\Si$-formula with $n+m$ free variables).

    Then the following sets are definable by a $\Si$-formula:
    \begin{itemize}
        \item The intersection of $B$ and $C$, the union of $B$ and $C$
            the complement of $B$.
        \item The set of $b \in B^n$ that satisfy $\phi(x)$:
            \[\set{b \in B^n \st \MM \model{\Si} \phi(b)}\]
        \item The elements $b \in \MM^n$ such that $\psi(x,b)$ defines a set of 
            at most $k$ elements.
        \item The elements of $\MM^n$ such that $\psi(x,b)$ defines a set of 
            at least $k$ elements.
        \item The elements of $\MM^n$ such that $\psi(x,b)$ defines a set of 
            cardinality $k$.
            \[\set{b \in \MM^n \st \abs{\psi(\MM,b)} = k}\]
            and even $\set{b \in B^n \st \abs{\psi(\MM,b)} = k}$ by 
            taking the intersection of two definable sets.
    \end{itemize}
    We will become lazier when dealing with definable sets as we gain an idea 
    of what should and should not be definable.
\end{lem}
\begin{proof}
    \begin{itemize}
        \item This is clear.
        \item 
        Since $B$ is $\Si$-definable we can take $\chi(x)$ as the 
        $\Si$-formula defining $B$
        and consider the $\Si$-formula 
        \[\phi(x_1,\dots,x_n) \AND \bigand{i = 1}{n} \chi(x_i)\]
        Clearly this defines $\set{b \in B^n \st \MM \model{\Si} \phi(b)}$.
        \item 
        To make $\set{b \in \MM^n \st \abs{\psi(\MM,b)} \leq k}$
        we take the $\Si$-formula $\chi(x)$:
        \[
            \chi(x) = \bigforall{i = 1}{k + 1} x_i,
            \bigand{i = 1}{k + 1} \psi(x_i,y) \to \bigor{i \ne j}{x_i = x_j}
        \]
        where potentially $x_i$ represents $m$ variables, which we can 
        quantify over as it is finite.
        \item 
        To make $\set{b \in \MM^n \st k \leq \abs{\psi(\MM,b)}}$
        we take the $\Si$-formula $\chi(x)$:
        \[
            \chi(x) = \bigexists{i = 1}{k} x_i, {x_i \ne x_j}
        \]
    \end{itemize}
\end{proof}

\begin{dfn}[Minimal, strongly minimal \cite{marker}]
    \link{strongly_minimal}
    Let $\MM$ be a $\Si$-structure.
    Let $D$ be an infinite $\Si$-definable subset of $\MM^n$.
    $D$ is minimal in $\MM$ if any $\Si$-definable subset of $D$
    if finite or cofinite.
    $D$ is strongly minimal if it is minimal in 
    $\NN$ for any elementary extension $\NN$ of $\MM$.
    A $\Si$-theory $T$ is strongly minimal if any $\Si$model of $T$
    is strongly minimal 
    (note that any $\Si$-structure is definable by the formula $v = v$).
\end{dfn}

\begin{prop}[Algebraic closure is a \linkto{pregeometry_dfn}{pregeometry}]
    \link{acl_d_is_pregeometry}
    Let $\MM$ be a $\Si$-structure. 
    Let $D$ be a minimal subset of $\MM$.
    Then $(D,\acl_D(\star)$ is almost a pregeometry.
\end{prop}
\begin{proof}[\cite{fandom}]
    Preserves order: 
    \emph{if $A \subs B \subs D$ then $\acl_D(A) \subs \acl_D(B)$.}
    Let $a \in \acl_D(A)$. 
    Then there exists a finite $\Si(A)$-definable set containing $a$.
    Any $\Si(A)$-formula is naturally a $\Si(B)$-formula thus $a \in \acl_D(B)$.

    Idempotence: \emph{for any $A \subs D$, $\acl_D(A) = \acl_D(\acl_D(A))$.}
    \begin{forward}
        We first show that for any subset $A \subs D$, $A \subs \acl_D(A)$.
        Let $a \in A$ then $a = x$ is a $\Si(A)$-formula that is defines a 
        finite set. 
        Thus $a \in \acl_D(A)$.
        Directly we have the corollary $\acl_D(A) \subs \acl_D(\acl_D(A))$.
    \end{forward}

    \begin{backward}
        We show that $\acl_D(\acl_D(A)) \subs \acl_D(A)$.
        Let $a \in \acl_D(\acl_D(A))$.
        Then there exists $\phi(x,v) = \phi(x,v_0,\dots,v_n)$ a $\Si$-formula 
        and $b_0,\dots, b_n \in \acl_D(A)$ such that $\phi(x,b)$
        defines a finite subset of $\MM$ containing $a$.
        Let $k$ be the finite cardinality of $\phi(\MM,b)$.
        \linkto{some_definable_sets}{There exists a $\Si$-formula $\psi(v)$} 
        that defines the set 
        $\set{b \in B \st \abs{\phi(\MM,b)} \leq n}$
        \[
            \phi'(x,v) := \phi(x,v) \AND \psi(v)
        \]
        We have that $a \in \phi(\MM,b) = \phi'(\MM,b)$
        and for any $c \in \MM^n$, $\phi'(\MM,c)$ is finite.
    
        For each $b_i$ appearing in $b$
        there exists a $\Si(A)$-formula $\psi_i(v_i)$ such that 
        $b_i \in \psi_i(\MM)$ and this definable set is finite.
        Define the $\Si(A)$-formula
        \[
            \phi''(x) := \bigexists{i = 1}{n} v_i,
            \phi'(x,v_0,\dots,v_n) \AND \bigand{i = 1}{n} \psi_i(v_i)
        \]
        Then $a \in \phi''(\MM)$ by taking the $v_i$ to be $b_i$ and 
        \begin{align*}
            d \in \phi''(\MM) 
            &\implies \exists c \in \MM^n, \MM \model{\Si}\phi'(d,c) 
            \text{ and for each $i$,} \MM \model{\Si}\psi_i(c_i)\\
            &\implies \text{there exist for each $i$ } c_i \in \psi_i(\MM),
            \MM \model{\Si}\phi'(d,c) \\
            &\implies d \in 
            \bigcup_{i = 0}^n \bigcup_{c_i \in \psi_i(\MM)} \phi'(\MM,c)
        \end{align*}
        The last expression is a finite union of finite sets which is finite.
        Hence $\phi''(\MM)$ is finite and $a \in \acl_D(A)$
    \end{backward}
    
    Finite character: \emph{if $A \subs D$ and $a \in \acl_D(A)$ then 
    there exists a finite subset $F \subs A$ such that $a \in \acl_D(F)$.}
    Take the $\Si(A)$-formula defining the finite set containing $a$.
    Pick out the (finitely many) constant symbols from $A$, 
    forming a finite subset $F \subs A$.
    Then $a \in \acl_D(F)$.

    Exchange: \emph{if $A \subs D$ and $a,b \in D$ such that 
    $a \in \acl_D(A,b)$ (shorthand for $A,\set{a}$)
    then $a \in \acl_D(A)$ or $b \in \acl_D(A,a)$.}
    Since $a \in \acl(A,b)$ there exists a $\Si(A)$-formula $\phi(v,w)$ such 
    that $a \in \phi(\MM,b)$ and $\phi(\MM,b)$ is finite - 
    say it has cardinality $n$ 
    (if $b$ does not appear in the formula then we immediately have 
    $a \in \acl(A)$).
    \linkto{some_definable_sets}{There exists a $\Si(A)$-formula} $\psi(w)$
    defining the set
    \[\psi(\MM) = \set{b' \in D \st n = \abs{\phi(\MM,b')}}\]
    As $\psi(\MM) \subs D$ and $D$ is minimal, 
    $\psi(\MM)$ is finite or cofinite.
    If it is finite then $b \in \psi(\MM)$ and so 
    $b \in \acl(A) \linkto{acl_d_is_pregeometry}{\subs} \acl(A,a)$.

    If it is $\psi(\MM)$ then consider the $\Si(A)$-formula 
    $\phi(v,w) \AND \psi(w)$.
    For each $a' \in D$ let $X(a')$ be the subset of $D$ defined by 
    $\phi(a',w) \AND \psi(w)$.
    Consider $b \in X(a)$, and case on whether it is finite or cofinite.
    If it is finite then $b \in \acl(A,a)$ as $\phi(a,w) \AND \psi(w)$
    is a $\Si(A)$-formula defining a finite set.

    If $X(a)$ is cofinite then let $m = \abs{D \setminus X(a)} \in \N$.
    \linkto{some_definable_sets}{There exists a $\Si(A)$-formula} $\chi(v)$
    defining the set
    \[\chi(\MM) = \set{a' \in D \st m = \abs{D \setminus X(a')}}\]
    If $\chi(\MM)$ is finite then $a \in \chi(\MM)$ and so $a \in \acl(A)$.
    If $\chi(\MM)$ is confinite then there exist $n + 1$ distinct elements 
    $a_i \in \chi(\MM)$ since $D$ is infinite by definition.
    Take the (finite) intersection of the cofinite $X(a_i)$,
    producing a non-empty (infinite) set.
    Take 
    \[b' \in \bigcap_{i = 1}^{n+1} X(a_i) = 
    \bigcap_{i = 1}^{n+1} \phi(a_i,\MM) \cap \psi(\MM)\] 
    Then for each $i$, $\MM \model{\Si} \phi(a_i,b')$, 
    hence $n + 1 \leq \abs{\phi(\MM,b')}$.
    However $\MM \model{\Si} \psi(b')$ implies $n = \abs{\phi(\MM,b')}$,
    a contradiction.
\end{proof}
\begin{rmk}
    The definition of \linkto{dimension_dfn}{dimension} 
    for pregeometries thus carries through for 
    subsets of $D$.
\end{rmk}



\printbibliography

\end{document} 